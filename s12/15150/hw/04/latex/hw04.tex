\documentclass[11pt,letterpaper]{article}

\usepackage{amsmath}
\usepackage{amssymb}
\usepackage{fancyhdr}

\oddsidemargin0cm
\topmargin-2cm
\textwidth16.5cm
\textheight23.5cm

\newcommand{\question}[2] {\vspace{.25in} \hrule\vspace{0.5em}
\noindent{\bf #1: #2} \vspace{0.5em}
\hrule \vspace{.10in}}
\renewcommand{\part}[1] {\vspace{.10in} {\bf (#1)}}

\newcommand{\myname}{Karan Sikka}
\newcommand{\myandrew}{ksikka@andrew.cmu.edu}
\newcommand{\myhwnum}{4}

\setlength{\parindent}{0pt}
\setlength{\parskip}{5pt plus 1pt}

\pagestyle{fancyplain}
\lhead{\fancyplain{}{\textbf{HW\myhwnum}}}
\rhead{\fancyplain{}{\myname\\ \myandrew}}
\chead{\fancyplain{}{15-150}}
%% SML font
\newcommand{\sml}[1]{\texttt{#1}}

%% Evaluation and step judgements
\newcommand{\eval}[2]{{#1} \Longrightarrow {#2}}
\newcommand{\evalto}{\ensuremath{\Longrightarrow}}
\newcommand{\stepseq}[2]{{#1} \longmapsto {#2}}
\newcommand{\nstepseq}[2]{{#1} \longmapsto^* {#2}}

%% SML Syntax in math mode
\newcommand{\msml}[1]{\mathtt{#1}}

\newcommand{\mval}{\msml{v}}
\newcommand{\mvalp}{\msml{v'}}
\newcommand{\mvalpp}{\msml{v''}}
\newcommand{\mconst}{\msml{c}}
\newcommand{\mconsts}[1]{\msml{c_{#1}}}
\newcommand{\mvals}[1]{\msml{v_{#1}}}
\newcommand{\mcase}[5]{\msml{case}~{#1}~\msml{of}~{#2} => {#3}~|~{#4} => {#5}}
\newcommand{\mapp}[2]{{#1}~{#2}}
\newcommand{\mpair}[2]{({#1}, {#2})}
\newcommand{\mtriple}[3]{({#1}, {#2}, {#3})}
\newcommand{\mnil}{\msml{nil}}
\newcommand{\mcons}[2]{{#1}::{#2}}
\newcommand{\mnat}[1]{\msml{\overline{#1}}}
\newcommand{\mrtimes}[2]{{#1} ** {#2}}
\newcommand{\mrfrac}[2]{{#1} // {#2}}
\newcommand{\asc}[2]{{#1}\ :\ {#2}}
\newcommand{\smlarr}[2]{{#1}\ ->\ {#2}}
\newcommand{\econs}[2]{{#1}\ ::\ {#2}}
\newcommand{\tint}{\msml{int}}
\newcommand{\tstr}{\msml{string}}
\newcommand{\tchr}{\msml{char}}
\newcommand{\tbool}{\msml{bool}}
\newcommand{\tfso}{\msml{fsobject}}
\newcommand{\tlst}[1]{{#1}~\msml{list}}
\newcommand{\ttree}[1]{{#1}~\msml{tree}}
\newcommand{\tarr}[2]{{#1} \to {#2}}
\newcommand{\tprod}[2]{{#1} * {#2}}
\newcommand{\tnf}{\msml{nf}}
\newcommand{\oftp}[2]{{#1}\ :\ {#2}}

\newcommand{\oftpm}[2]{\msml{{#1}}\ :\ {#2}}
\newcommand{\ce}[2]{{#1} \cong {#2}}
\newcommand{\mappm}[2]{{\msml{#1}}~{\msml{#2}}}
\newcommand{\vnil}{\texttt{[]}}

%% Propositional calculus
\newcommand{\pand}{\wedge}
\newcommand{\por}{\vee}
\newcommand{\pimp}{\supset}



\begin{document}

\medskip

\thispagestyle{plain}
\begin{center}                  % Center the following lines
{\Large 15-150 Assignment \myhwnum} \\
\myname \\
\myandrew \\
G \\
Due 2/15/12 \\
\end{center}


\question{1}{Task 3.1}

{\bf Claim:} For all values \sml{n:int} and \sml{l:int list},
 if \sml{l sorted} then \sml{insert(n,l) sorted}.

We proceed by induction on \sml{l}.

{\bf Case \sml{[]}:}

WTS that if \sml{[] sorted} then \sml{insert(n,[]) sorted}.

Given that \sml{[]} is sorted, consider:

\begin{align*}
  & \sml{insert(n,[])} && \\
  \cong & \sml{case [] => n :: [] | ...} && step\\
  \cong & \sml{n :: []} && step \\
  \cong & \sml{[n]} && step \\
\end{align*}

Since \sml{[n]} is a singleton list, it is sorted by defintion. QED.

{\bf Case \sml{x::xs}:}

Want to show that if \sml{x::xs sorted} then \sml{insert(n,x::xs) sorted}.

Inductive Hypothesis:

Assume that for some int list xs, if \sml{xs sorted} then \sml{insert(n,xs)} is also sorted.\\

Inductive Step:

Want to show that if the claim holds for xs, it holds for x::xs.

Consider \sml{x::xs}. If it is not sorted, the claim is true.

Consider \sml{insert(n,x::xs)} where \sml{x::xs} is sorted.

\begin{align*}
\cong & \sml{insert(n,x::xs)} && \\
\cong & \sml{case x::xs of [] => ... | (x :: xs) => (case n < x of true ... } && step \\
\cong & \sml{case n < x of true => n :: (x :: xs) | false => x :: (insert (n , xs)) } && step \\
\end{align*}

Now, \sml{n < x} can be either true or false. Case true:

\begin{align*}
\cong & \sml{n :: (x :: xs)} && step \\
\end{align*}

This result is sorted, because \sml{x::xs} is sorted, and \sml{n < x}. Now, case false:

\begin{align*}
\cong & \sml{x :: (insert (n , xs))} && step \\
\end{align*}

\sml{insert(n,xs)} is sorted by the IH, so therefore, the result is sorted by the definition of sortedness.

In all cases, the inductive step holds.

$\blacksquare$

\question{2}{Task 3.2}
{\bf Claim:} For all valuable expressions \sml{e:int} and \sml{es:int} list, \\
if \sml{es sorted} then \sml{insert(e,es) sorted}.

All valuable expressions \sml{e} and \sml{es} will evaluate to values \sml{x} and \sml{xs}. Then, 
by the theorem proved in Task 3.1, if if \sml{x sorted} then \sml{insert(x,xs) sorted}.
Since, \sml{x} $\cong$ \sml{e}, and \sml{xs} $\cong$ \sml{es}, the claim holds.

\question{3}{Task 3.3}
{\bf Claim:} For all values \sml{l:int list}, \sml{(isort l) sorted}.

We proceed by induction on l.

{\bf Base Case:}\\
WTS \sml{isort []} is sorted.\\

\begin{align*}
      & \sml{isort []} && \\
\cong & \sml{case [] of [] => [] | ...} && step \\
\cong & \sml{[]} && step \\
\end{align*}
This is sorted by definition, therefore \sml{isort []} is sorted.

Inductive Hypothesis:

Assume \sml{isort xs} is sorted, for some \sml{xs}.

Inductive Step:

WTS \sml{isort x::xs} is sorted.

Consider:

\begin{align*}
     & \sml{isort x::xs} && step \\
\cong & \sml{case x::xs of [] => [] | (x :: xs) => insert (x , isort xs)} && step \\
\cong & \sml{insert (x , isort xs)} && step \\
\end{align*}
isort xs is sorted, by the IH.
By the lemma proved in Task 3.1, this is also sorted.

\question{4}{Task 4.4}
Claim: For all values \sml{t1 t2:tree},
\sml{depth (combine(t1,t2)) $\leq$ 1+max(depth t1, depth t2)}.

{\bf Base Case:}

Want to show that \sml{depth (combine(Empty,t2)) $\leq$ 1+max(depth Empty, depth t2)}.

Consider:

\begin{align*}
      & \sml{depth (combine(Empty,t2))} && step \\
\cong & \sml{depth (case Empty of Empty => t2 | Node(l,x,r) => ...} && step \\
\cong & \sml{depth t2} && step \\
\end{align*}

\sml{depth t2} is less than \sml{1+depth(t2)}, so the claim holds for the base case.

{\bf Inductive Hypothesis:}
?
{\bf Inductive Step:}
?
\question{5}{Task 5.2}
\begin{enumerate}
\item $W_{takeanddrop}(0) = k_0$ \\
$W_{takeanddrop}(d) = k_0 + W_{takeanddrop}(d-1)$ \\
This recurrance is $O(d)$ because it expands to the sum of d constants.
\item $S_{takeanddrop}(0) = k_0$ \\
$S_{takeanddrop}(d) = k_0 + S_{takeanddrop}(d-1)$ \\
This recurrance is $O(d)$ because there is no room for parallelism. Each step 
depends on something in a sequential manner. 
\item $W_{halves}(d) \leq 2 W_{takeanddrop}(d)$ \\
This recurrance is $O(d)$.
\item $S_{halves}(d) = max(S_{takeanddrop}(d),S_{takeanddrop}(d)) = S_{takeanddrop}(d)$ \\
This recurrance is $O(d)$.
\item $W_{rebalance}(0) = k_0 \\
W_{rebalance}(n) = k_0 + W_{halves}(n) + 2 W_{rebalance}(\frac{n}{2}) \\
W_{rebalance}(n) = k_0 + log(n) + 2 W_{rebalance}(\frac{n}{2})$ \\
By the tree method, the right-most term can be reformulated to be:\\
$W_{rebalance}(n) = k_0 + log(n) + \sum_{i=1}^{log n}k_0$ \\
$W_{rebalance}(n) = k_0 + log(n) + k_0 log(n)$ \\
Therefore, the bound for this is $O(log(n))$.
\item $S_{rebalance}(0) = k_0 \\
S_{rebalance}(n) = k_0 + S_{halves}(n) + max( W_{rebalance}(\frac{n}{2}), W_{rebalance}(\frac{n}{2}) )  \\
S_{rebalance}(n) = log(n) + W_{rebalance}(\frac{n}{2}) 
S_{rebalance}(n) = log(n) + \sum_{i=1}^{log(n)}n \\
S_{rebalance}(n) = log(n) + log(n) $
This is bounded by $log(n)$.
\end{enumerate}

\end{document}
