\documentclass[11pt,letterpaper]{article}

\usepackage{amsmath}
\usepackage{amssymb}
\usepackage{fancyhdr}

\oddsidemargin0cm
\topmargin-2cm
\textwidth16.5cm
\textheight23.5cm

\newcommand{\question}[2] {\vspace{.25in} \hrule\vspace{0.5em}
\noindent{\bf #1: #2} \vspace{0.5em}
\hrule \vspace{.10in}}
\renewcommand{\part}[1] {\vspace{.10in} {\bf (#1)}}

\newcommand{\myname}{Karan Sikka}
\newcommand{\myandrew}{ksikka@andrew.cmu.edu}
\newcommand{\myhwnum}{5}

\setlength{\parindent}{0pt}
\setlength{\parskip}{5pt plus 1pt}

\pagestyle{fancyplain}
\lhead{\fancyplain{}{\textbf{HW\myhwnum}}}
\rhead{\fancyplain{}{\myname\\ \myandrew}}
\chead{\fancyplain{}{15-150}}
%% SML font
\newcommand{\sml}[1]{\texttt{#1}}

%% Evaluation and step judgements
\newcommand{\eval}[2]{{#1} \Longrightarrow {#2}}
\newcommand{\evalto}{\ensuremath{\Longrightarrow}}
\newcommand{\stepseq}[2]{{#1} \longmapsto {#2}}
\newcommand{\nstepseq}[2]{{#1} \longmapsto^* {#2}}

%% SML Syntax in math mode
\newcommand{\msml}[1]{\mathtt{#1}}

\newcommand{\mval}{\msml{v}}
\newcommand{\mvalp}{\msml{v'}}
\newcommand{\mvalpp}{\msml{v''}}
\newcommand{\mconst}{\msml{c}}
\newcommand{\mconsts}[1]{\msml{c_{#1}}}
\newcommand{\mvals}[1]{\msml{v_{#1}}}
\newcommand{\mcase}[5]{\msml{case}~{#1}~\msml{of}~{#2} => {#3}~|~{#4} => {#5}}
\newcommand{\mapp}[2]{{#1}~{#2}}
\newcommand{\mpair}[2]{({#1}, {#2})}
\newcommand{\mtriple}[3]{({#1}, {#2}, {#3})}
\newcommand{\mnil}{\msml{nil}}
\newcommand{\mcons}[2]{{#1}::{#2}}
\newcommand{\mnat}[1]{\msml{\overline{#1}}}
\newcommand{\mrtimes}[2]{{#1} ** {#2}}
\newcommand{\mrfrac}[2]{{#1} // {#2}}
\newcommand{\asc}[2]{{#1}\ :\ {#2}}
\newcommand{\smlarr}[2]{{#1}\ ->\ {#2}}
\newcommand{\econs}[2]{{#1}\ ::\ {#2}}
\newcommand{\tint}{\msml{int}}
\newcommand{\tstr}{\msml{string}}
\newcommand{\tchr}{\msml{char}}
\newcommand{\tbool}{\msml{bool}}
\newcommand{\tfso}{\msml{fsobject}}
\newcommand{\tlst}[1]{{#1}~\msml{list}}
\newcommand{\ttree}[1]{{#1}~\msml{tree}}
\newcommand{\tarr}[2]{{#1} \to {#2}}
\newcommand{\tprod}[2]{{#1} * {#2}}
\newcommand{\tnf}{\msml{nf}}
\newcommand{\oftp}[2]{{#1}\ :\ {#2}}

\newcommand{\oftpm}[2]{\msml{{#1}}\ :\ {#2}}
\newcommand{\ce}[2]{{#1} \cong {#2}}
\newcommand{\mappm}[2]{{\msml{#1}}~{\msml{#2}}}
\newcommand{\vnil}{\texttt{[]}}

%% Propositional calculus
\newcommand{\pand}{\wedge}
\newcommand{\por}{\vee}
\newcommand{\pimp}{\supset}



\begin{document}

\medskip

\thispagestyle{plain}
\begin{center}                  % Center the following lines
{\Large 15-150 Assignment \myhwnum} \\
\myname \\
\myandrew \\
G \\
2/25/12 \\
\end{center}


\question{1}{Task 2.2}
\textbf{Claim:} For all \texttt{l1: $\alpha$ list} and all \texttt{l2: 
$\alpha$ list list}, 
  \begin{center}
    \texttt{ap(l1, concatap l2) $\cong$ concatap(l1 :: l2)}
  \end{center}

\textbf{Proof:}\\
\begin{align*}
     & \texttt{concatap(l1::l2)} &\\ 
\cong& \texttt{case l1::l2 of [] => [] | x::xs => ap(x,concatap(xs))} & \text{step}\\
\end{align*}
At this point, l1::l2 can be either be \texttt{nil} or \texttt{x::xs}. We will show 
both cases.

Case \texttt{nil}:\\
(both \texttt{l1} and \texttt{l2} are \texttt{nil}) since \texttt{[]::[] = []}:
\begin{align*}
      & \texttt{case [] of [] => [] | x::xs => ap(x,concatap(xs))} & \text{def of concatap}\\
\cong & \texttt{[]} & \text{step}\\
\cong & \texttt{l2} & \text{value of l2}\\
\cong & \texttt{case [] of [] =>l2 | ...} & \text{rev. step from ap}\\
\cong & \texttt{ap([],[])} & \text{rev. step, def of ap}\\
\cong & \texttt{ap([],case [] of [] => [])} & \text{ rev. step, def of concatap}\\
\cong & \texttt{ap([],concatap([])} & \text{rev. step from concatap}\\
\cong & \texttt{ap(l1,concatap(l2)} & \text{values of l1, l2}\\
\end{align*}

Case \texttt{x::xs}:
\begin{align*}
\cong& \texttt{case l1::l2 of [] => [] | x::xs => ap(x,concatap(xs))} & \text{def of concatap}\\
\cong& \texttt{ap(l1,concatap(l2))} & \text{reverse step}\\
\end{align*}
In both cases, the lemma holds. QED.

\question{2}{Task 2.3}
\texttt{Claim:} For all \texttt{l : $\alpha$ list list},
\begin{center}
\texttt{concat l $\cong$ concatap l}
\end{center}

\textbf{Proof:} We proceed by structural induction on \sml{l}

\textbf{Case []:}
Want to show: \texttt{concat [] $\cong$ concatap []}
Consider:
\begin{align*}
        & \texttt{concat []} & \text{step}\\
  \cong & \texttt{case [] of [] => [] ...} & \text{step}\\
  \cong & \texttt{[]} & \text{step}\\
  \cong & \texttt{case [] of [] => [] ...} & \text{reverse step} \\
  \cong & \texttt{concatap []} & \text{reverse step} \\
\end{align*}

\textbf{Case x::xs:}
Inductive Hypothesis: \texttt{concat xs $\cong$ concatap xs}
Inductive Step:


Consider:
\begin{align*}
      & \texttt{concat x::xs} & \\
\cong & \texttt{case x::xs of ... | x::xs => ...} & \text{step}\\
\cong & \texttt{case x of [] => concat xs | y::ys => y::concat (ys::xs) )} & \text{step}\\
\end{align*}
In the list of lists xs, the first element can be the empty list, or a non-empty list.
We will consider both cases.
Case \texttt{x = []}:
\begin{align*}
\cong & \texttt{case [] of [] => concat xs | y::ys => y::concat (ys::xs) )} & \text{cont. from above}\\
\cong & \texttt{concat xs} & \text{step}\\
\cong & \texttt{concatap xs} & \text{IH}\\
\cong & \texttt{ap([],concatap xs)} & \text{Lemma 2}\\
\cong & \texttt{case []::xs of ... | x::xs => ap(x,concatap xs)} & \text{rev. step}\\
\cong & \texttt{concatap []::xs} & \text{rev. step}\\
\end{align*}

Case \texttt{x = y::ys}:
\begin{align*}
\cong & \texttt{case y::ys of [] => concat xs | y::ys => y::concat (ys::xs) )} & \text{cont. from above}\\
\cong & \texttt{ y::concat (ys::xs) } & \text{step}\\
\cong & \texttt{ y::(case ys::xs of ... | x::xs => case x of ... )} & \text{step, evaluate the case}\\
\cong & \texttt{ y::(case ys of [] => concat xs | z::zs =>... )} & \text{step}\\
\end{align*}
Here we run into another fork in the road. We said x = y::ys, but this function
evaluates differently based on whether or not ys is nil. So we will consider both cases, 
since both are possible.
Case \texttt{x = y::[]}
\begin{align*}
\cong & \texttt{ y::(case [] of [] => concat xs | z::zs =>... )} & \text{cont. from above}\\
\cong & \texttt{ y::(concat xs)} & \text{step}\\
\cong & \texttt{ y::(concatap xs)} & \text{IH}\\
\cong & \texttt{ y::(ap([],concatap xs))} & \text{Lemma 2}\\
\cong & \texttt{ case y::[] of [] => l2 | x::xs => x::(ap(xs,l2))} & \text{rev. step, from ap}\\
\cong & \texttt{ ap(y::[],concatap(xs))} & \text{rev. step}\\
\cong & \texttt{ ap(x,concatap(xs)) } & \text{x=y::[]}\\
\cong & \texttt{ case x::xs of [] => [] | x::xs => ap(x,concatap(xs)) } & \text{rev. step (concatap)}\\
\cong & \texttt{ concatap (x::xs) } & \text{rev. step}\\
\end{align*}
Here we see that the lemma holds for the subcase when x = y::[].

Case \texttt{x = y::ys}:
\begin{align*}
\cong & \texttt{ y::(case ys of ... | z::zs => z::(concat... )} & \text{cont. from above}\\
\cong & \texttt{ y::(z::(concat(zs::ys))} & \text{step}\\
\cong & \texttt{ y::(z::(concat(zs::ys))} & \text{step}\\
\end{align*}
This decomposition by concat will happen until the tail of the list is nil.
Since lists are assumed to be finite, the decomposition will inevitably terminate.
When the tail becomes nil, Case x=y::[] holds.

I can't figure out the rest of this proof.
However, I tried starting at the bottom of the proof and working up. Here is 
what I came up with...
\begin{align*}
\cong & \texttt{ y::(z::(ap (zs,concat xs)))} & \text{cont. from above.}\\
\cong & \texttt{ y::(z::(ap (zs,concatap xs)))} & \text{cont. from above.}\\
\cong & \texttt{ y::(z::(concatap zs::xs)))} & \text{cont. from above.}\\
\cong & \texttt{ y::(concatap ys::xs)} & \text{rev. step}\\
\cong & \texttt{ y::(ap(ys,concat xs))} & \text{rev. step}\\
\cong & \texttt{ y::(ap(ys,concatap xs))} & \text{IH}\\
\cong & \texttt{ case y::ys of [] => l2 | x::xs => x::(ap(xs,l2))} & \text{rev. step, from ap}\\
\cong & \texttt{ ap(y::ys,concatap(xs))} & \text{rev. step}\\
\cong & \texttt{ ap(x,concatap(xs)) } & \text{x=y::[]}\\
\cong & \texttt{ case x::xs of [] => [] | x::xs => ap(x,concatap(xs)) } & \text{rev. step (concatap)}\\
\cong & \texttt{ concatap (x::xs) } & \text{rev. step}\\
\end{align*}
I was almost there :(. Partial credit please?


\question{3}{Task 2.4}

\begin{enumerate}
\item lemma 3
\item lemma 3
\item since g and f are valuable, \texttt{gof} is valuable
\item since g and f are total, \texttt{gof} are total
\item lemma 3
\item step
\item step
\item \texttt{map g(map f[]) $\cong$ map(gof)[]}
\item step
\item step
\item step, def of map
\item \texttt{map g(map f xs) $\cong$ map gof xs}
\item \texttt{map g(map f xLLxs) $\cong$ map (gof) (x::xs)}
\item step
\item def of function composition
\item def of function composition
\item ***, transitivity, and totality of map
\end{enumerate}
\end{document}

