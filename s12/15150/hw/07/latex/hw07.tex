\documentclass[11pt,letterpaper]{article}

\usepackage{amsmath}
\usepackage{amssymb}
\usepackage{fancyhdr}

\oddsidemargin0cm
\topmargin-2cm
\textwidth16.5cm
\textheight23.5cm

\newcommand{\question}[2] {\vspace{.25in} \hrule\vspace{0.5em}
\noindent{\bf #1: #2} \vspace{0.5em}
\hrule \vspace{.10in}}
\renewcommand{\part}[1] {\vspace{.10in} {\bf (#1)}}

\newcommand{\myname}{Karan Sikka}
\newcommand{\myandrew}{ksikka@andrew.cmu.edu}
\newcommand{\myhwnum}{7}

\setlength{\parindent}{0pt}
\setlength{\parskip}{5pt plus 1pt}

\pagestyle{fancyplain}
\lhead{\fancyplain{}{\textbf{HW\myhwnum}}}
\rhead{\fancyplain{}{\myname\\ \myandrew}}
\chead{\fancyplain{}{15-150}}
%% SML font
\newcommand{\sml}[1]{\texttt{#1}}

%% Evaluation and step judgements
\newcommand{\eval}[2]{{#1} \Longrightarrow {#2}}
\newcommand{\evalto}{\ensuremath{\Longrightarrow}}
\newcommand{\stepseq}[2]{{#1} \longmapsto {#2}}
\newcommand{\nstepseq}[2]{{#1} \longmapsto^* {#2}}

%% SML Syntax in math mode
\newcommand{\msml}[1]{\mathtt{#1}}

\newcommand{\mval}{\msml{v}}
\newcommand{\mvalp}{\msml{v'}}
\newcommand{\mvalpp}{\msml{v''}}
\newcommand{\mconst}{\msml{c}}
\newcommand{\mconsts}[1]{\msml{c_{#1}}}
\newcommand{\mvals}[1]{\msml{v_{#1}}}
\newcommand{\mcase}[5]{\msml{case}~{#1}~\msml{of}~{#2} => {#3}~|~{#4} => {#5}}
\newcommand{\mapp}[2]{{#1}~{#2}}
\newcommand{\mpair}[2]{({#1}, {#2})}
\newcommand{\mtriple}[3]{({#1}, {#2}, {#3})}
\newcommand{\mnil}{\msml{nil}}
\newcommand{\mcons}[2]{{#1}::{#2}}
\newcommand{\mnat}[1]{\msml{\overline{#1}}}
\newcommand{\mrtimes}[2]{{#1} ** {#2}}
\newcommand{\mrfrac}[2]{{#1} // {#2}}
\newcommand{\asc}[2]{{#1}\ :\ {#2}}
\newcommand{\smlarr}[2]{{#1}\ ->\ {#2}}
\newcommand{\econs}[2]{{#1}\ ::\ {#2}}
\newcommand{\tint}{\msml{int}}
\newcommand{\tstr}{\msml{string}}
\newcommand{\tchr}{\msml{char}}
\newcommand{\tbool}{\msml{bool}}
\newcommand{\tfso}{\msml{fsobject}}
\newcommand{\tlst}[1]{{#1}~\msml{list}}
\newcommand{\ttree}[1]{{#1}~\msml{tree}}
\newcommand{\tarr}[2]{{#1} \to {#2}}
\newcommand{\tprod}[2]{{#1} * {#2}}
\newcommand{\tnf}{\msml{nf}}
\newcommand{\oftp}[2]{{#1}\ :\ {#2}}

\newcommand{\oftpm}[2]{\msml{{#1}}\ :\ {#2}}
\newcommand{\ce}[2]{{#1} \cong {#2}}
\newcommand{\mappm}[2]{{\msml{#1}}~{\msml{#2}}}
\newcommand{\vnil}{\texttt{[]}}

%% Propositional calculus
\newcommand{\pand}{\wedge}
\newcommand{\por}{\vee}
\newcommand{\pimp}{\supset}



\begin{document}

\medskip

\thispagestyle{plain}
\begin{center}                  % Center the following lines
{\Large 15-150 Assignment \myhwnum} \\
\myname \\
\myandrew \\
G \\
\today \\
\end{center}


\question{1}{Task 4.1}
Tabulate performs an operation $n$ times, where $n$ is the sum of 
the lengths of \sml{s1:'a Seq.seq} and \sml{s2:'a Seq.seq}. Each operation
is constant time, since in the \sml{true} case, \sml{nth} is const time and
in the \sml{false} case, \sml{nth} and \sml{length} are const time. Since
a constant time operation is performed $n$ times, myAppend has 
work of $\mathbf{O(n)}$ where $n$ is the sum of the lengths of the sequences.

\question{2}{Task 4.2}
Tabulate performs an operation $n$ times, where $n$ is the sum of 
the lengths of \sml{s1:'a Seq.seq} and \sml{s2:'a Seq.seq}. Note that these 
operations may be done in parallel. Each operation is constant time, since 
in the \sml{true} case, \sml{nth} is const time and in the \sml{false} case, 
\sml{nth} and \sml{length} are const time. Since tabulate has span of $\mathbf{O(1)}$ 
and each operation is const time, myAppend has span of $O(1)$.

\question{3}{Task 4.3}
If we draw the mapreduce tree, we can see that for each of the $log(n)$ levels,
myAppend does $n$ total work. Singleton is const time, and happens $n$ times.
Seq.empty is also const time, and happens fewer than $n$ times. Therefore, the
work is tightly upper bounded by $\mathbf{O(nlog(n))}$. This differs from the other 
reverse function because in that function, a constant time operation occurred
$n$ times. Here a linear operation occurs $log(n)$ times.

\question{4}{Task 4.4}
The span is the span of the longest branch of the mapreduce tree. 
The height of the mapreduce tree is proportional to $log(n)$, and
the span of myAppend is constant, so the span of of \sml{reverse'}
is $O(log(n))$. Since \sml{reverse'} uses mapreduce, there is a 
dependency on previous operations to finish whereas in \sml{reduce},
each tabulate operation can be computed independently

\question{5}{Task 4.5}
Suffixes has a tight upper bound of $\mathbf{O(n^2)}$ because it evaluates Seq.length
once (const time) and Seq.drop, which has linear work, $n$ times.

\question{6}{Task 4.6}
\sml{Seq.drop} has a span of $O(1)$, and tabulate has a span of $O(1)$ for a constant time operation.
Therefore \sml{suffixes} has a span of $\mathbf{O(1)}$.

\question{7}{Task 4.7}
Seq.zip does linear work, and suffixes does linear work.
Both functions are called once. Therefore the overall work
of withSuffixes is $\mathbf{O(n)}$.

\question{8}{Task 4.8}
The span of suffixes is $O(1)$. After this executes, Seq.zip is called,
which has a span of $O(1)$. All in all, withSuffixes has a span of $\mathbf{O(1)}$.

\question{9}{Task 4.9}
First, \sml{maxS} is applied to each sequence in the outer sequence. Since mapS
makes a call to reduce using the comparator Int.max, the total work is $O(nk_i)$
where $k_i$ is $max(k_1 ... k_n)$. Then maxS is called on the resulting sequence
of the previously operation, which has length $n$. Since $O(n)$ is a lower bound
than the previous bound, the overall bound for the work is $\mathbf{O(nk_i)}$.

\question{10}{Task 4.10}
First, \sml{maxS} is applied to each sequence in the outer sequence. Since mapS
makes a call to reduce using the comparator Int.max, the total span is $O(log(k_i))$
where $k_i$ is $max(k_1 ... k_n)$. Then maxS is called on the resulting sequence
of the previously operation, which has length $n$ and span $O(log(n))$.
Since this step depends on the first, the span is $O(log(n) + log(k_i))$
which can be rewritten as $\mathbf{O(log(nk_i))}$.

\question{11}{Task 4.11}
First, \sml{withSuffixes} is called, and as previously determined, this has 
work of $O(n)$, where $n$ is the length of $s$. Then, the map within a map
happens. $n$ times, an $n$-times map occurs. Therefore the work of this is
$O(n^2)$. The result is an $n$-length sequence containing sequences of length
at most $n$. As previously found, the work of \sml{maxAll} is $O(nk_i)$, which 
is $O(n^2)$ in this case. Since these steps happen sequentially, the work is 
bounded by the largest time bound, which is $\mathbf{O(n^2)}$.

\question{12}{Task 4.12}
First, \sml{withSuffixes} is called, and as previously determined, this has 
span of $O(1)$. Then, the map within a map happens. The inner map has span of O(1)
since the operation is constant time, and therefore the outer map is also O(1). 
The overall span for this step is O(1). Then, for \sml{maxAll}, we know the span 
of this is $O(log(nk_i))$, or in this case, $O(log(n^2))=O(2log(n))=\mathbf{O(log(n))}$. 
Since these steps happen sequentially, the span is bounded by the sum of these spans,
which is $\mathbf{O(log(n))}$.
\end{document}

