\documentclass[11pt,letterpaper]{article}

\usepackage{amsmath}
\usepackage{amssymb}
\usepackage{fancyhdr}
\usepackage{longtable}

\oddsidemargin0cm
\topmargin-2cm
\textwidth16.5cm
\textheight23.5cm

\newcommand{\question}[2] {\vspace{.25in} \hrule\vspace{0.5em}
\noindent{\bf #1: #2} \vspace{0.5em}
\hrule \vspace{.10in}}
\renewcommand{\part}[1] {\vspace{.10in} {\bf (#1)}}

\newcommand{\myname}{Karan Sikka}
\newcommand{\myandrew}{ksikka@andrew.cmu.edu}
\newcommand{\myhwnum}{3}

\setlength{\parindent}{0pt}
\setlength{\parskip}{5pt plus 1pt}

\pagestyle{fancyplain}
\lhead{\fancyplain{}{\textbf{HW\myhwnum}}}
\rhead{\fancyplain{}{\myname\\ \myandrew}}
\chead{\fancyplain{}{15-150}}
%% SML font
\newcommand{\sml}[1]{\texttt{#1}}

%% Evaluation and step judgements
\newcommand{\eval}[2]{{#1} \Longrightarrow {#2}}
\newcommand{\evalto}{\ensuremath{\Longrightarrow}}
\newcommand{\stepseq}[2]{{#1} \longmapsto {#2}}
\newcommand{\nstepseq}[2]{{#1} \longmapsto^* {#2}}

%% SML Syntax in math mode
\newcommand{\msml}[1]{\mathtt{#1}}

\newcommand{\mval}{\msml{v}}
\newcommand{\mvalp}{\msml{v'}}
\newcommand{\mvalpp}{\msml{v''}}
\newcommand{\mconst}{\msml{c}}
\newcommand{\mconsts}[1]{\msml{c_{#1}}}
\newcommand{\mvals}[1]{\msml{v_{#1}}}
\newcommand{\mcase}[5]{\msml{case}~{#1}~\msml{of}~{#2} => {#3}~|~{#4} => {#5}}
\newcommand{\mapp}[2]{{#1}~{#2}}
\newcommand{\mpair}[2]{({#1}, {#2})}
\newcommand{\mtriple}[3]{({#1}, {#2}, {#3})}
\newcommand{\mnil}{\msml{nil}}
\newcommand{\mcons}[2]{{#1}::{#2}}
\newcommand{\mnat}[1]{\msml{\overline{#1}}}
\newcommand{\mrtimes}[2]{{#1} ** {#2}}
\newcommand{\mrfrac}[2]{{#1} // {#2}}
\newcommand{\asc}[2]{{#1}\ :\ {#2}}
\newcommand{\smlarr}[2]{{#1}\ ->\ {#2}}
\newcommand{\econs}[2]{{#1}\ ::\ {#2}}
\newcommand{\tint}{\msml{int}}
\newcommand{\tstr}{\msml{string}}
\newcommand{\tchr}{\msml{char}}
\newcommand{\tbool}{\msml{bool}}
\newcommand{\tfso}{\msml{fsobject}}
\newcommand{\tlst}[1]{{#1}~\msml{list}}
\newcommand{\ttree}[1]{{#1}~\msml{tree}}
\newcommand{\tarr}[2]{{#1} \to {#2}}
\newcommand{\tprod}[2]{{#1} * {#2}}
\newcommand{\tnf}{\msml{nf}}
\newcommand{\oftp}[2]{{#1}\ :\ {#2}}

\newcommand{\oftpm}[2]{\msml{{#1}}\ :\ {#2}}
\newcommand{\ce}[2]{{#1} \cong {#2}}
\newcommand{\mappm}[2]{{\msml{#1}}~{\msml{#2}}}
\newcommand{\vnil}{\texttt{[]}}

%% Propositional calculus
\newcommand{\pand}{\wedge}
\newcommand{\por}{\vee}
\newcommand{\pimp}{\supset}



\begin{document}

\medskip

\thispagestyle{plain}
\begin{center}                  % Center the following lines
{\Large 15-150 Assignment \myhwnum} \\
\myname \\
\myandrew \\
G \\
2/7/12 \\
\end{center}


\question{1}{Task 2.3}

Claim: For all \sml{l : (int * string) list,\hspace{1cm} zip(unzip l) $\cong$ l}\\

Proof: The proof is by structural induction on \sml{l}.\\
\begin{tabular}{lp{12cm}l}
&\textbf{Case for []}&\\
&To show \sml{zip(unzip([])) $\cong$ l}:&\\\\
&\sml{zip(unzip([]))}&given\\
$\cong$&\sml{zip(case [] of [] => ([],[]) | ...)}&step\\
$\cong$&\sml{zip([],[])}&step\\
$\cong$&\sml{case [] of [] => [] |...}&step\\
$\cong$&\sml{[]}&step\\
\\
&\textbf{Case for x::xs.} Inductive hypothesis: \sml{zip(unzip(xs)) $\cong$ xs}&\\
$\cong$&\verb!zip(unzip(x::xs))!&\\
$\cong$&\begin{verbatim}
zip(case x::xs of
        [] => ([],[])
      | x::xs => let val (pint, pstr) = x
                     val (ints, strs) = unzip(xs)
                 in (pint::ints,pstr::strs)
                 end)\end{verbatim}&step\\
$\cong$&\begin{verbatim}
zip(let val (pint, pstr) = x
        val (ints, strs) = unzip(xs)
    in (pint::ints,pstr::strs)
    end) \end{verbatim}&step, evaluate case\\
$\cong$&\verb!zip(pint::ints,pstr::strs)!&step, evaluate let\\
$\cong$&\begin{verbatim}
case pint::ints of [] => []
                 | pint::ints => let val y::ys = pstr::strs
                                 in (pint,y)::zip(ints,ys)
                                 end \end{verbatim}&step\\
\end{tabular}
\begin{tabular}{lp{10cm}l}
$\cong$&\begin{verbatim}
let val y::ys = pstr::strs
in (pint,y)::zip(ints,ys)
end \end{verbatim}&step, evaluate case\\
$\cong$&\verb!(pint,pstr)::zip(ints,strs)!&step, evaluate let\\
$\cong$&\verb!x::zip(unzip(xs))!&by definition in let\\
$\cong$&\verb!x::xs!&IH\\
\end{tabular}



\question{2}{Task 2.4}
\textbf{Claim:}\\
For all \sml{l1 : int list} and \sml{l2 : string list},\\
\sml{unzip(zip (l1,l2)) $\cong$ (l1,l2)}
Counterexample:\\
\\
Let l1 be the value \sml{[]:int list}\\
Let l2 be the value \sml{["7"]:string list}\\
\\
$\cong$ \sml{unzip(zip (l1,l2))}\\
$\cong$ \sml{unzip(zip ([],["7"]))}\\
$\cong$ \sml{unzip(case [] of
       [] => []
     | x::xs => ...)}
$\cong$ \sml{unzip([])}\\
$\cong$ \sml{case [] of [] => ([],[]) | x::xs...}\\
$\cong$ \sml{([],[])}\\
This is not equivalent to \sml{([],["7"])}\\
\\
Therefore, the claim is not true.


\question{3}{Task 4.2}
Let $W_n$ be the work of \sml{prefixSum}
$$W_n = k_0 + W_{n-1} + k(n-1)$$
$$W_n = k_0 + k_0 + W_{n-2} + k(n-2) + k(n-1)$$
$$W_n = k_0 + k_0 + k_0 + W_{n-3} + k(n-3) + k(n-2) + k(n-1)$$
$$.$$
$$.$$
$$.$$
$$W_n = n(k_0) + \sum_{i=1}^{n}k(n-i)$$
$$W_n \leq n(k_0) + \sum_{i=1}^{n}k(n)$$
$$W_n \leq n(k_0) + n^2 k$$
Therefore, $W_n$ has a $O(n^2)$ time complexity.


\question{4}{Task 4.4}
Let $W_n$ be the work of prefixSumFast for a list of length n, $n > 0$.
$$W_n= k_0 + W_{prefixSumHelp(n)}$$

Therefore, the work of prefixSumFast is bounded by the work of prefixSumHelp.\\
Let $W_n$ be the work of prefixSumHelp for a list of length n, $n > 0$.

$$W_n= k_0 + W_{n-1}$$

This is because the function calls itself exactly once on a sublist of length $n-1$. The other operations which occur on the function are constant time and can be considered as $k_0$. The closed form for this recurrance is obvious, and has been shown many times before in class: 
$$W_n= n k_0$$
Therefore the function is bounded by $O(n)$.


\question{5}{Task 5.1}
$i \geq 0, \qquad k\geq 0, \qquad k\geq \msml{length(l)} -i $

\question{6}{Task 6.4}
If \sml{subset\_sum\_dc (s,t)} $\cong$ true, then there exists a subset of s which sums to t.\\
Assume the hypothesis. The only way for \sml{subset\_sum\_dc (s,t)} to be equivalent to true, is if it returns true. Let's see when the function returns true:
\begin{verbatim}
fun subset_sum_dc (l : int list, s : int) : bool =
    let val (sumP, yoU) = subset_sum_cert(l,s)
    in
      case sumP of
           false => false
         | true => (case ((sum_list(yoU) = s) andalso (contained(yoU,l))) of
                        false => raise Fail "invalid certificate"
                      | true => true)
    end
\end{verbatim}

We can see that the function if the function returns true if and only if the following evaluates to true:
\begin{verbatim}
(case ((sum_list(yoU) = s) andalso (contained(yoU,l))) of
     false => raise Fail "invalid certificate"
   | true => true)
\end{verbatim}
This case statement will only return true if the logical condition:\\
 \sml{(sum\_list(yoU) = t) andalso (contained(yoU,s))} is equivalent to true, where yoU is of type \sml{int list} .\\
This occurs iff \sml{sum\_list(yoU)=t} is true and \sml{contained(yoU,l)} is true.\\
Here we assume that sum\_list and contained behaved as described in the task.\\
Since (contained(yoU,s) $\cong$ true), then according to the spec, yoU is a subset of s.\\
Since (sum\_list(yoU) $\cong$ = t), then according to the spec, yoU sums to t.\\
Therefore, there exists a subset of s which sums to t. $\blacksquare$

\end{document}


