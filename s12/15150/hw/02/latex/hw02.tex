\documentclass[11pt,letterpaper]{article}

\usepackage{amsmath}
\usepackage{amssymb}
\usepackage{fancyhdr}

%% cost graph constants
\newcommand{\costzero}{\mathbf{0}}
\newcommand{\costunit}{\mathbf{1}}

\newcommand{\maximum}[2]{\mathrm{max}(#1, #2)}

%% Evaluation and step judgements
\newcommand{\eval}[2]{{#1} \Longrightarrow {#2}}
\newcommand{\stepseq}[2]{{#1} \longmapsto {#2}}
\newcommand{\nstepseq}[2]{{#1} \longmapsto^* {#2}}

%% SML Syntax
\newcommand{\sml}[1]{\texttt{#1}}
\newcommand{\msml}[1]{\mathtt{#1}}

\newcommand{\mexp}{\msml{e}}
\newcommand{\mexps}[1]{\msml{e_{#1}}}
\newcommand{\mval}{\msml{v}}
\newcommand{\mvals}[1]{\msml{v_{#1}}}
\newcommand{\mcase}[5]{\msml{case}~{#1}~\msml{of}~{#2} => {#3}~|~{#4} => {#5}}
\newcommand{\mapp}[2]{{#1}~{#2}}
\newcommand{\mtrue}{\msml{true}}
\newcommand{\mfalse}{\msml{false}}
\newcommand{\mlist}{\msml{l}}
\newcommand{\mlists}[1]{\msml{l_{#1}}}
\newcommand{\mappend}[2]{{#1} @ {#2}}
\newcommand{\mnat}{\msml{n}}
\newcommand{\mnats}[1]{\msml{n_{#1}}}
\newcommand{\mvar}{\msml{x}}
\newcommand{\mvars}[1]{\msml{x_{#1}}}
\newcommand{\mnil}{\msml{nil}}
\newcommand{\mcons}[2]{{#1}::{#2}}
\newcommand{\mzero}{\msml{0}}
\newcommand{\mpair}[2]{({#1}, {#2})}
\newcommand{\mraise}[1]{\msml{raise}~{#1}}
\newcommand{\mraiseFail}[1]{\mraise{(\msml{Fail}~{#1})}}
\newcommand{\tint}{\msml{int}}
\newcommand{\tlist}[1]{{#1}~\msml{list}}
\newcommand{\talpha}{\msml{'a}}
\newcommand{\tarr}[2]{{#1} \to {#2}}
\newcommand{\tprod}[2]{{#1} * {#2}}
\newcommand{\oftp}[2]{{#1}\ :\ {#2}}

%% Fitted parentheses
\newcommand{\fp}[1]{\left ( {#1} \right )}
\newcommand{\fcp}[1]{\left \{ {#1} \right \}}
\newcommand{\fpfrac}[2]{\fp{\frac{#1}{#2}}}


%% Cost
\newcommand{\work}[2]{W_{\text{#1}}\fp {#2}}
\newcommand{\depth}[2]{D_{\text{#1}}\fp {#2}}
\newcommand{\spce}[2]{S_{\text{#1}}\fp {#2}}

\renewcommand{\O}[1]{O \fp{#1}}

\newcommand{\seq}[1]{\left \langle {#1} \right \rangle}
\renewcommand{\mlist}[1]{\left [ {#1} \right ]}

%% Formatting type ascription properly in text mode
\newcommand{\asc}[2]{{#1}\ :\ {#2}}


\oddsidemargin0cm
\topmargin-2cm
\textwidth16.5cm
\textheight23.5cm

\newcommand{\question}[2] {\vspace{.25in} \hrule\vspace{0.5em}
\noindent{\bf #1: #2} \vspace{0.5em}
\hrule \vspace{.10in}}
\renewcommand{\part}[1] {\vspace{.10in} {\bf (#1)}}

\newcommand{\myname}{Karan Sikka}
\newcommand{\myandrew}{ksikka@andrew.cmu.edu}
\newcommand{\myhwnum}{2}

\setlength{\parindent}{0pt}
\setlength{\parskip}{5pt plus 1pt}

\pagestyle{fancyplain}
\lhead{\fancyplain{}{\textbf{HW\myhwnum}}}
\rhead{\fancyplain{}{\myname\\ \myandrew}}
\chead{\fancyplain{}{15-150}}

\begin{document}

\medskip

\thispagestyle{plain}
\begin{center}                  % Center the following lines
{\Large 15-150 Assignment \myhwnum} \\
\myname \\
\myandrew \\
Section G but im not sure\\
January 25, 2012 \\
\end{center}


\question{1}{Task 2.1}
No, this does not type check. When smlnj tries to evaluate the expression 
\verb|square 7.0|, it fails to do so, because 7.0 is not an int but 
\verb|square| is a function of type \verb|int -> int|.

\question{2}{Task 2.2}
a. The value 3 : int is put in the place of the x in line 5. This is because of
 the val declaration in line 4 takes precedence over the x defined in the 
arguments of the function, due to the scope properties of the let clause.

b. The value of the expression \verb|5.2 * (real x)| is substituted for p, where
 x is 3, as said in part a. This evaluates to \verb|5.2 * (real 3)|, then 
 \verb|5.2 * 3.0|, then \verb|15.6|. So ultimately the value which replaces p is 
15.6. This is because p is within the \verb|in| block so it takes the value 
prescribed to it in the \verb|let| block, specifically line 5. Also, SML is 
pass-by-value, so the value is passed to the \verb|in| block, not the 
expression.

c. The value of the first argument passed to the fucntion replaces x, because 
the declaration of the function names the first argument x. The x in line 4 or 
7 do not substitute for the x in line 11, because those are within the scope 
of the inner let statement, and they disappear after the end on line 9.

d. assemble (x, 2.0) evaluates to 58.

\question{3}{Task 2.3}\begin{verbatim}
let val x : real = real (square 6) 
in 3 + (trunc x) 
end
\end{verbatim}
$\mapsto$
\begin{verbatim}let val x : real = real (6 * 6)
in 3 + (trunc x)
end
\end{verbatim}
$\mapsto$ \begin{verbatim}
let val x : real = real (36)
in 3 + (trunc x)
end
\end{verbatim}
$\mapsto$ \begin{verbatim}
let val x : real = 36.0
in 3 + (trunc x)
end
\end{verbatim}
$\mapsto$ \verb|3 + (trunc 36.0)| \\\\
$\mapsto$ \verb|3 + 36| \\\\
$\mapsto$ \verb|39| \\\\


\question{4}{Task 2.4}
They are equivalent. The fact function steps to a case statement, which 
eventually steps to \sml{$\sim$1*fact($\sim$2)}. This repeats for quite a long time, and 
assuming integers do not overflow, and that our heap is infinitely large, this function will infinite loop.

In the case of f 10, this function steps to f 10. It steps to the same 
thing consistently, using no additional memory. This will also infinite loop.

Therefore, the expressions are equivalent.

\question{5}{Task 2.5}
Consider the expression \verb|fact 3|.\\

\sml{fact(3)}\\
$\mapsto$
\sml{case 3 of 0 => 1 | \_ => 3 * fact(3-1)} \\
$\mapsto$
\sml{3 * fact(3-1)} \\
$\mapsto$
\sml{3 * fact(2)} \\
$\mapsto$
\sml{3 * case 2 of 0 => 1 | \_ 2 * fact(2-1)}\\
$\mapsto$
\sml{3 * 2 * fact(2-1)}\\
$\mapsto$
\sml{3 * 2 * fact(1)}\\
$\mapsto$
\sml{6 * fact(1)}\\
$\mapsto$
\sml{6 * case 1 of 0 => 1 | \_ 1 * fact(1-1)}\\
$\mapsto$
\sml{6 * 1 * fact(1-1)}\\
$\mapsto$
\sml{6 * 1 * fact(0)}\\
$\mapsto$
\sml{6 * fact(0)}\\
$\mapsto$
\sml{6 * case 0 of 0 => 1 | \_ 0 * fact(0-1)}\\
$\mapsto$
\sml{6 * 1}\\
$\mapsto$
\sml{6}\\
Thus, fact(3) evaluates to 6.

Now consider:\\
\begin{tabular}{lll}
       &\sml{fact(fact(3))}& \\
$\cong$&\sml{fact(6)}&equivalence proven above\\
$\cong$&\sml{case 6 of 0 => 1 | \_ 6 * fact(6-1)}&step\\
$\cong$&\sml{6 * fact(6-1)}&step\\
$\cong$&\sml{fact(3) * fact(fact(3) - 1)}&step, equiv. is symmetric, fact(3)=6 as shown above.
\end{tabular}
\\
Therefore:\\
\sml{fact(fact(3))} $\cong$ \sml{fact(3) * fact(fact(3) - 1)}

\question{6}{Task 3.2}
$H_0$ is $0$. $H_1$ is $1$. $H_2$ is $\frac{3}{2}$. $H_3$ is $\frac{11}{6}$. $H_4$ is $\frac{25}{12}$.

\question{7}{Task 3.4}
$I_0$ is $0$. $I_1$ is $1$. $I_2$ is $\frac{1}{2}$. $I_3$ is $\frac{5}{6}$. $I_4$ is $\frac{7}{12}$.

\question{8}{Task 4.1}
Claim: For all natural numbers n, \sml{2(n+1)} $\cong$ \sml{double(n+1)} . \\
We proceed by induction on $n$.
Base Case:\\
\begin{tabular}{lll}
       &\sml{double n} where n = 0                 & \\
$\cong$&\sml{double 0}                             & substituting\\
$\cong$&\sml{case 0 of 0 => 0 | \_ 2 + double(n-1)}&step\\
$\cong$&\sml{0}                                    &step\\
$\cong$&\sml{2*0}                                  &math, also step is symmetric\\
$\cong$&\sml{2n where n = 0}                       &step\\

\end{tabular}
\\
Inductive Hypothesis:
Assume \sml{double k} $\cong 2*k$ for some $k \in \mathbb{N}$
\\
Inductive Step:
Consider:\\
\begin{tabular}{lll}
&\sml{2(k+1)}&\\
$\cong$&\sml{2k + 2}&math, distributive property\\
$\cong$&\sml{2 + double(k)}&IH\\
$\cong$&\sml{2 + double(k + 1 - 1)}&math, step, equiv. is sym.\\
$\cong$&\sml{case 0 of 0 => 0 | 2 + double(k + 1 - 1)}&step, equiv. is symmetric\\
$\cong$&\sml{double(k+1)}&step, equiv. is symmetric
\end{tabular}
\\Therefore \sml{2(k+1)} $\cong$ \sml{double(k+1)}. By induction, the claim holds.
\\

\question{9}{Task 4.2}
Claim: For all natural numbers $n$, \sml{summ n} $\cong$ \sml{(n*(n+1)) div 2} . \\
We proceed by induction on n.

Base Case:\\
\begin{tabular}{lll}
&\sml{summ n} where n = 0&\\
$\cong$&\sml{summ 0}&substitution\\
$\cong$&\sml{case 0 of 0 => 0 | \_ 0 + (summ(0-1))}&step\\
$\cong$&\sml{0}&step\\
$\cong$&\sml{0*1}&math\\
$\cong$&\sml{0*(0+1)}&math\\
$\cong$&\sml{0*(0+1) div 2}&math\\
$\cong$&\sml{n*(n+1) div 2 where n = 0}&substitution\\
\end{tabular}
\\
Inductive Hypothesis:\\
Assume \sml{summ n }$\cong$ \sml{n*(n+1) div 2} for some $n \in \mathbb{N}$.\\

Inductive Step:\\
First, we make the assumption that n+1 evaluates to a value that we will call n+1.
Consider:\\
\begin{tabular}{lll}

$\cong$&\sml{summ(n+1)}&\\
$\cong$&\sml{case n+1 of 0 => 0 | \_ n+1 + summ((n+1)-1)}&step\\
$\cong$&\sml{n+1 + summ((n+1)-1)}&step\\
$\cong$&\sml{n+1 + summ(n)}&math, addition is associative\\
$\cong$&\sml{n+1 + n*(n+1) div 2}&IH\\
$\cong$&\sml{((n+1)(n+2)) div 2}&algebra shown below\\

\end{tabular}
\\
\\
The algebra:
$$\frac{n^2+n}{2} + \frac{2(n+1)}{2}$$
Distribute the right fraction:
$$\frac{n^2+n}{2} + \frac{2n+2}{2}$$
Add the fractions:
$$\frac{n^2+3n + 2}{2}$$
Factor:
$$\frac{(n+1)(n+2)}{2}$$

By induction, the claim holds.
\end{document}

