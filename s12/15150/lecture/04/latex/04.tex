\documentclass[11pt]{article}
\usepackage{fullpage}
\usepackage{times}
\usepackage{amsmath,amsthm}
\usepackage{url}
\usepackage{framed}
\usepackage{verbatim}
\usepackage[bookmarks=true,colorlinks=true,unicode=true,breaklinks]{hyperref}
\newcommand{\exercise}[0]{\noindent \textbf{Exercise.} \smallskip }

\newtheorem{theorem}{Theorem}
\newtheorem{lemma}{Lemma}

\newcommand{\ttt}[1]{\texttt{#1}}
\newcommand{\evalsto}[0]{\ensuremath{\Longrightarrow}}

\title{15-150 Lecture 4: \\ Lists}
\author{Lecture by Ian Voysey}
\date{26 January 2012}

%% cost graph constants
\newcommand{\costzero}{\mathbf{0}}
\newcommand{\costunit}{\mathbf{1}}

\newcommand{\maximum}[2]{\mathrm{max}(#1, #2)}

%% Evaluation and step judgements
\newcommand{\eval}[2]{{#1} \Longrightarrow {#2}}
\newcommand{\stepseq}[2]{{#1} \longmapsto {#2}}
\newcommand{\nstepseq}[2]{{#1} \longmapsto^* {#2}}

%% SML Syntax
\newcommand{\sml}[1]{\texttt{#1}}
\newcommand{\msml}[1]{\mathtt{#1}}

\newcommand{\mexp}{\msml{e}}
\newcommand{\mexps}[1]{\msml{e_{#1}}}
\newcommand{\mval}{\msml{v}}
\newcommand{\mvals}[1]{\msml{v_{#1}}}
\newcommand{\mcase}[5]{\msml{case}~{#1}~\msml{of}~{#2} => {#3}~|~{#4} => {#5}}
\newcommand{\mapp}[2]{{#1}~{#2}}
\newcommand{\mtrue}{\msml{true}}
\newcommand{\mfalse}{\msml{false}}
\newcommand{\mlist}{\msml{l}}
\newcommand{\mlists}[1]{\msml{l_{#1}}}
\newcommand{\mappend}[2]{{#1} @ {#2}}
\newcommand{\mnat}{\msml{n}}
\newcommand{\mnats}[1]{\msml{n_{#1}}}
\newcommand{\mvar}{\msml{x}}
\newcommand{\mvars}[1]{\msml{x_{#1}}}
\newcommand{\mnil}{\msml{nil}}
\newcommand{\mcons}[2]{{#1}::{#2}}
\newcommand{\mzero}{\msml{0}}
\newcommand{\mpair}[2]{({#1}, {#2})}
\newcommand{\mraise}[1]{\msml{raise}~{#1}}
\newcommand{\mraiseFail}[1]{\mraise{(\msml{Fail}~{#1})}}
\newcommand{\tint}{\msml{int}}
\newcommand{\tlist}[1]{{#1}~\msml{list}}
\newcommand{\talpha}{\msml{'a}}
\newcommand{\tarr}[2]{{#1} \to {#2}}
\newcommand{\tprod}[2]{{#1} * {#2}}
\newcommand{\oftp}[2]{{#1}\ :\ {#2}}

%% Fitted parentheses
\newcommand{\fp}[1]{\left ( {#1} \right )}
\newcommand{\fcp}[1]{\left \{ {#1} \right \}}
\newcommand{\fpfrac}[2]{\fp{\frac{#1}{#2}}}


%% Cost
\newcommand{\work}[2]{W_{\text{#1}}\fp {#2}}
\newcommand{\depth}[2]{D_{\text{#1}}\fp {#2}}
\newcommand{\spce}[2]{S_{\text{#1}}\fp {#2}}

\renewcommand{\O}[1]{O \fp{#1}}

\newcommand{\seq}[1]{\left \langle {#1} \right \rangle}
\renewcommand{\mlist}[1]{\left [ {#1} \right ]}

%% Formatting type ascription properly in text mode
\newcommand{\asc}[2]{{#1}\ :\ {#2}}


\begin{document}
\maketitle

\noindent
\tableofcontents \newpage

\section{What Are Lists?}
\begin{framed}
\begin{quote}
A list of integers (a value of type \verb|int list|) is either
\begin{itemize}
\item[] \verb|[]|, or
\item[] \verb|x :: xs| where \verb|x : int| and \verb|xs : int list|.
\end{itemize}
And that's it!
\end{quote}
\end{framed}

\verb|[]| is pronounced ``nil'' or ``the empty list''; \verb|::| is
pronounced ``cons''. Therefore, the values of type \verb|int list| are
lists like these:
\begin{verbatim}
  1 :: (2 :: (3 :: (4 :: [])))
\end{verbatim}
This can also be written without the parens as
\begin{verbatim}
  1 :: 2 :: 3 :: 4 :: []
\end{verbatim}
because \verb|::| is right-associative. For a particular list with a fixed
number of elements, you can also write the elements inside of square-brackets
separated by commas, as in
\begin{verbatim}
  [1,2,3,4]
\end{verbatim}
This is just a convenient notation from SML; it's short hand for the above
form.
\begin{framed}
The operation on lists is case analysis (and recursion):
\begin{verbatim}
case l of
    [] => <branch1>
  | x :: xs => <branch2, with (x : int) and (xs : int list) in scope>
\end{verbatim}
\end{framed}

giving a branch for \verb|[]| and a branch for \verb|::|. In the body of
the \verb|::| branch, the variable \verb|x| stands for the first element of
the list, and the variable \verb|xs| stands for the rest of the list.

Note that \verb|::| is being used both to create lists, in value
declarations, and take lists apart, in the part of the \verb|::| branch to
the left of \sml{=>}.

\section{Structurally Recursive Functions on Lists}

\subsection{length}
Let's write a function compute the length of a list, which is to say the
number of elements in the list, or the number of times that \verb|::|
appears in the structure of the list.
\verbatiminput{length.sml}

The length of the empty list is \verb|0|: it has no elements, and we know
that because \verb|::| does not appear in \verb|[]|. If we assume that the
recursive call is correct inductively, the length of \verb|x :: xs| is one
more than the length of the tail \verb|xs|: we saw an instance of
\verb|::|, so we add $1$ and recur.

Here's an example of \sml{length} running on a small list:

\begin{verbatim}
length (1 :: (2 :: []))
|-> case (1 :: (2 :: [])) of [] => 0 | x :: xs => 1 + length xs
|-> 1 + length (2 :: [])
|-> 1 + (case 2 :: [] of [] => 0 | x :: xs => 1 + length xs)
|-> 1 + (1 + (length []))
|-> 1 + (1 + (case [] of [] => 0 | x :: xs => 1 + length xs))
|-> 1 + (1 + (0))
|-> 1 + 1
|-> 2
\end{verbatim}

\sml{length} transforms a list into an integer by walking down the list,
replacing each element with \sml{1}, and each \texttt{::} with \sml{+}, and
then evaluating up all of the additions once the recursion finishes. So
above, we start with the int list expression
\begin{verbatim}
  1 :: 2 :: []
\end{verbatim}
and we get the int expression
\begin{verbatim}
  1 +  1 +  0
\end{verbatim}

\subsection{sum}
We can write a function to compute the sum of the elements in a list very
similarly:
\verbatiminput{sum.sml}
\sml{sum} transforms a list into an integer by walking down the list,
leaving each element alone, replacing each \sml{::} with a \sml{+}, and
evaluating up all of the additions once the recursion finishes.

So, for example, in the trace
\begin{verbatim}
sum (1 :: (2 :: []))
|-> case (1 :: (2 :: [])) of [] => 0 | x :: xs => x + sum xs
|-> 1 + sum (2 :: [])
|-> 1 + (case 2 :: [] of [] => 0 | x :: xs => x + sum xs)
|-> 1 + (2 + (sum []))
|-> 1 + (2 + (case [] of [] => 0 | x :: xs => x + sum xs))
|-> 1 + (2 + (0))
|-> 1 + 2
|-> 3
\end{verbatim}
we transform the int list expression
\begin{verbatim}
  1 :: 2 :: []
\end{verbatim}
into the int expression
\begin{verbatim}
  1 +  2 +  0
\end{verbatim}

\subsection{General Form}
These two functions suggest a general template for a function on lists:
\begin{framed}
\begin{verbatim}
(* Purpose: operate on every element of a the list *)
fun f (l : int list) : A =
    case l of
        [] => < expression of type A >
      | x :: xs => < expression of type A in terms of x : int,
                                                     xs : int list,
                                             and (f xs) : A >
\end{verbatim} 
\end{framed}
In the cons case, you can use the first element of the list \verb|x|, the
rest \verb|xs|, and a recursive call to \verb|f| on the smaller list
\verb|xs|. Note that any call to \verb|f| on any expression equivalent to
\verb|l| that gets evaluated in the recursive case causes
non-termination.

\subsection{Raise Salaries}
Let's do something a bit more practical. I have here a list of the TA's
salaries and---since Dan isn't around to stop me---I'm going to give everyone
a big raise.

\verbatiminput{raise.sml}

Note that this function takes two arguments, like the \verb|add| function
you wrote in lab; see the end of the Lecture 3 notes if you need to review
pairs. Also note that this function transforms integer lists into other
integer lists, which is slightly more interesting than the examples above;
this is a case where we've picked the type \sml{A} in the general form to
be the type \sml{int list}.

\section{Structural Induction on Lists}

\newcommand{\rb}{rB}
\newcommand{\LHS}[1]{\mappm{\rb}{(\mappm{\rb}{({#1},a)},b)}}
\newcommand{\RHS}[1]{\mappm{\rb}{({#1},a+b)}}
\newcommand{\pred}[1]{\ce{\LHS{#1}}{\RHS{#1}}}
\newcommand{\rbbod}[2]{\mcase{#1}{\vnil}{\vnil}
  {\msml{x::xs}}{\msml{(x+{#2})}\msml{::}\mappm{\rb}{(xs,{#2})}}}

\subsection{Proof of Fusion Property}
Let's prove that two nested calls of \sml{raiseBy} can be fused into one
call without changing the result.  Specifically, we'll prove Theorem
\ref{thrm:fusion}. This is saying that we can optimize and traverse the
list once instead of twice, and no program will be able to tell the
difference.

\begin{theorem}[Fusion]
\label{thrm:fusion}
For all values $\oftpm{l}{\tlst{\tint}}, \oftpm{a}{\tint},
\oftpm{b}{\tint}$, $$\pred{l}$$
\end{theorem}
\begin{proof}
The proof is by structural induction on $l$. In all of the cases below, let
\sml{a} and \sml{b} be any values of type $\tint$.

\begin{description}
\item [\textbf{Case for \ttt{[]}}]
  To show: $$\pred{\vnil}$$
  Proof:
  \begin{align*}
    &\LHS{\vnil} &\\
    \cong~& \mappm{\rb}{(\rbbod{\vnil}{a},b)} &\text{step}\\
    \cong~& \mappm{\rb}{(\vnil,b)}            &\text{step}\\
    \cong~& \rbbod{\vnil}{b}                  &\text{step}\\
    \cong~& \vnil                             &\text{step}\\
    \cong~& \rbbod{\vnil}{(a+b)}              &\text{step, sym}\\
    \cong~& \RHS{\vnil}                       &\text{step, sym}
  \end{align*}
  By transitivity of $\cong$, this concludes this case.

\item [\textbf{Case for \ttt{x::xs}}]
  inductive hypothesis: $$\pred{xs}$$
  To show: $$\pred{x::xs}$$
  Proof:
  \begin{align*}
    &\LHS{x::xs} &\\
    \cong~& \mappm{\rb}{(\rbbod{x::xs}{a},b)}               &\text{step}\\
    \cong~& \mappm{\rb}{((x+a)::\mappm{\rb}{(xs,a)},b)}     &\text{step}\\
    \cong~& \rbbod{(x+a)::\mappm{\rb}{(xs,a)}}{b}           &\text{***}\\
    \cong~& ((x+a)+b)::\mappm{\rb}{(\mappm{\rb}{(xs,a)},b)} &\text{step}\\
    \cong~& ((x+a)+b)::\RHS{xs}                             &\text{IH}\\
    \cong~& (x+(a+b))::\RHS{xs}                             &\text{math}\\
    \cong~& \rbbod{x::xs}{(a+b)}                            &\text{step, sym}\\
    \cong~& \RHS{x::xs}                                     &\text{step, sym}
  \end{align*}
  By transitivity of $\cong$, and taking *** on faith, this concludes this
  case and the proof.
\end{description}
\end{proof}

\subsection{Valuability}
To finish the proof of Theorem \ref{thrm:fusion} above, we need to give a
justification for why
$$\mappm{\rb}{((x+a)::\mappm{\rb}{(xs,a)},b)}$$ and $$\rbbod{(x+a)
  ::\mappm{\rb}{(xs,a)}}{b}$$ are equivalent.

Why do we need to be careful here? We make a very similar assertion in the
base case when we said
that $$\ce{\mappm{\rb}{(\vnil,b)}}{\rbbod{\vnil}{b}}$$ In both cases, we're
substituting an argument for a parameter in the body of a function; the
critical difference is that in the base case the arguments to that function
are both values, but in the inductive case one of them is an expression but
not a value.

So we need a rule stating to what a function application is equivalent that
works in both situations. We seem to want to be able to say that, when
given any function
\begin{verbatim}
  fun f (x : A) : B = e1
\end{verbatim}
and any expression \texttt{e2 : A}, $$\ce{(\mappm{f}{e2})}
{\msml{[e2/x]e1}}$$ where $\msml{[e2/x]e1}$ is the expression that results
from substituting $\msml{e2}$ for all instances of $\msml{x}$ in
$\msml{e1}$.

To see why this isn't good enough, consider the function
\begin{verbatim}
  fun f (x : int) : int = 7
\end{verbatim}
and the expression
\begin{verbatim}
  f (1 div 0)
\end{verbatim}
By our proposed rule, we can substitute for all zero occurrences of \sml{x}
in \sml{7} and get $$\ce{\mappm{f}{(1 ~div ~0)}}{7}$$ But SML has a
call-by-value evaluation semantics! So if we actually evaluate the
expression, it raises an error because you can't divide by zero.

Because $\cong$ is an equivalence relation, allowing both interpretations
would mean that $$\ce{\msml{7}}{\msml{raise ~Div}}$$ which this directly
contradicts the definition of $\cong$, where we said that programs that
raise exceptions are not equivalent to programs that do not raise
exceptions. The evaluation semantics of SML aren't going to change, so
something must be wrong with our rule about equivalence of function
application expressions.

To patch up our rule, we need a couple of definitions:
\begin{enumerate}
\item \emph{Definition.} An expression $e$ is valuable if and only if there
  exists some value $v$ such that $\ce{e}{v}$. Specifically to this proof,
  \begin{itemize}
  \item \emph{(pairs)} If $e = (e_1,e_2)$ then $e$ is valuable iff $e_1$ is
    valuable and $e_2$ is valuable.

  \item \emph{(sums)} If $e = e_1+e_2$ then $e$ is valuable iff $e_1$ is
    valuable and $e_2$ is valuable.

  \item \emph{(cons)} If $e = e_1::e_2$ then $e$ is valuable iff $e_1$ is
    valuable and $e_2$ is valuable.

  \item \emph{(app)} If $e = (f ~e_1)$ then $e$ is valuable iff $f$ is
    total and $e_1$ is valuable

  \end{itemize}

\item \emph{Definition.} A function $\oftp{f}{\tarr{\alpha}{\beta}}$ is
  total if and only if for all values $\oftp{v}{\alpha}$, $(\mapp{f}{v})$
  valuable.
\end{enumerate}
Now, we can give the following rule:

\begin{framed}
  \noindent
  Given any function
\begin{verbatim}
  fun f (x : A) : B = e1
\end{verbatim}
and any expression \texttt{e2 : A}, $$\ce{(\mappm{f}{e2})}
{\msml{[e2/x]e1}}$$ if and only if \texttt{e2} is valuable.
\end{framed}

So, finally, the justification that we really wanted at *** is
\begin{quote}
  \begin{enumerate}
    \item
    \begin{itemize}
    \item \sml{+} is total
    \item \sml{x} is assumed to be a value
    \item \sml{a} is assumed to be a value
    \item Therefore, \sml{x+a} is valuable by the rule for sums.
    \end{itemize}

    \item
    \begin{itemize}
    \item \sml{xs} is assumed to be a value
    \item \sml{a} is assumed to be a value
    \item Therefore, \sml{(xs,a)} is valuable by the rule for pairs.
    \end{itemize}

    \item
    \begin{itemize}
    \item \sml{\rb} is total (by Theorem \ref{thrm:total} proven below)
    \item Therefore, \sml{\rb(xs,a)} is valuable by the rule for
    applications.
    \end{itemize}

    \item
    \begin{itemize}
    \item Therefore, \sml{(x+a)::(\rb(xs,a))} is valuable by the rule for
      cons.
    \end{itemize}

    \item
    \begin{itemize}
    \item \sml{b} is assumed to be a value
    \item Therefore, \sml{\rb((x+a)::(\rb(xs,a)), b)} is valuable by the rule
      for pairs.
    \end{itemize}

    \item Finally, then, $$\mappm{\rb}{((x+a)::\mappm{\rb}{(xs,a)},b)}$$ is
      contextually equivalent to $$\rbbod{(x+a) ::\mappm{\rb}{(xs,a)}}{b}$$
      by the valuability argument above and a step.
  \end{enumerate}
\end{quote}
This is very verbose and we'll often suppress most of these concerns---like
the totality of simple SML built-ins, explicitly citing assumptions, and
totality lemmas about structurally recursive functions such as
\sml{\rb}.

The main point, though, is that you can only treat an expression like a
name for a value and step through function application with it when
\emph{you know it will actually produce a value}.

\subsection{Template for Structural Induction on Lists}

Induction is applicable if you're trying to prove a theorem of the form
\begin{quote}
  ``for all $\oftp{\msml{l}}{\tlst{\tint}}$, [some statement about $l$ is
    true]''
\end{quote}
Here's the format that any proof by structural induction on lists should
have:
\begin{framed}
\begin{proof}
The proof is by structural induction on $l$.

\begin{description}
\item [\textbf{Case for \ttt{[]}}]
  To show: [substitute \ttt{[]} into the statement for every instance of $l$] \\
  Proof: \ldots

\item [\textbf{Case for \ttt{x::xs}.}]  Inductive hypothesis: [substitute
  \ttt{xs} into the predicate].\\
  To show: [substitute \ttt{x::xs} into the statement for every instance of $l$].\\
  Proof: \ldots
\end{description}
\end{proof}
\end{framed}

\subsubsection{Critical Observation}
It's critical to note that following this schema does \emph{not} produce a
proof by induction on the length of the list: we're arguing about the
structure of the list directly and length is never involved. It happens to
be the case that lists and natural numbers have a similar structure, but
that's incidental. We never mentioned length in our proof of Theorem
\ref{thrm:fusion} above or Theorem \ref{thrm:total} below.

\subsection{Totality}
\textit{This proof was not presented in lecture, but it's included for
  completeness and because it's a good example of a proof by structural
  induction on lists.}

\begin{theorem}[Totality]
\label{thrm:total}
\sml{\rb} is total.
\end{theorem}

\newcommand{\totalp}[1]{$\exists \oftp{v}{\tlst{\tint}}$ such that $v$ is a
  value and $\ce{\mappm{\rb}{({#1},a)}}{v}$}

\begin{proof}
By the definitions of totality and valuability, it suffices to prove
\begin{quote}
  For all values $\oftpm{l}{\tlst{\tint}}, \oftpm{a}{\tint}$, $\exists
  \oftp{v}{\tlst{\tint}}$ such that $$\ce{\mappm{\rb}{(l,a)}}{v}$$
\end{quote}
We will proceed, therefore, by induction on \sml{l}. In all of the cases
below, let \sml{a} be any values of type $\tint$.

\begin{description}
\item [\textbf{Case for \ttt{[]}}] To show: \totalp{\vnil}\\
  Proof:
  \begin{align*}
          & \mappm{\rb}{(\vnil,a)} &\\
    \cong~& \rbbod{\vnil}{a} & \text{step}\\
    \cong~& \vnil & \text{step}
  \end{align*}
  Choose $v$ to be the value $\oftp{\vnil}{\tlst{\tint}}$. By the
  transitivity of $\cong$, this concludes the case.

\item [\textbf{Case for \ttt{x::xs}.}]
  Inductive hypothesis: \totalp{\msml{xs}}\\
  To show: \totalp{\msml{x::xs}}\\
  Proof:
  \begin{align*}
          & \mappm{\rb}{(\msml{x::xs},a)} &\\
    \cong~& \rbbod{\msml{x::xs}}{a} & \text{step}\\
    \cong~& \msml{(x+a)::\rb(xs,a)} & \text{step}\\
  \end{align*}
  By the inductive hypothesis, \totalp{\msml{xs}}. Therefore, let
  $\oftp{\msml{rl}}{\tlst{\tint}}$ be given such that
  $\ce{\msml{\rb(xs,a)}}{\msml{rl}}$. Continuing from above, this gives us
  \begin{align*}
    \cong~& \msml{(x+a)::rl} & \text{IH}\\
  \end{align*}
  We assume that built in addition on values of type $\tint$ is total, so
  let $\msml{rs}$ be a value such that $$\ce{\msml{rs}}{(x+a)}$$ Again
  continuing from above, this give us
  \begin{align*}
    \cong~& \msml{rs::rl} & \text{+ total}\\
  \end{align*}
  We assume that built in \verb|::| is total, so let $\msml{rf}$ be a value
  such that $$\ce{\msml{rf}}{\msml{rs::rl}}$$ Again continuing from above, this
  give us
  \begin{align*}
    \cong~& \msml{rf} & \text{:: total}\\
  \end{align*}
  Choose $v$ to be the value $\oftp{\msml{rf}}{\tlst{\tint}}$. By the
  transitivity of $\cong$, this concludes the case.

\end{description}
\end{proof}

\subsection{Other Theorems}
The three simple functions discussed above are actually enough to prove a
number of interesting theorems. Here are just a couple, with the proofs
left to the reader.


\begin{theorem}[Self Inverse]
\label{thrm:inverse}
For all values $\oftpm{l}{\tlst{\tint}},
\oftpm{a}{\tint}$, $$\ce{\msml{\rb(\rb{(l,\sim a)},a)}}{\msml{l}}$$
\end{theorem}

\begin{theorem}[Length Preservation]
\label{thrm:length}
For all values $\oftpm{l}{\tlst{\tint}},
\oftpm{a}{\tint}$, $$\ce{\msml{(\mapp{length}{l})}}{\msml{length(rb(l,a))}}$$
\end{theorem}

\begin{theorem}[Sum Scale]
\label{thrm:scale}
For all values $\oftpm{l}{\tlst{\tint}},
\oftpm{a}{\tint}$, $$\ce{\msml{sum(\rb(l,a))}}{\msml{(\mapp{sum}{l}) +
    ((\mapp{length}{\msml{l}}) * a)}}$$
\end{theorem}
\end{document}
