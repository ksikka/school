\documentclass[11pt]{article}
\usepackage{enumerate}
\usepackage{fullpage}
\usepackage{fancyhdr}
\usepackage{amsmath, amsfonts, amsthm, amssymb}
\setlength{\parindent}{0pt}
\setlength{\parskip}{5pt plus 1pt}
\pagestyle{empty}

\def\indented#1{\list{}{}\item[]}
\let\indented=\endlist

\newcounter{questionCounter}
\newcounter{partCounter}[questionCounter]
\newenvironment{question}[2][\arabic{questionCounter}]{%
    \setcounter{partCounter}{0}%
    \vspace{.25in} \hrule \vspace{0.5em}%
        \noindent{\bf #2}%
    \vspace{0.8em} \hrule \vspace{.10in}%
    \addtocounter{questionCounter}{1}%
}{}
\renewenvironment{part}[1][\alph{partCounter}]{%
    \addtocounter{partCounter}{1}%
    \vspace{.10in}%
    \begin{indented}%
       {\bf (#1)} %
}{\end{indented}}

%%%%%%%%%%%%%%%%% Identifying Information %%%%%%%%%%%%%%%%%
%% This is here, so that you can make your homework look %%
%% pretty when you compile it.                           %%
%%     DO NOT PUT YOUR NAME ANYWHERE ELSE!!!!            %%
%%%%%%%%%%%%%%%%%%%%%%%%%%%%%%%%%%%%%%%%%%%%%%%%%%%%%%%%%%%
\newcommand{\myname}{Karan Sikka}
\newcommand{\myandrew}{ksikka@andrew.cmu.edu}
\newcommand{\myhwname}{Assignment 9}
\newcommand{\myrecitation}{E}
%%%%%%%%%%%%%%%%%%%%%%%%%%%%%%%%%%%%%%%%%%%%%%%%%%%%%%%%%%%

\begin{document}
\thispagestyle{plain}

\begin{center}
{\Large \myhwname} \\
\myname \\
\myandrew \\
\myrecitation \\
\today
\end{center}
\begin{question}{Love}
The proof is by contradiction.

\textbf{AFSOC} that $$|L| > |\mathcal{P}(L)|$$ 
This implies that there exists some surjective function $f:L \rightarrow\mathcal{P}(L)$.
In other words, $f$ maps elements from $L$ to elements in $\mathcal{P}(L)$

Let $S$ be a subset of $L$ and define $S = \{b:b \notin f(b)\}$. 
Recall that f(b) is an element in the power set of L, which means it's
actually a subset of L. Again, $b$ is in $S$ iff $b$ is not an element of
the subset of $L$ which f maps to.

For all elements b in $L$, there are two cases:$b \in S$, $b \not\in S$. Say that s
is some element in L such that $f(s) = S$, which must be true for some s
since f is surjective. 

\textbf{Case  $s \in S$}\\
If $s \in S$ then, by definition of $S$ and $f$, $s \notin f(s)$.
This is a contradiction, because recall that $f(s) = S$, 
so  we are stating the equivalent of $s \notin
S$. $s \notin f(s) \implies s\notin S$. This is a contradiction
with the premise of the case.

\textbf{Case  $s \notin S$}\\
If $s \notin S$ then, by definition of $S$, $s \in f(s)$. Recall,
that $f(s) = S$, so we are asserting that $s \in
S$. This is a contradiction with the premise of the case.

we have disproved the original AFSOC statement and we 
can conclude that $|\mathcal{P}(L)| > |L|$.
\end{question}
\begin{question}{Hope and Comfort}
\begin{part}
Call the set of the roots of rational polynomials S

Recall that all polynomials have a finite degree, which is a natural number.
Also recall that a polynomial with degree k has at most k roots.

We can partition the set of all rational polynomials by their degree.
Let $P_k$ be the set of polynomials of degree $k$. Note that a polynomial
of degree $k$ can be uniquely expressed by it's $k+1$ coefficients. You can put
the coefficients in an ordered tuple, so the cardinality of $P_k$ is at most $|\mathbb{Q}^{k+1}|$, which
is countable since the finite tuple of a countable set is countable (proven in concepts hw last year).

By the def'n of partitioning, $R_1 \cup R_2 \cup ... = S$ where $R_i$ is the set of roots of all polynomials in $P_i$. 
Recall that $P_i$ is countable for all $i\in\mathbb{N}$, so the cardinality of $R_i$ is at most $i$ times
the cardinality of $P_i$, so $R_i$ is countable. Therefore its elements can be bijected to the naturals.
Let $C_k$ be a subset of the elements in $S$, but only the ones in $R_i$ where $i < k$, and only the first $k$ elements 
of each $R_i$. The $C_k$ is a finite set: it has cardinality  $k*k$ or $k^2$.  
Now, we can enumerate the elements of the list formed by the concatenation of $C_1C_2...$ except exclude repetitions. Note that this 
enumerates the elements in S, and therefore S is countable. \qed
\end{part}
\begin{part}
Let $f: \{X,O\}^{\infty} \rightarrow \{X,O\}^{\infty} \backslash
\{(X,X,X,\dots)\}$.

Let $s$ be a string in $\{X,O\}^{\infty}$.

Case s has at least one X and no O's: Call $n$ the number of X's $s$ has.
$f$ maps $s$ to a string of $2n-1$ O's.

Case $s$ has at least one O and no X's: Call $n$ the number of O's $s$ has.
$f$ maps $s$ to $2n$ O's.

Case $s$ is not one of the above cases,\\
ie. s must be a combination of X and Os, or it is the empty string:
$f(s) = s$

Injectivity:
Say you have two strings $a,b$ in the domain of $f$ such that $f(a) = f(b)$.
Claim: $f(a)=f(b) \implies a = b$
Proof:

Case $f(a)=f(b)$ has $2n-1$ O's and 0 X's for some $n$ in the naturals, ie $f(a)$ has an odd number of O's and no X's.
Then $f(a)=f(b)$ must have been formed from the first case described above, since the second case only produces
strings with an even number of O's, and the third method only produces strings with X's. $f(a)$ and $f(b)$ both
have $2n-1$ O's so a and b must both have exactly $n$ X's. Then $a=b$.

Case $f(a)=f(b)$ has $2n$ O's and 0 X's for some $n$ in the naturals, ie $f(a)$ has an even number of 0's and no X's.
Then $f(a)=f(b)$ must have been formed from the second case described above, by similar logic above. $f(a)$ and $f(b)$ both
have $2n$ O's so a and b must both have exactly $n$ O's. Then $a=b$.

Case $f(a)=f(b)$ has a positive number of X's. Then it must have been formed by the third case above, because 
that is the only way to produce strings with X's in them. This is the identity function, so $f(a)=f(b) \implies a=b$ by
the definition of the identity function.

Case $f(a)=f(b)$ is the empty string. $a$ and $b$ must both be the empty string and $f(a)$ must have been formed by case 3. Then
$a=b$ since the empty string equals itself.

Surjectivity:
Say you have two strings $a,b$ in the domain of $f$ such that $f(a) = f(b)$.
Claim: $f$ is onto. 
Proof: 
Case 1 of $f$ produces all strings with an odd number of Os and no Xs, for all odd natural numbers.
Case 2 of $f$ produces all strings with an even number of Os and no Xs, for all even natural numbers.
Case 3 of $f$ is the identity function for all inputs not taken care of by Case 1 and 2. 
Recall that the input string must be a combination of X and Os, or it the empty string to be in this case.
Therefore, this case produces combinations of X and Os, or the empty string, by the definition of the 
identity function.

Therefore, $f$ produces strings with a positive number of O's and no X's, and it produces all other combinations
of X and O except strings consisting of X's without O's. Therefore every element in the target of $f$ can be mapped to from the domain of $f$. 
$f$ is onto.

Therefore we have shown a direct bijectiion $f$ between the two sets.
\end{part}
\end{question}
\begin{question}{Patience}
\end{question}
\begin{question}{Grace}
\textbf{Lemma:} For $i,j \in \mathbb{N}, i < j$, there exists an $x\in\mathbb{N}, x\geq1$ such that $i+x$ is prime and 
$j+x$ is composite, or vice versa.

The proof is by induction on $j$

\textbf{Base Case:} If $j=0$, then it is not possible to find an $i$ less than $j$ in the naturals.
So we start with $j = 1$. 

If $j = 1$, then $i$ must be 0. Let $x$ be a prime number greater than $2$. $i+x = 0+x = x$, which we already
established was prime. $j+x =1+x$, which must be an even number since all primes greater than two are odd, and one plus
an odd number is an even number. Therefore, $j+x$ is composite since it is an even number greater than 2.
The claim holds for the base case.

\textbf{Inductive Hypothesis:}
Assume that for some $j \in \mathbb{N}$, the lemma holds all values from 1 upto and including $j$.

\textbf{Inductive Step:}
WTS that the lemma holds for $j+1$.

Case $i=0$:
Let $x$ be a prime factor of $j+1$. Therefore, 
$$x|j+1$$
$$\implies x|(j+1+x)$$
Which means $j+1+x$ is composite since x is greater than 1.
Since $x+0$ is prime and $j+1+x$ is composite, the inductive step holds for this case.

Case $0 < i < j$:
By the IH, there exists an $x$ that satisfies $i-1$ and $j$, by our IH.
This implies that we have some $x_{IH}$ such that one element in $\{(i-1+x_{IH}),(j+x_{IH})\}$ is prime and one is composite.
We can rewrite this as follows:
$$\{(i-1+x_{IH}),(j+x_{IH})\}$$
$$\implies \{(i+(x_{IH}-1)),(j+1+(x_{IH}-1))\}$$
If we set our new $x$ to be $x_{IH}-1$, then we get the set $\{i+x,j+x\}$.
Since we haven't modified the original set in doing so, the claim still holds for this case.

The inductive step holds in all cases of $i$. We have proved the lemma by induction. \qed 

\textbf{Proof:}
AFSOC that the language $L= \{a^c | c\text{ is composite}\}$ can be modeled by a DFA $M$. Say WLOG that M has
$k$ states. Given $k+1$ distinct elements in $L$, at least two elements must terminate on the same
state, by the pidgeonhole principle. Say those strings are $a^i$ and $a^j$. Note that $i\neq j$. Since they terminate
on the same state, they are either both composite or both prime. Append $a^x$ to $a^i$ and $a^j$, and you should
get $a^{i+x}$ and $a^{j+x}$ respectively. When you put them in the
DFA, they will end on the same state (we appended the same number of a's to both strings) which means that they are 
both prime or both composite. This is a contradiction because the lemma says we can choose x such that
either $i+x$ or $j+x$ is composite and the other is prime.

Thus, there is no DFA for which decides the language $L= \{a^c | c\text{ is composite}\}$. The language is irregular. \qed
\end{question}
\begin{question}{Joy}

\begin{part}\\
\textbf{Claim:}\\
$L_1 \cup L_2$ is regular and $L_1$ is finite $\implies L_2$ is regular.

\textbf{Proof:}\\
We know by the Principle of Inclusion/Exclusion that\\
$|L_1 \cup L_2| = |L_1 + L_2| - |(L_1 \cap L_2)|$\\
$\implies |L_2| = |(L_1 \cup L_2)| - |L_1| + |(L_1 \cap L_2)|$\\
$\implies |L_2| = |(L_1 \cup L_2)| - (|L_1| - |(L_1 \cap L_2)|)$\\
Assume $L_1 \cup L_2$ is regular and $L_1$ is finite. Since $L_1$ is finite, $L_1 \cap L_2$ is finite.
Therefore, $L_1 - (L_1 \cap L_2)$ is finite because the difference of two finite languages is finite, 
and therefore a regular language. 
$(L_1 \cup L_2) - (L_1 - (L_1 \cap L_2))$ is then the difference of two regular languages, which, by problem 5 in
this assignment, is a regular language. \qed
\end{part}

\begin{part}\\
\textbf{Claim:}\\
$L_1 \cup L_2$ is regular and $L_1$ is regular $\implies L_2$ is regular.

\textbf{Counterexample:}\\
Let $L_1 = \Sigma^*$ with $\{0, 1\} = \Sigma$. $L_1$ is regular, the DFA is that which has a single accept state
with an arrow to itself, accepting all strings.
Then let $L_2 = \{0^n1^n : n \in \mathbb{N}\}$. $L_2$ is not regular as shown in lecture.
However, $L_1 \cup L_2$ is regular because $L_2 \subseteq L_1$ so $L_1 \cup L_2 = L_1$, which is regular.
\end{part}

\begin{part}\\
\textbf{Claim:}\\
$L_1 \cdot L_2$ is regular and $L_1$ is finite $\implies L_2$ is regular.

\textbf{Counterexample:}\\
Let $L_2 = \{a^c | c \text{ is composite}\}$ which is irregular as proved in problem 3. Let a $L_1 = \{a\}$.
Observe that 2 is the only prime even number and any even number greater than 2 is composite.
Hence, let $L_3$ be a subset of $L_2$ such that $L_3 = \{a^{x} | x>2 \wedge \text{ x is even}\}$,
and the concatenation of the elements in $L_3$ with $a \in L_1$ is $L_4 = \{a^{x} | x>3 \wedge \text { x is odd}\}$.
$L_1 \cdot L_2 = \{a^n | n \geq 4\}$ is regular since $L_3 \subseteq L_2$ and 
$L_1 \cdot L_3 = L_3 \cup L_4 = \{a^n | n \geq 4\}$, and we can make a DFA that reaches an accepting state that self-loops
 after four occurrences of $a$.
 Therefore, it is possible for $L_2$ to not be regular when $L_1 \cdot L_2$ is regular and $L_1$ is finite.
\end{part}

\begin{part}\\
\textbf{Claim:}\\
$L_1 \cdot L_2$ is regular and $L_1$ is regular $\implies L_2$ is regular.

\textbf{Counterexample:}\\
Refer to the proof of (c), and note that if $L_1$ is finite then it is regular. This
was proven in lecture.
\end{part}

\begin{part}\\
\textbf{Claim:}\\
$L_1^*$ is regular $\implies L_1$ is regular.

\textbf{Counterexample:}\\
Let $L_1 = \{a^c | c \text{ is composite}\} \cup \{a\}$.
Proof:  That $L_1$ is not regular.\\
AFSOC $L_1$ is regular. $\{a\}$ is regular because it is finite.
 
If $\{a^c | c \text{ is composite}\} \cup \{a\}$ and $\{a\}$ are regular then $\{a^c | c \text{ is composite}\}$ is
regular because the difference of two regular languages is regular (given in the problem statement for 5).
However, as shown in problem 3, $\{a^c | c \text{ is composite}\}$ is not regular, so this is a contradiction and
what was assumed must be false. Therefore the claim is not true.

Then $L_1^* = \{a^n | n \in \mathbb{N}\}$ because $L_1$ contains $\{a\}$. 
This is a regular language because a DFA with a single accept state which loops back to itself and 
accepts any string of $a$'s will accept $L_1^*$.
Therefore, we have shown an irregular $L_1$  and a regular $L_1^*$.
\end{part}
\end{question}
\end{document}
