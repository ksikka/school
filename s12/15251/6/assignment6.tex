\documentclass[11pt]{article}
\usepackage{enumerate}
\usepackage{fullpage}
\usepackage{fancyhdr}
\usepackage{amsmath, amsfonts, amsthm, amssymb}
\setlength{\parindent}{0pt}
\setlength{\parskip}{5pt plus 1pt}
\pagestyle{empty}

\def\indented#1{\list{}{}\item[]}
\let\indented=\endlist

\newcounter{questionCounter}
\newcounter{partCounter}[questionCounter]
\newenvironment{question}[2][\arabic{questionCounter}]{%
    \setcounter{partCounter}{0}%
    \vspace{.25in} \hrule \vspace{0.5em}%
        \noindent{\bf #2}%
    \vspace{0.8em} \hrule \vspace{.10in}%
    \addtocounter{questionCounter}{1}%
}{}
\renewenvironment{part}[1][\alph{partCounter}]{%
    \addtocounter{partCounter}{1}%
    \vspace{.10in}%
    \begin{indented}%
       {\bf (#1)} %
}{\end{indented}}

%%%%%%%%%%%%%%%%% Identifying Information %%%%%%%%%%%%%%%%%
%% This is here, so that you can make your homework look %%
%% pretty when you compile it.                           %%
%%     DO NOT PUT YOUR NAME ANYWHERE ELSE!!!!            %%
%%%%%%%%%%%%%%%%%%%%%%%%%%%%%%%%%%%%%%%%%%%%%%%%%%%%%%%%%%%
\newcommand{\myname}{Karan Sikka}
\newcommand{\myandrew}{ksikka@andrew.cmu.edu}
\newcommand{\myhwname}{Assignment 6}
\newcommand{\myrecitation}{E}
%%%%%%%%%%%%%%%%%%%%%%%%%%%%%%%%%%%%%%%%%%%%%%%%%%%%%%%%%%%

\begin{document}
\thispagestyle{plain}

\begin{center}
{\Large \myhwname} \\
\myname \\
\myandrew \\
\myrecitation \\
\today
\end{center}
\begin{question}{Like a fly in a web}
\textbf{Part a}

Let $P(n) = $``For all simple graphs with $n$ vertices and $m$ edges, 
this algorithm produces a bipartition in which 
at least $\frac{m}{2}$ of the edges cross the partition."

The proof is by induction on $n$.

\textbf{Base case:} $(n=1)$\\
$m$ must be 0 since there are no vertices to connect to. 
The algorithm does nothing, and $\frac{m}{2} = 0$ so $P(1) =$ true.

\textbf{Inductive Hypothesis:}
For some $n \in \mathbb{N}$, for all simple graphs with $n$ vertices and $m$ edges, 
this algorithm produces a bipartition in which 
at least $\frac{m}{2}$ of the edges cross the partition.

\textbf{Inductive Step:}
Add a vertex and connect it to $k$ of the $n$ other vertices. Then there are 
a total of $m+k$ edges

Place $n$ vertices on the left or the right according to the algorithm. By the IH, 
we know at least $\frac{m}{2}$ edges cross the center.

You have two options for the $(n+1)^{th}$ vertex: place it on the left or on the right.
If you place it on the left a set of edges $A$ will cross the center. If you place it
on the right, a set of edges $B$ will cross the center. $A \cap B = \null$ since 
no edge will cross the center in both cases (duh). Therefore, $|A| + |B| = k$.

The algorithm says to place the last vertex on the left or right so that 
the larger of $A$ and $B$ crosses the center. Without loss of generality, we'll
say that $A$ is larger and therefore crosses the center. 
\begin{align*}
|A| + |B| &= k\\
2|A| \geq |A| + |B| &= k\\
|A| &\geq \frac{k}{2}
\end{align*}

Therefore, at least $\frac{m}{2}+\frac{k}{2} = \frac{m+k}{2}$ edges will be cut.

\textbf{Part b}

\textbf{Claim:}\\
If the planar graph $G$ is isomorphic to its dual $G'$, then $G'$ has $2n-2$ edges.

Let $G$ have $n$ vertices, $e$ edges, and $f$ faces.
Let $G'$ have $n'$ vertices, $e'$ edges, and $f'$ faces.

\textbf{Fun Facts:}
\begin{enumerate}
\item Since $G \simeq G'$, there is a bijection between their vertex sets, so $n=n'$.
\item According to the definition stated in the problem, there is also a bijection 
between the sets of edges, so $e=e'$.
\item Since $G'$ is the dual of $G$, $n' = f$ because to form a dual of a graph, you 
put one vertex for every face of the graph.
\end{enumerate}

\textbf{QEDmyGraph:}\\
Since $G'$ is the dual of $G$, and $G$ is planar, we know that $G'$ is planar
and therefore $n' + f' = e' + 2$
\begin{align*}
e' &= n' + f' - 2\\
   &= n + f' - 2\\
   &= n + n - 2\\
e' &= 2n - 2
\end{align*}
QED
\end{question}
\begin{question}{Oh no, not again}
WTS: Consider a regular, bipartite graph G whose edges go only between A and B. 
Observe that the graph is regular (every node in G has the same degree). 
Assuming G has at least one edge, show that G has a perfect matching.
Also show that the edges of G are a union of disjoint perfect matchings.

(part a is by the lecture notes.)

Proof that a regular bipartite graph is the disjoint union of the edges of perfect matchings:
Say you have a regular bipartite graph where all nodes have degree $k$. By the lemma proved in 
part A, it must have a perfect matching. Remove the edges in that perfect matching. By definition 
of perfect matching, all nodes now have degree $k-1$. Lo and behold, another regular bipartite graph. 
Therefore it must have a perfect matching. Continue to remove the edges in the perfect matchings 
until all the nodes have degree 0, ie. there are no more edges to remove. Therefore, if we added 
the sets of edges that we 
removed at each step, we'd see that their union gets us back to the original graph. Also
because of the method used to remove the sets of edges, they are clearly disjoint.

\end{question}
\begin{question}{The Blob}
 Suppose Mutant Town starts off with only the axiom K3 . Show that in the resulting
 system, the set of theorems is the same as the set of graphs which are NOT 2-colorable.
\begin{enumerate}[a.]
\item First, note that graphs where $n\leq 2$ are obviously colorable.
Also, using the axiomatic system, you cannot form graphs where $n\leq 2$, since 
the only axiom has 3 nodes and there is no operation which decreases the number of nodes.
Therefore the claim holds for all graphs where $n=0,1,2$.

\textbf{Claim:} If the only axiom is $K_3$, then the resulting set of theorems is not 2-colorable.

We proceed by structural induction on the number of vertices.

\textbf{Base case:}\\
The axiom $K_3$ is the complete graph with 3 nodes. This is not 2 colorable because it is
an odd cycle. 

\textbf{Inductive Hypothesis:}\\
Assume the claim to be true for all graphs where $n<k$ for some $k\in\mathbb{N}$$G$ be a graph formed from the axiomatic system which is not 2-colorable.

\textbf{Inductive Step:}\\
Case GROW: The claim holds if you apply grow to any graph in the axiomatic system where $n<k$. By the IH, these graphs are not 2-colorable. By adding an edge or vertex, the subgraph
which is not 2-colorable still exists in the graph. Therefore, the graph is not 2 colorable.\

Case CRUSH: Say you are crushing 2 not-2-colorable graphs by the IH. Then these graphs both contain an odd cycle, because they are not
2 colorable. The crush operation does not change that fact. All crush does is modify the neighbor-set of 2 vertices, and it does
not remove anything from the neighbor-set since it performs the union. Therefore CRUSH cannot eliminate any odd cycles. The resulting
graph must not be 2-colorable because it contains an odd cycle.

Case BLOB: Again, BLOB does not eliminate odd cycles. It is like CRUSH, but it also adds an edge. The key is that it does not remove 
anything from the neighbor-set, which is why it cannot eliminate odd cycles. More detail
(u,v) (x,y)
It merges u and x into the same vertex which I will call x.
(x,v) (x,y)
It then adds the edge (v,y). You can see that we have not removed the odd cycles in any way, we've only possibly created another
odd cycle.

Therefore, we have shown by induction that no graph produced by the axiomatic system is 2-colorable. Now we must show that 
any graph which is not 2-colorable can be produced by the axiomatic system. However, I didn't get up to this part.

\item
\textbf{Claim:} Given only the axiom $K_4$, no resulting theorem is 3-colorable.

We proceed by structural induction on the number of vertices.

\textbf{Base Case: }\\
The graph $K_4$ contains the subgraph $K_3$, which is 3-colorable. Then add a node to the graph, and connect every node to it by an edge. 
No matter what color you choose for the new node, it will connect to a node with the same color. Therefore $K_4$ is not 3-colorable.

\textbf{Induction Hypothesis: }\\
Any theorem that is a result of $k$ or fewer deductive rules is not $3$-colorable. \\

\textbf{Inductive Step: } In each of our three cases we show that after an additional deductive rule, the claim still holds.\\

\textbf{Case GROW: } We are given a theorem $G$ which is not 3-colorable by the IH. 
The application of GROW results in a new theorem $G'$. GROW only adds a vertex or an edge, 
Therefore $G$ is a subgraph of $G'$. Since $G$ is not $3$-colorable, $G'$ is not $3$-colorable. \\

\textbf{Case CRUSH: } $G$ and $H$ which is are not 3-colorable by our induction hypothesis. 
There must exists a node $n$ where one neighbor of $n$ is the same color as $n$, since $G$ is not 3-colorable. 
Let $G_s$ be the subgraph of $G$ which is not 3-colorable. If we apply CRUSH($u$,$v$) such that $u,v \notin G_s$ 
we see that $G_s$ remains unaffected. So, the resulting theorem has the subgraph $G_s$, 
and is therefore not $3$-colorable. \\
In the other case, you do CRUSH($u$,$v$) such that at least one of $u,v \in G_s$. CRUSH does not remove edges so by the same reasoning in
part A, the resulting theorem is not 3-colorable.

\textbf{BLOB: } We are given two theorems $G_1$ and $G_2$ which are not 3-colorable by the IH. 

I did not complete the rest of the proof.
\end{enumerate}
\end{question}
\begin{question}{Running in Circles}
For this proof, let the term ``duo'' denote a subgraph that consists of two parallel edges between two nodes.

Let $k_{n}$ be the number of labeled graphs $G$ with $n$ vertices where $G$ is isomorphic to its own edge graph, $\widetilde{G}$. 

$$k_n = k_{n-1} + (n-1)k_{n-1} - \binom{n-1}{2}k_{n-3}$$

To create a graph with $n$ nodes from a graph with $n-1$ nodes, there are two ways to proceed. First add a node.
Either:
\begin{enumerate}
\item Connect the node to the (n-1) graph
\item Do not connect the node to the rest of the nodes. Instead give it a loop.
\end{enumerate}

For the second case, this results in $k_{n-1}$ graphs of size $n$ because there is only 1 way to add
a node with a loop to the (n-1) graph.

For the first case, there are $n-1$ possible spots to insert a new $n^{th}$ node in each graph; this 
results in $(n-1)k_{n-1}$ graphs of size $n$. However, for each case in the $k_{n-1}$ graph arrangements with a duo,
we overcount the number of graphs. Notice that we count each edge of the duo, when in reality, there is only one 
new graph formed by inserting the node into the duo.

To account for the overcounting, we subtract the set of all the $n$-size graphs that result 
from inserting a new node into a $(n-1)$-size graph with a duo. In order to determine the 
total number of $(n-1)$-size graphs that contain a duo, we look at $k_{n-3}$ because it represents the 
total number of $(n-1)$-size graphs without the 2 nodes that make up the duo.

We multiply $k_{n-3}$ by $\binom{n-1}{2}$ because of the Rule of Produce, and because we're choosing 2 labeled
nodes. 

Therefore, by the Rule of Sum, the number of labeled graphs on $n$ vertices is 

$$k_n = k_{n-1} + (n-1)k_{n-1} - \binom{n-1}{2}k_{n-3}$$
$$= nk_{n-1} + \frac{(n-1)(n-2)}{2}k_{n-3}$$ 

\end{question}
\begin{question}{Pizzadeliv'ryman}
\textbf{Claim:} For all $n\in\mathbb{N}$, for all complete directed graphs with $n$ nodes, there exists a path 
in $G$ which touches all $n$ nodes.

\textbf{Base Case:}\\
$n=0$\\
There are no nodes, so the path which satisfies the claim is the empty path.

$n=1$\\
There is one node. The path which satisfies the claim is the path with that node.

\textbf{Inductive Hypothesis:}\\
For some $k\in\mathbb{N}$, all complete directed graphs with $k$ nodes have a path 
that touches all $k$ nodes.


\textbf{Inductive Step:}\\
Consider a complete directed graph G with $k$ nodes. Therefore, by the IH, there 
exists a path $p$ such which contains every node. Now, add a node $f$, and connect every
node in $G$ with an edge to $f$. Therefore this new graph, $G'$, is complete.

Paths must have 2 endpoints. Since these graphs are directed, their paths will
also be directed. Say $p$, from above, has two endpoints $(e_1,e_2)$ such that
the edge connected to $e_1$ points away from $e_1$, and the edge connected to $e_2$ 
points toward $e_2$.

Remember that node $f$ from above is connected to $e_1$ and $e_2$. We 
call the edge that connects $f$ to $e_1$ $a$ and the $b$ is the edge which 
connects $f$ to $e_2$.

Case f \verb|-->| $e_1$\\
prepend f to the path p

Case $e_2$ \verb|-->| f\\
append f to the path p

Case f \verb|-->| $e_1$ and $e_2$ \verb|<--| f\\
Traverse path $p$ from $e_1$ to $e_2$, and at each node in the way,
look at the edge which connects it to $f$. At $e_1$, the node is pointing
to $f$, but by the time you get to $e_2$, the node is pointing from $f$ to $e_2$.
This change of direction means that somewhere along the way, the direction 
alternated between to $f$ and from $f$. Then you can put $f$ in between those two
nodes such that $e_1$ \verb|-->| ... \verb|-->| $x$ \verb|->| $f$ \verb|->| $y$ \verb|-->| ... \verb|-->| $e_2$. 

\end{question}
\begin{question}{On the Run}
%%Put your answer here
\end{question}
\end{document}
