\documentclass{article}
\parindent=0pt
\newcommand{\op}[1]{\textbf{#1}}
\newcommand{\lcalc}[1]{\begin{center}\ensuremath{#1}\end{center}}
\newcommand{\error}[1]{Something went wrong.}
\newcommand{\begintodo}[1]{\begin{center}define the function \op{#1} here\end{center}}
\newcommand{\begindone}[1]{\begin{center}\op{#1} was defined here\end{center}}
\newcommand{\finishtodo}[1]{}
\newcommand{\finishdone}[1]{}
\newcommand{\test}[2]{#1 = #2\par}
\newcommand{\begintests}[1]{\begin{center}tests for the function(s) \op{#1} here\\}
\newcommand{\finishtests}[1]{\end{center}}
\renewcommand{\c}[2]{\texttt{cons}\{#1\}\{#2\}}

\begin{document}
\begin{center}
    {\large Lambda\LaTeX: Implementing the Lambda Calculus in \LaTeX}
\end{center}
\section{Introduction}
In this question, you will build up an implementation of the lambda calculus using only
built-in \LaTeX$\;$(macro) features.  Keep in mind that a lot of the lambda expressions
were given to you in the lecture, and it would be a good idea to refer to it periodically.
Before you begin, here's a guide to how \LaTeX macros work.  To create a new macro
called ``foobar'', you write the following line of \LaTeX$\;$code:
\begin{center}
\verb+\newcommand{\foobar}[X]{WHAT_YOU_WANT_IT_TO_DO}+
\end{center}

\verb+\foobar+ is the name of the command; \verb+X+ is the number of arguments it takes 
(this parameter is optional and you can omit the square brackets entirely); \verb+WHAT_YOU_WANT_IT_TO_DO+ is the body
of the macro (the part where you actually tell it what to do.  To reference arguments in your function use \verb+#1+, \verb+#2+, etc.

\textbf{You will want to follow along in the tex source as well as the pdf, because you will implement macros
in this tex file.}


The first function we will define is the identity function \op{id}:

\lcalc{\op{id} \equiv \lambda x.x}

\begindone{id}
\newcommand{\id}[1]{#1}
\finishdone{id}

Please put all definitions that you write in this file and expect to be graded between the \verb+begintodo+ and \verb+finishtodo+ macros.
These will be used to slurp pieces of your file out for grading.  Please don't delete them or you risk getting a zero on this problem.

\begintests{id}
\test{\id{10}}{10}
\test{\id{1337}}{1337}
\finishtests{id}

Most functions will have tests pre-written for you (like the above).  They will start commented out so that the tex file compiles, but you should feel free to
uncomment them to test your code.

The last piece of \LaTeX$\;$syntax that we need to describe is macro (function) application.  In the lambda calculus, if we write $\lambda x. \lambda y. (y x)$, we take
the inner part to mean ``give $x$ to $y$ as an argument'' (in other words, apply $x$ to $y$).  An analogous \LaTeX$\;$macro would be defined as follows:
\begin{center}
    \verb+\newcommand{\func}[2]{#2{#1}}+
\end{center}
There are two important things to note about this implementation of the lambda expression.  The first is that by saying the macro takes two arguments, we are mirroring that
the lambda expression is abstracted twice (intuitively, there are two arguments).  The second is that in \LaTeX, \verb+#2 #1+ would \textbf{NOT} be function application.
To apply a function in \LaTeX, you \textbf{MUST} use curly braces.  Try this out by defining the pairing function as given in lecture:

\lcalc{\op{pair} \equiv \lambda x. \lambda y. \lambda f. f x y}

% TODO: You need to define this macro
\begintodo{pair}
\newcommand{\pair}[3]{#3{#1}{#2}}
\finishtodo{pair}

\section{Booleans}
To really get started, we need a couple more functions which we define for you:

\lcalc{\op{true} \equiv \lambda x . \lambda y. x}
\begindone{true}
\newcommand{\true}[2]{#1}
\finishdone{true}

\lcalc{\op{false} \equiv \lambda x . \lambda y. y}
\begindone{false}
\newcommand{\false}[2]{#2}
\finishdone{false}

\begintests{booleans}
\test{\false{10}{20}}{20}
\test{\true{10}{20}}{10}
\finishtests{booleans}

If we're going to write some real functions, we're probably going to need the normal 
boolean operations: \op{and}, \op{or}, \op{not}.  Define these functions here:

% TODO: You need to define this macro
\begintodo{and}
\newcommand{\cAnd}[2]{#1{#2}{#1}}
\finishtodo{and}

% TODO: You need to define this macro
\begintodo{or}
\newcommand{\cOr}[2]{#1{#1}{#2}}
\finishtodo{or}

% TODO: You need to define this macro
\begintodo{not}
\newcommand{\cNot}[3]{#1{#3}{#2}}
\finishtodo{not}

\begintests{boolean functions}
\test{\cAnd{\false}{\true}{1}{0}}{0}
\test{\cAnd{\false}{\false}{1}{0}}{0}
\test{\cAnd{\true}{\true}{1}{0}}{1}
\test{\cOr{\false}{\true}{1}{0}}{1}
\test{\cOr{\false}{\false}{1}{0}}{0}
\test{\cOr{\true}{\true}{1}{0}}{1}
\test{\cNot{\true}{1}{0}}{0}
\test{\cNot{\false}{1}{0}}{1}
\finishtests{boolean functions}

One more thing we'll need is the ``if-then-else'' operator we covered in lecture.
% TODO: You need to define this macro
\begintodo{ifthenelse}
\newcommand{\ifthenelse}[3]{#1{#2}{#3}}
\finishtodo{ifthenelse}

\begintests{ifthenelse}
\test{\ifthenelse{\true}{1}{\error}}{1}
\test{\ifthenelse{\false}{\error}{2}}{2}
\finishtests{ifthenelse}

\section{Lists}
Now that we have basic booleans, let's continue making useful constructs to get to
pairs and lists.  If we have a pair, we'd like to be able to get the first and
second elements out of the pair.  We use \op{first} and \op{second} for this:

\lcalc{\op{first} \equiv \lambda p . p (\lambda x. \lambda y. x)}
\begindone{first}
\newcommand{\first}[1]{#1{\true}}
\finishdone{first}

\lcalc{\op{second} \equiv \lambda p . p (\lambda x. \lambda y. y)}
\begindone{second}
\newcommand{\second}[1]{#1{\false}}
\finishdone{second}

Now that we have pairs, we'd like to represent entire lists.  We use
cons cells like in the lecture. We represent the list $[1, 2, 3]$ as
$(1\;(2\;(3\;\texttt{nil})))$. We can construct lists by repeated 
applications of $\texttt{cons}$. To get the above list, we use
$$\c{1}{\c{2}{\c{3}{\texttt{nil}}}}$$
\op{nil} is easy enough to implement:

\lcalc{\op{nil} \equiv \op{false}}
\begindone{nil}
\newcommand{\nil}{\false}
\finishdone{nil}

\lcalc{\op{cons} \equiv \op{pair}}
\begindone{cons}
\newcommand{\cons}{\pair}
\finishdone{cons}

To finish off our list definitions, we define \op{car}, which gets the first
element of the list, and \op{cdr} which gets everything but the first element
of the list.  Here they are:

\begindone{car}
\newcommand{\car}{\first}
\finishdone{car}
\begindone{cdr}
\newcommand{\cdr}{\second}
\finishdone{cdr}

We would like to be able to tell if a list is empty; so, we define a macro \op{isempty}
which returns \op{true} if the list is empty and \op{false} otherwise.  Implement this
function here (Hint: You'll want to use a helper function).

% TODO: You need to define this macro
\begintodo{isempty}
\newcommand{\help}[2]{\cNot}
\newcommand{\isempty}[1]{#1{\help}{\true}}
\finishtodo{isempty}


\section{Recursion}
As we discussed in lecture, we can use a fixed-point combinator to create recursive
functions in the lambda calculus.  We present two of them for you here:

\begindone{fix}
\newcommand{\fixer}[3]{#2{#1{#1}{#2}}{#3}}
\newcommand{\om}[1]{#1{#1}}
\newcommand{\fix}{\om{\fixer}}
\finishdone{fix}

\begindone{Y}
\newcommand{\dupPrint}{\ensuremath{\lambda f. \lambda x . f (x x)}}
\newcommand{\dup}[2]{#1 {#2{#2}}}
\newcommand{\Y}[1]{\dup{#1}{\dup{#1}}}
\finishdone{Y}

Throughout the rest of this question, you'll probably want to use the Y-combinator,
but you could exchange it for the other one if you wanted to!

Now let's use the Y-combinator to actually create a recursive function!  \op{listId}
takes in a list as an argument, recurses over every element in the list, and returns
the list unchanged.

\begindone{listId}
\newcommand{\recListId}[2]{\ifthenelse{\isempty{#2}}{\nil}{\cons{\car{#2}}{#1{\cdr{#2}}}}}
\newcommand{\listId}[1]{\Y {\recListId} {#1}}
\finishdone{listId}


To get a feel for how this type of recursion works, define the following function:
\op{printlist} which prints out the list in a readable way.  For example, if the
given list is \c{1}{\c{2}{\c{3}{\texttt{nil}}}}, then \op{printlist} should output
\begin{center}
(1 (2 (3 nil)))
\end{center}

% TODO: You need to define this macro
\begintodo{printlist}
\newcommand{\helpthree}[2]{\ifthenelse{\isempty{#2}}{nil}{(\first{#2} {#1{\second{#2}}})}}
\newcommand{\printlist}[1]{\Y{\helpthree}{#1}}
\finishtodo{printlist}

\begintests{printlist}
\newcommand{\lstA}{\cons{1}{\cons{2}{\cons{3}{\nil}}}}
\newcommand{\lstB}{\cons{1}{\cons{2}{\nil}}}
\newcommand{\lstC}{\cons{1}{\nil}}
\test{\printlist{\nil}}{nil}
\test{\printlist{\lstC}}{(1 nil)}
\test{\printlist{\lstB}}{(1 (2 nil))}
\test{\printlist{\lstA}}{(1 (2 (3 nil)))}
\finishtests{printlist}

\section{Church Numerals}
As in lecture, we'd like a way of representing numbers; so, we turn to Church Numerals.  We'll
define zero for you:

\begindone{z}
\newcommand{\z}[1]{\id}
\finishdone{z}

You define successor:

% TODO: You need to define this macro
\begintodo{s}
\newcommand{\s}[3]{#2{#1{#2}{#3}}}
\finishtodo{s}

...and plus:

% TODO: You need to define this macro
\begintodo{plus}
\newcommand{\plus}[4]{#1{#3}{#2{#3}{#4}}}
\finishtodo{plus}

Just like with lists, we need a way of testing for the base case.  Please define
\op{iszero}:

% TODO: You need to define this macro
\begintodo{iszero}
\newcommand{\helptwo}[1]{\false}
\newcommand{\iszero}[1]{#1{\helptwo}{\true}}
\finishtodo{iszero}

Here's predecessor (which is still really annoying):

\begindone{p}
\newcommand{\phelpa}[3]{#3{#2{#1}}}
\newcommand{\phelpb}[2]{#1}
\newcommand{\p}[3]{#1{\phelpa{#2}}{\phelpb{#3}}{\id}}
\finishdone{p}

We provide two ways of printing church numerals (via unary and via \LaTeX's
counters):

\begindone{unary}
\newcommand{\recUnary}[2]{\ifthenelse{\iszero{#2}}{}{1{#1 {\p{#2}}}}}
\newcommand{\unary}[1]{\Y{\recUnary}{#1}}
\finishdone{unary}

\begindone{printnum}
\newcounter{num}
\newcommand{\recPrintNum}[2]{\ifthenelse{\iszero{#2}}{}{\addtocounter{num}{1}{#1{\p{#2}}}}}
\newcommand{\printnum}[1]{\setcounter{num}{0}{\Y{\recPrintNum}{#1}}\arabic{num}}
\finishdone{printnum}

\begintests{church numerals}
\test{\printnum{\s{\s{\s{\s{\z}}}}}}{4}
\test{\unary{\plus{\s{\z}}{\s{\s{\z}}}}}{111}
\finishtests{church numerals}

\section{Putting It All Together}
Implement an \op{append} function which takes in two arguments $x$ and $L$, and appends $x$ to the list $L$.

% TODO: You need to define this macro
\begintodo{append}
\newcommand{\helpfour}[3]{\ifthenelse
        {\isempty{#3}}
        {\cons{#2}{#3}}
        {\cons{\car{#3}}{#1{#2}{\cdr{#3}}}}
    }
\newcommand{\append}[2]{\Y{\helpfour}{#1}{#2}}
\finishtodo{append}

Implement a \op{reverse} function which takes in a list $L$ and returns a new list with the elements of $L$ 
in reverse order.

% TODO: You need to define this macro
\begintodo{reverse}
\newcommand{\helpfive}[2]{\ifthenelse
        {\isempty{#2}}
        {#2}
        {\append{\car{#2}}{#1{\cdr{#2}}}}
    }
\newcommand{\reverse}[1]{\Y{\helpfive}{#1}}
\finishtodo{reverse}

Finally, implement a function \op{range} which takes in a church numeral which represents the number $n$ and
outputs the list \c{1}{\c{2}{\dots\c{n}{\texttt{nil}}\dots}}:

% TODO: You need to define this macro
\begintodo{range}
\newcommand{\helpsix}[2]{\ifthenelse{\iszero{#2}}
    {\nil}
    {\cons
        {\printnum{#2}}
        {#1{\p{#2}}}
    }
}
\newcommand{\range}[1]{\reverse{\Y{\helpsix}{#1}}}
\finishtodo{range}

\begintests{range}
\test{\printlist{\range{\s{\s{\s{\s{\z}}}}}}}{(1 (2 (3 (4 nil))))}
\finishtests{range}
\end{document}
