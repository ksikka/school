\documentclass[11pt]{article}
\usepackage{enumerate}
\usepackage{fullpage}
\usepackage{fancyhdr}
\usepackage{amsmath, amsfonts, amsthm, amssymb}
\setlength{\parindent}{0pt}
\setlength{\parskip}{5pt plus 1pt}
\pagestyle{empty}

\def\indented#1{\list{}{}\item[]}
\let\indented=\endlist

\newcounter{questionCounter}
\newcounter{partCounter}[questionCounter]
\newenvironment{question}[2][\arabic{questionCounter}]{%
    \setcounter{partCounter}{0}%
    \vspace{.25in} \hrule \vspace{0.5em}%
        \noindent{\bf #2}%
    \vspace{0.8em} \hrule \vspace{.10in}%
    \addtocounter{questionCounter}{1}%
}{}
\renewenvironment{part}[1][\alph{partCounter}]{%
    \addtocounter{partCounter}{1}%
    \vspace{.10in}%
    \begin{indented}%
       {\bf (#1)} %
}{\end{indented}}

%%%%%%%%%%%%%%%%% Identifying Information %%%%%%%%%%%%%%%%%
%% This is here, so that you can make your homework look %%
%% pretty when you compile it.                           %%
%%     DO NOT PUT YOUR NAME ANYWHERE ELSE!!!!            %%
%%%%%%%%%%%%%%%%%%%%%%%%%%%%%%%%%%%%%%%%%%%%%%%%%%%%%%%%%%%
\newcommand{\myname}{Karan Sikka}
\newcommand{\myandrew}{ksikka@andrew.cmu.edu}
\newcommand{\myhwname}{Assignment 7}
\newcommand{\myrecitation}{E}
%%%%%%%%%%%%%%%%%%%%%%%%%%%%%%%%%%%%%%%%%%%%%%%%%%%%%%%%%%%

\begin{document}
\thispagestyle{plain}

\begin{center}
{\Large \myhwname} \\
\myname \\
\myandrew \\
\myrecitation \\
\today
\end{center}
\begin{question}{A Long Expected Party}
\begin{enumerate}[a.]
\item $(G, [,,])$ follows the rules Lhach and Nar, so it is an Elen.
\textbf{Proof of Llach}\\
$[a,b,c] = a \circ b^{-1} \circ c$\\
$\implies$ $[[a,b,c],d,e] = a \circ b^{-1} \circ c \circ d^{-1} \circ e$.

$[d,c,b] = d \circ c^{-1} \circ b$\\
$\implies$ $[a,[d,c,b],e] = a \circ (d \circ c^{-1} \circ b)^{-1} \circ e = a \circ b^{-1} \circ c \circ d^{-1} \circ e$.

$[c,d,e] = c \circ d^{-1} \circ e$\\
$\implies$ $[a,b,[c,d,e]] = a \circ b^{-1} \circ c \circ d^{-1} \circ e$

From the above, it is clear that $[[a,b,c],d,e] = [a,[d,c,b],e] = [a,b,[c,d,e]]$. Therefore, Llach is satisfied.

\textbf{Proof of Nar}\\
$[a,a,b] = a \circ a^{-1} \circ b = 1 \circ b = b$

$[b,a,a] = b \circ a^{-1} \circ a = b \circ 1 = b$

Since $[a,a,b] = [b,a,a] = b$, Nar is satisfied.

Since Lhach and Nar are both satisfied, $(G, [,,])$ is an Elen.

\item $(A,*)$ is a group if it is closed, associative, contains the identity $u$, and has an inverse.

Let $a$, $b$, and $c$ be in $A$.

\textbf{Closure:} (to show $a*b \in A$)
This is true by the definition of $*$.

\textbf{Associativity:} (to show $a*(b*c) = (a*b)*c$)\\
$b*c = [b,u,c]$\\
$\implies a*(b*c) = [a,u,[b,u,c]]$

$a*b = [a,u,b]$\\
$\implies (a*b)*c = [[a,u,b],u,c]$

By the rule of Lhach, the above two results may be equated,
and $(A,*)$ is associative.

\textbf{Identity:} (to show $a*u=a=u*a$)\\
$a*u = [a,u,u] = a$ (Nar)

$u*a = [u,u,a] = a$ (Nar)

Therefore, $a*u = a = u*a$ and the identity exists.

\textbf{Inverse:} (to show $a*a^{-1}=u=a^{-1}*a$)\\


$a^{-1} = [u,a,u]$

Right Inverse:\\
$a*a^{-1} = a*[u,a,u]$ (substitution)\\
$= [a,u,[u,a,u]]$ (Def of $*$)\\
$= [a,[a,u,u],u]$ (Llach)\\
$= [a,a*u,u]$ (Def of $*$)\\
$= [a,a,u]$ (Identity)\\
$= u$ (Nar)

Left Inverse:\\
$a^{-1}*a = [u,a,u]*a$ (Def of $*$)\\
$[[u,a,u],u,a]$ (Def of $*$)\\
$= [u,[u,u,a],a]$ (Llach)\\
$= [u,u*a,a]$ (Def of $*$)\\
$= [u,u,a]$ (Identity)\\
$= u$ (Nar)

Since $a*a^{-1}=u=a^{-1}*a$, $a^{-1}$ is an inverse such that $a^{-1} = [u,a,u]$.

Therefore, we have shown that $(A,*)$ is a group.
\end{enumerate}
\end{question}


\begin{question}{The Council of Elrond}

Let there be $n$ members in the Council, where $n$ is a positive integer.
Enumerate the members of the Council in ascending order starting from Frodo with 1.
Instead of a circle, put them in a line:

\verb| 1  2  ...  n-1  n |

Elrond chooses a person, skips k people, and chooses the next person,
where k starts at 1 and increases by 1 every iteration.
Call $E(k)$ the index of the kth person chosen.
$$E(1) = 1$$
$$E(k) = k + E(k-1)$$
Upon inspection, we see that this is also one more than the sum of positive integers up to k mod n.
$$ E(k) = 1+(\sum_{i=1}^{k} i \mod n) = 1 + (\frac{k(k+1)}{2} \mod n)$$

\textbf{Claim:} When $n$ is a power of two, all members of Elrond are chosen eventually.
In mathematical terms, When $n = 2^i$ for some integer $i$, $E(k)$ will assume all values
from 1 to $n$, where $k$ is a positive number. More strongly, it is true that E(k) will cover
all these values using $k$ from $1$ to $2k$.

\textbf{Claim:} When $n$ is not a power of two, at least one member of the council of Elrond 
will never be chosen. In math terms, there exists an $m$ between 1 and $n$ such that
$E(k) \neq m$ for all positive integer values of $k$.


All members of the council will be chosen eventually if there a power of two members in the council.
\end{question}


\begin{question}{Battle of the Hornburg}
a. Let $a,b \in G$ where G is a group under the operation $*$, with the 
special property described in the problem.
We know $\forall u \in G, u^2 = u*u = e$ where $e$ is the identity, 
since each element has an order of at most 2. Consider:
. \begin{align*}
a*b &= e*a*b*e & \text{(identity)}\\
    &= b^2*a*b*a^2 & \text{(order at most 2)}\\
    &= b*(b*a)^2*a & \text{(closure and associativity)}\\
    &= b*e*a & \text{(order at most 2)}\\
    &= b*a & \text{(identity)}
    \end{align*}

b.
To prove that $h$ is a Thalion Gul iff $G$ is abelian, it is sufficient
to prove both:
\begin{enumerate}
\item If $h$ is a Thalion Gul, then $G$ is abelian.
\item If $G$ is abelian, then $h$ is a Thalion Gul.
\end{enumerate}

Let $a,b \in G$, where $G$ is a group under $*$.

\textbf{Proof of 1}\\
Let $h$ be a Thalion Gul where $h(x) = x^{-1}$.

From the definition of Thalion Gul, we know $\forall a,b \in G$,
$h(a*b) = h(a) * h(b)$. Thus, it follows that:
\begin{align*}
            &h(a*b) = h(a) * h(b)        &\text{def of Thalion Gul}\\
\implies    &(a*b)^{-1} = a^{-1} * b^{-1}    &\text{(def of $h$)}\\
\implies    &(a*b)^{-1} = (b*a)^{-1}          &\text{(definition of inverse)}\\
\implies    &h[(a*b)^{-1}] = h[(b*a)^{-1}]        &\text{(apply $h$ to both sides)}\\
\implies    &[(a*b)^{-1}]^{-1} = [(b*a)^{-1}]^{-1}        &\text{(inverse of inverse of $u$ is $u$)}\\
\implies    &a*b = b(a                        &\text{(def of $h$)}\\
\end{align*}
Thus, we see that if $h$ is a Thalion Gul, then $G$ satisfies commutativity and $G$ is Abelian.

\textbf{Proof of 2}\\
Let $G$ be Abelian.

Therefore $a*b = b*a$.
\begin{align*}
 h(a*b) &= h(b*a)        &\text{($G$ is abelian)}\\
=(a*b)^{-1} &= (b*a)^{-1}    &\text{(def of $h$)}\\
=b^{-1} * a^{-1} &= a^{-1} * b^{-1}   &\text{(Theorem 4)}\\
=h(b) * h(a) &= h(a) * h(b) &\text{def of $h$}
\end{align*}
Since $h(a*b) = h(a) * h(b)$, $h$ is a Thalion Gul.
\end{question}
\begin{question}{Shelob's Lair}
Sym 5 is the group of transformations on the permutations of $[5]$.
We will define the group under the operation of composition ($*$).

Starting from an arbitrary permutation $p\in\text{ Sym 5}$, you can transform $p$
to be one of $5!$ possible permutations. Therefore the order of Sym 5 is $5! = 120$.

G is the group of transformations on the labelings of the graph of Shelob's Lair.
We will define the group under the operation of composition ($*'$).

We label Shelob's graph under the following rules. Each node will have an unordered pair
of distinct numbers from the set of $[5]$. Additionally, for all nodes $n$, $n$ must not have 
any numbers in common with the neighbors of $n$. For example, if $n$ is labeled as $\{1,2\}$,
then it's neighbor's labelings must not have 1 or 2. 

We will show that G also has order 120. Say you have an arbitrary labeling $l$. For an arbitrary
node $n$ in the graph, choose 2 elements from $[5]$ numbers to represent the node. 
There are $\binom{5}{2}=10$ ways to do this.
Since $n$ has degree 3, there are 3 other nodes which must have labels, with numbers excluding the first 
two chosen. So there are $\binom{3}{2}=3$ different labelings to choose from, and $n$ has 3 neighbors.
Therefore, there are $3!$ ways to label the neighbors of $n$. For each of these nodes in the neighborset
of $n$, there are 2 neighbors which are not $n$ and only 2 labelings left. Then there are 2 ways to do
this step. It is clear to see that there is only 1 way to label the rest of the nodes. In total, that makes
$$10*3!*2=120$$
Woah!
$$|G|=|Sym 5|=120$$

Now we will try to show that there exists a function $f:Sym 5 \rightarrow G$ which is an isomorphism. That is to say there 
exists a \textbf{homomorphism} $f$ which is bijective. To be a homomorphism $f$ must satisfy the property that
$f(a*b)=f(a)*'f(b)$ where $a$ and $b$ are elements of Sym 5. Specifically, $f$ is a function which maps a permutation
to a labeling of schelob's graph.

Say you have two permutations in Sym 5 called $a$ and $b$. Recall that $*$ is the composition operator for 
permutations and $*'$ is the composition operator for graph labelings. If you do $a*b$, you get a permutation
$c$ because groups satisfy closure. If you do $f(a)$, you get a graph labeling $a'$. If you do $f(b)$
you get a graph labeling $b'$. When you compose these two labelings, you get a labeling $c'$. Without a 
thorough proof, $f(c)=c'$.

\textbf{Bijectivity:}\\
Injectivity:\\
Say you have two different permutations. Each generates a different graph labeling. Therefore
two distinct permutations generate two distinct graph labelings.

This fact, combined with the fact that $|Sym 5| = |G|$ leads to the conclusion 
that the homomorphism is a bijection. Therefore, there exists an isomorphism
between Sym 5 and G.


\end{question}
\begin{question}{The Siege of Gondor}
\textbf{Part a.}\\
For a given $n\in\mathbb{N}$, consider the consecutive set $S$ of $n$ integers
$$S=\{(n+1)!+2, (n+1)!+3, \cdots, (n+1)!+(n+1)\}$$

By the definition of factorial,$\forall k\in\mathbb{N}, 1\leq k \leq n+1$
$$k|(n+1)!$$
By the definition of divides, for each $k$, $\exists c\geq 1$  such that:
$$(n+1)! = ck$$
Therefore,
$$(n+1)! + k = ck + k = k(c+1)$$
Therefore, $k|((n+1)! + k)$ by the definition of divides.
Therefore, every element in $S$ defined above is composite.

Therefore, $S$ is a set of $n$ composite, consecutive integers which can be 
generated for any given $n$. Therefore $S$ exists.

\textbf{Part b.}\\
First, we prove the following:
$x^b-1 = (x-1)(x^{b-1}+x^{b-2}+\cdots + 1)$

The proof is by algebra:
\begin{align*}
(x-1)( x^{ b-1}+x^{b-2}+ \cdots + 1) &= x(x^ {b-1}+x^{b-2} +\cdots+1)-(x^{b-1} + x^{b-2}+\cdots + 1)\\
&= (x^{b} + x^{b-1}+\cdots + x)-(x^{b-1} + x^{b-2} + \cdots+1)\\
&= x^b - 1
\end{align*}

Since $n$ is composite, we know $\exists a,b>1$ such that $n=ab$.
$$2^n - 1 = 2^{ab} - 1 =(2^a)^b - 1$$

Let $x = 2^a$, and then by the result proven above, 
$$(2^a)^b - 1 = (2^a-1)((2^a)^{b-1}+(2^a)^{b-2}+\cdots + 1)$$
The factor on the left is greater than 1, since $a>1$ and $2^a - 1 \geq 4-1 = 3 > 1$. The factor on the right is a sum of positive numbers, each greater than one. 

Therefore, if $n$ is composite, $2^n - 1$ can be factored into two integers greater than 1. By definition, it is composite. 

\textbf{Part c.}\\
\end{question}




\begin{question}{Mount Doom}
\textbf{Part a.}\\
The proof is by induction.

\textbf{Claim:}\\
$P(n)=``C_n \geq  2^n" \forall n \in \mathbb{N}$

\textbf{Base Case:}\\
$P(0) \iff C_0 \geq 2^0 \iff 1 \geq 1$\\
$P(1) \iff C_1 \geq 2^1 \iff \binom{2*1}{1} \geq 2 \iff 2 \geq 2$\\
The claim holds for the $n=0$ and $n=1$.

\textbf{Inductive Hypothesis:}\\
Assume $P(1) \wedge P(2) \wedge \cdots \wedge P(k)$ for some $k \in \mathbb{N},\mathbf{k\geq 1}$.

\textbf{Inductive Step:}\\
(To show $C_{k+1} \geq 2^{k+1}$)\\
\begin{align*}
       &C_{k+1} \geq 2^{k+1}           &\text{}\\
\iff    &C_{k+1} \geq 2^k \cdot 2       &\text{(Algebra)}\\
\iff    &C_{k+1} \geq 2C_{k}            &\text{(By the IH)}\\
\iff    &\frac{(2k+2)!}{(k+1)!(k+1)!} \geq \frac{2(2k)!}{k!k!} &\text{[Def of $C_n$]}\\
\iff    &\frac{(2k+2)!}{(k+1)!(k+1)!} \geq \frac{2(2k)!}{k!k!} \cdot \frac{(k+1)(k+1)}{(k+1)(k+1)}      &\text{(Multiply by 1)}\\
\iff    &\frac{(2k+2)!}{(k+1)!(k+1)!} \geq \frac{2(k+1)^2(2k)!}{(k+1)!(k+1)!} &\text{(Algebra)}\\
\iff    &(2k+2)! \geq 2(k+1)^2(2k)!     &\text{(Algebra)}\\
        &(2k+2)(2k+1) \geq 2(k+1)^2       &\text{(Divide by $(2k)!$)}\\
\iff    &4k^2 + 6k + 2 \geq 2k^2 + 4k + 2   &\text{(Expand)}\\
\iff    &2k^2 + 2k \geq 0                   &\text{(Simplify)}\\
\iff    &true                           &\text{(k is positive)}\\
\end{align*}
Note that since $k\geq1$, all division above is legal.
Since all the steps are reversible, we have proved that $P(k) \implies P(k+1)$.

By mathematical induction, the claim holds.
\end{question}
\end{document}
