\documentclass[11pt]{article}
\usepackage{enumerate}
\usepackage{fullpage}
\usepackage{fancyhdr}
\usepackage{amsmath, amsfonts, amsthm, amssymb}
\setlength{\parindent}{0pt}
\setlength{\parskip}{5pt plus 1pt}
\pagestyle{empty}

\def\indented#1{\list{}{}\item[]}
\let\indented=\endlist

\newcounter{questionCounter}
\newcounter{partCounter}[questionCounter]
\newenvironment{question}[2][\arabic{questionCounter}]{%
    \setcounter{partCounter}{0}%
    \vspace{.25in} \hrule \vspace{0.5em}%
        \noindent{\bf #2}%
    \vspace{0.8em} \hrule \vspace{.10in}%
    \addtocounter{questionCounter}{1}%
}{}
\renewenvironment{part}[1][\alph{partCounter}]{%
    \addtocounter{partCounter}{1}%
    \vspace{.10in}%
    \begin{indented}%
       {\bf (#1)} %
}{\end{indented}}

%%%%%%%%%%%%%%%%% Identifying Information %%%%%%%%%%%%%%%%%
%% This is here, so that you can make your homework look %%
%% pretty when you compile it.                           %%
%%     DO NOT PUT YOUR NAME ANYWHERE ELSE!!!!            %%
%%%%%%%%%%%%%%%%%%%%%%%%%%%%%%%%%%%%%%%%%%%%%%%%%%%%%%%%%%%
\newcommand{\myname}{Karan Sikka}
\newcommand{\myandrew}{ksikka@andrew.cmu.edu}
\newcommand{\myhwname}{Assignment 4}
\newcommand{\myrecitation}{E}
%%%%%%%%%%%%%%%%%%%%%%%%%%%%%%%%%%%%%%%%%%%%%%%%%%%%%%%%%%%

\begin{document}
\thispagestyle{plain}

\begin{center}
{\Large \myhwname} \\
\myname \\
\myandrew \\
\myrecitation \\
\today
\end{center}

\begin{question}{Apocalypse Averted}
\textbf{Part a:} Recall from a lecture that $ (1 + x)^{n} = \sum_{k=0}^{n}{\binom{n}{k}x^i}$.

Therefore we can represent $n^{k+1}$ as $(1 + (n-1))^{k+1}$, which can be expanded 
using the binomial theorem to $$n^{k+1}=\sum_{i=0}^{k+1}{\binom{k+1}{i}(n-1)^i}$$

Consider $2^{k+1} + 3^{k+1} + ... + (n)^{k+1} + (n+1)^{k+1}$.\\
But please consider it on the next page.

\begin{align*}
2^{k+1} + ... + (n+1)^{k+1} &= \sum_{i=0}^{k+1}{\binom{k+1}{i}(2-1)^{i}} + ... + \sum_{i=0}^{k+1}{\binom{k+1}{i}(n+1-1)^{i}}\\
&\mbox{then by algebra,}\\
2^{k+1} + ... + (n+1)^{k+1} &= \sum_{i=0}^{k+1}{\binom{k+1}{i}(1)^{i}} + ... + \sum_{i=0}^{k+1}{\binom{k+1}{i}(n)^{i}}\\
&\mbox{then by factoring out the binomial coefficients,}\\
2^{k+1} + ... + (n+1)^{k+1} &= \binom{k+1}{0}(1^0 + ... + n^0) + ... + \binom{k+1}{k+1}(1^{k+1} + ... + n^{k+1})\\
&\mbox{then by the definition of $S_k(n)$,}\\
2^{k+1} + ... + (n+1)^{k+1} &= \binom{k+1}{0}S_0(n) + ... + \binom{k+1}{k+1}S_{k+1}(n)\\
&\mbox{then by using Sigma notation for the sum,}\\
2^{k+1} + ... + (n+1)^{k+1} &= \sum_{i=0}^{k+1}\binom{k+1}{i}S_i(n)\\
&\mbox{then by adding 1 to both sides}\\
1^{k+1} + 2^{k+1} + ... + (n+1)^{k+1} &= 1 + \sum_{i=0}^{k+1}\binom{k+1}{i}S_i(n)\\
&\mbox{then by the definition of $S_{k+1}(n)$ }\\
S_{k+1}(n) + (n+1)^{k+1} &= 1 + \sum_{i=0}^{k+1}\binom{k+1}{i}S_i(n)\\
&\mbox{then by subtracting $S_{k+1}(n)$ from both sides }\\
(n+1)^{k+1} &= 1 + \sum_{i=0}^{k}\binom{k+1}{i}S_i(n)\\
&\mbox{then by subtracting $1$ from both sides }\\
(n+1)^{k+1} - 1 &= \sum_{i=0}^{k}\binom{k+1}{i}S_i(n)\\
\end{align*}

\textbf{Part b:}

\textbf{Claim:}\\
For all $n \in \mathbb{N}$, $S_{n}(x)$ is a polynomial with degree $n+1$.

\textbf{Base Case:}\\
When n = 0, $S_{0}(x) = 1^0 + ... + x^0$\\
This is a sum of $x$ 1's. Therefore the $S_{0}(x)=x$. Therefore the claim holds 
for the base case.

\textbf{Inductive Hypothesis:}\\
Assume the claim holds for all natural numbers $j$ upto some natural number $k$.

\textbf{Inductive Step:}\\
We know from Part A that $$\sum_{i=0}^{k}\binom{k+1}{i}S_i(n) = (n+1)^{k+1} - 1$$
Then, the degree of the LHS is $k+1$ by the IH, and also by the RHS. 
If we add another term to the LHS, namely $S_{k+1}$, we get the following 
result:
$$\sum_{i=0}^{k+1}\binom{k+2}{i}S_i(n) = (n+1)^{k+2} - 1$$
The degree of this polynomial is the degree of highest degreed polynomial being 
summed on the LHS. From the RHS, we know that this degree is $k+2$.

Since the addition of $S_{k+1}$ changed the degree of the LHS from $k+1$ to $k+2$, 
the degree of $S_{k+1}$ must be $k+2$. 

Therefore, we have proved by 
course of values induction that the claim holds for all $n \in \mathbb{N}$.

\end{question}

\begin{question}{In Lament of Locks}
Consider a lock with n dials that can be either 0, 1, or 2. 
Let $n$ be an integer such that $n\geq1$.

Since there are $n$ dials, and each dial has three possible 
values (0, 1, or 2), there are $3^{n}$ different combinations. We can 
partition the set of all the combinations into two sets: the set of 
combinations with an even number of zeros and the set of combinations with an 
odd number of zeros. 

\textbf{Notation:}
\begin{enumerate}
\item An ``O-combination'' has an odd number of zeros.
\item An ``E-combination'' has an even number of zeros.
\end{enumerate}

\textbf{Fun facts:}\\
$3^n$ is an odd number because 3 is an odd number, and an odd number 
multiplied an odd number is odd.

Since $3^n$ is an odd number, $3^n = 2k+1$ for some integer k.

\textbf{Claim:}\\
For all n in the natural numbers, given a lock with n dials and 
$3^n=2k+1$ total possible combinations, there are $k$ O-combinations and $k+1$ 
E-combinations, for some k in the natural numbers.

We proceed by mathematical induction on $n$. 

\textbf{Base Case:}\\
When there is one dial, the possible combinations are 0, 1, or 2. 

The combination (0) has an 1 zero: an odd number of zeroes. Therefore there is 
1 O-combination.

The combinations (1) and (2) have 0 zeroes: an even number of zeroes. Therefore 
there are 2 E-combinations. 

Therefore the claim holds, because there are $k$ O-combinations and $k+1$ 
E-combinations where k is 1 in this case.

\textbf{Inductive Hypothesis:}\\
For some $z$ in the natural numbers, given a lock with z dials and 
$3^z=2k+1$ total possible combinations, there are $k$ O-combinations and $k+1$ 
E-combinations.

\textbf{Want to show:}\\
Given a lock with $z+1$ dials and $3^{z+1}=2j+1$ total possible combinations, there 
are $j$ O-combinations and $j+1$ E-combinations.

\textbf{Inductive Step:}\\
Consider a lock with $z+1$ dials.

Its combinations can be formed by taking all the combinations of a z-dial lock,
and adding an additional dial with a value of 0, 1, or 2.

To form an E-combination for the lock with z+1 dials...:\\
Method 1- add a 0 onto an O-combination. There is 1 way to do this. Therefore, by 
the IH, this method forms $k$ combinations.\\
Method 2- add a 1 or 2 onto an E-combination because then the number of zeroes 
does not change and it remains an E-combination. There are 2 ways to do this for each
combination. Therefore, by the IH, this method forms $2(k+1)$ combinations.
The total number of E-combinations is $2(k+1)+k$, which simplifies to $3k+2$.

To form an O-combination for the lock with $z+1$ dials...:\\
Method 1- add a 0 onto an E-combination, because then it has an odd number of 
zeroes. There is 1 way to do this. Therefore, by 
the IH, this method forms $k+1$ combinations.\\
Method 2- add a 1 or 2 onto an O-combination, because then the number of zeroes 
does not change and it remains an O-combination. There are 2 ways to do this for
each combination. Therefore, by the IH, this method forms $2(k)$ combinations.
The total number of E-combinations is $2(k)+(k+1)$, which simplifies to $3k+1$.

Let $j = 3k+1$. There are $j$ O-combinations and $j+1$ E-combinations. The inductive 
step holds.

\textbf{A Formula:}\\
Now, we can derive a formula for the number of E-combinations in terms of the 
number of dials.\\
Let the number of E-combinations be k+1.
\begin{align*}
k+1 &= \frac{2k}{2} + 1 & \mbox{by algebra}\\
&= \lfloor \frac{2k}{2} + \frac{1}{2} \rfloor + 1 & \mbox{by properties of floor}\\
&= \lfloor \frac{2k+1}{2} \rfloor + 1 & \mbox{by algebra}\\
&= \lfloor \frac{3^{n}}{2} \rfloor + 1 & 2k+1 = 3^n = \mbox{num. of total combinations}\\
\end{align*}

If there are 15251 dials, it follows that there are $\lfloor \frac{3^{15251}}{2} \rfloor + 1$ 
combinations with an even number of zeros.
\end{question}



\begin{question}{Mirror, Mirror On The Wall}
A path is essentially a sequence of steps, where a step is an element in the set 
\{LEFT, RIGHT, UP, DOWN\}.

There are $2n$ steps. Choose $n$ of them to be $a$. Then theres only 1 way to make
the rest of them $b$.

Then, of the $2n$ steps, choose $n$ of them to be $z'$ where $z$ is the letter. 

To clarify, the following is an example of the above process for n=4:

\begin{align}
aabbabaa && \binom{2n}{n} \mbox{ways to make such a sequence}\\
a'ab'b'ab'aa && \binom{2n}{n} \mbox{ways to assign primes to $n$ variables}
\end{align}

\begin{align*}
\mbox{Let } & a' = \mbox{UP}\\
            & b  = \mbox{DOWN}\\
            & b'  = \mbox{LEFT}\\
            & a  = \mbox{RIGHT}
\end{align*}

\textbf{Claim:}\\
Using this method of generating paths, there are the same number of UPs as DOWNs
and the same number of LEFTs as RIGHTs.

\textbf{Proof:}\\
In the first step in the two step process, you choose $n$ $a$'s, so implicitly 
there must be $2n-n=n$ $b$'s. Therefore there are as many $a$'s as $b$'s. (not 
sure if that was essential to the proof, but it is included just in case.)

In the second step, for every $a$ you prime, you implicity choose not to prime a $b$. 
This is an exact one-to-one correspondance, so the number of $a'$'s is equal
to the number of $b$'s. Using the translation above, the number of UPs equals 
the number of DOWNs.

Similarly, for every $b$ you prime, you implicity choose not to prime an $a$. 
Therefore there are an equal number of RIGHTs and LEFTs.

Say you prime $k$ $a$'s and $j$ $b$'s, where $k + j = n$. Then you choose not to prime 
k b's and j a's. There are k UPs, k DOWNs, j LEFTs, and j RIGHTs. Altogether
this is $2n$ steps. Since the equal number of UPs and DOWNs eliminate each 
other's vertical progress, and likewise for the horizontal direction, the path
ends at the same point it started at.

Therefore, we have shown a method of counting all the length $2n$ paths which
start and end at the same place, where each step in the path involves moving in
one of the 4 cardinal directions. By the rule of product, there are 
$$ \binom{2n}{n}\binom{2n}{n} $$
distinct paths.

\end{question}

\begin{question}{Macaroni Revisited}
\textbf{Notation:}\\
\begin{itemize}
\item 1 represents a macaroni
\item 0 represents a bead
\end{itemize}
In this problem, there are $n+1$ macaronis and $n$ beads for each necklace.

Instead of a necklace of macaroni and beads, imagine a circle of 1's and 0's. 
(1 = macaroni, 0 = beads)

\textbf{Generate a characteristic sequence for every necklace:}\\

We describe a method to produce a characteristic sequence of 1's and 0's for 
each macaroni necklace:
\begin{enumerate}
\item Find the longest continuous sequence of 1's.
\item Cut the necklace right before the first 1 of that sequence. The beginning
of the characteristic is the portion that starts with the longest continuous number
of 1's. Note that the last number must be a 0.
\item Omit the first 1.
\end{enumerate}
\emph{Note: Step 1 may not be deterministic. I explain at the end of the proof how 
to fix this. For now, let's assume without proof that this method generates a 
unique sequence of numbers for a given macaroni necklace.}

Let k be the longest continuous sequence of 1's before omiting the first 1.
\textbf{Claim:} $k > 1$

\textbf{Proof:}\\
Assume for the sake of contradiction that the longest continuous sequence of macaronis
is 1 length. That means that each macaroni is separated by at least 1 bead.
That implies that the number of macaronies is less than or equal to the number of 
beads. This is a contradiction, because there are $n+1$ macaronies and $n$ beads.
Therefore there must exist a continuous sequence of macaronis of length 2.

Since we now know that $k>1$, when we omit the first 1, there must be another 1
to start the sequence. In other words, after omitting a 1, the sequence must start 
with 1 since there are at least 2 1's in the beginning of the sequence.

\textbf{Claim:} As you're reading a macaroni sequence from left to right,
 the number of 1's is always greater than or equal to the number of 0's.
 
\textbf{Proof:} A necklace has $n+1$ macaronis and $n$ beads. After omitting a 1
to form its macaroni sequence, the sequence has $n+1 -1=n$ 1's and $n$ 0's.
This is analogous to the proof showed in lecture that when the car is going around
in circles, it will never run out of gas. Why is it analogous? We are starting 
at the largest "gas station", which in this problem, is the longest sequence of 
macaronies. Then we go around, and as we pass each bead, our gas depletes. However,
in the end, we'll always have enough gas to go one pass without running out.

We can say that this method represents a way to transform an element of the set 
of all macaroni necklaces to an element in the set of number sequences. This 
transformation is a function, because 1 necklaces corresponds to only 1 number 
sequence (we make this assumption according to the note).

\textbf{Claim:}The function, as described above, is bijective.

\textbf{Proof:}Every necklace maps to a distinct sequence of numbers. Every 
sequence of numbers can construct a necklace by reversing the steps above. 
First prepend a 1 to the number sequences Then connect the beginning and end in a 
circle, and replace the 1's with macaroni and the 0 with beads.

\textbf{Method for producing subsets of $\mathbb{N}$}\\
We define $S_n$ as the set of all subsets of $\mathbb{N}$ where the following 
conditions hold:

The set contains 0
Let i be an arbitrary element in the set. 
Then the set also contains i+n and i+n+1.

Furthermore, we derive the following facts:
Since the set contains i, it contains i+n. Then must also contain (1+n)+n. Inductively, it must 
contain i + kn for all $k \in \mathbb{N}$
By the same logic for i +n+1, the set must contain i + kn + 1.

\textbf{Method for creating lattice paths:}
For each $S_n$, we can create a grid with $n$ columns, and infinite rows.
Fill the grid with the Natural numbers using the following method.
Say $n=4$, and start at 1 ...
\begin{verbatim}
                                _____ ____ ____ _____
                                |    |    |    |    |
                                 ---- ---- ---- ----
                                |    |    |    | 4  |
                                 ---- ---- ---- ----
                                |    |    | 3  | 8  |
                                 ---- ---- ---- ----
                                |    | 2  | 7  | 12 |
                                 ---- ---- ---- ----
                                | 1  | 6  | 11 | 16 |
                                 ---- ---- ---- ----
                                | 5  | 10 | 15 | 20 |
\end{verbatim}
... and this pattern continues for the rest of the rows. 

Note that as you go down the column, the pattern is i, i + n, i + 2n, ... This 
is no coincedence, it is derived from the way we filled the grid and the fact that
there are 4 columns. Also, note that as you go accross a row, the pattern is i, 
i+n+1, i+2(n+1), i+3(n+1)... This is because the element to the right of an element
is 1 greater than the element below it, by the way we filled the grid.

Now, to create a ``Lattice path", start from the top-left corner of the cell
in which there is a 1. Then, draw a line either right or up, until you reach the
top-right most point of the grid, or if the move will not be adjacent to a number
Note: the first move
will be to the right, since a move up on the first move will not be adjacent to a number.
Note: the
last move will be up, since moving to the right to the top-right-most point is impossible
without making a step that is not adjacent to a number.

\textbf{Bijection from lattice paths to number sequences}
Each lattice path represents a subset of $S_n$. Here's how. First create a lattice
path. Then all the numbers below the lattice path and included in the set, and all
those above the lattice path are excluded. Why does this work? Well 0 is in the set,
which explains why we always start the lattice path with a move to the right.
Every time you move up, you include all the numbers to the right of that number. 
That makes sense, because as we previously determined, if i is in the set, then 
i + n + 1 is in the set, so i + k(n + 1) is in the set. Every time you move down, 
you include all the numbers below that number. This makes sense because if i is 
in the set, then i + n is in the set, so i + kn is in the set. In conclusion,
there is an injective mapping from lattice paths to number sets. There is also an injective mapping in the other direction. If you have two different
subsets of $S_n$, they cannot be represented by the same lattice path. Therefore this mapping 
is bijective.

\textbf{Bijection from macaroni number sequences to lattice paths}
You can go from macaroni to lattice paths using the following method:

Imagine that you are reading a macaroni necklace's characteristic sequence from left
to right. As you read it, the numbers signify as follows:\\
1. 1 means move right\\
2. 0 means move up\\

We proved way back that as
you're reading a macaroni sequence from left to right, the number of 1's is always
greater than or equal to the number of 0's. This means the move you make will always
be adjacent to some number in the grid. Also, the macaroni sequence starts with 1 and 
ends with 0. Therefore the first move is right and the last move is up.  Also 
the method is injective, since each macaroni necklace 
clearly maps to a different lattice path. To go from lattice paths to macaroni
sequences, you analyze the lattice path from left to right and replace UP with 0 and 
RIGHT with 1. This is also clearly injective by definition. Since the forward process
and the reverse process are injective, this method is bijective.

The bijection from macaroni necklaces of length $n$ to subsets of $S_n$ 
is from macaroni, to 1/0 sequences, to lattice paths, to subsets.

Disclaimer: In the method I described for generating number sequences from 
macaroni necklaces, I assumed that the one necklace generates a unique num.
sequence. In reality, this is not the case.

Claim: It is possible to generate exactly one number sequence from a distinct macaroni
necklace such that 
\begin{enumerate}
\item it begins with the longest sequence of 1's
\item as you go from left to right, the number of 1's $\geq$ the number of 0's
\end{enumerate}

I just haven't found the method for this yet, nor the proof that it is possible.
However the rest of the proof is correct.
\end{question}
\begin{question}{Your Wandering Mind}
Consider an $m$ x $n$ grid of balls, where the columns are sorted in ascending
order. Pick two arbitrary balls in an arbitrary row. Let the left-most ball be
$a$ and the right-most ball be $b$. 

Pick arbitrary $e,f,c,d$ as follows:
\begin{verbatim}
| e |     | f |
  .         .
  .         .
  .         .
| a | ... | b |
  .         .
  .         .
  .         .
| c |     | d |
\end{verbatim}
(This doesn't work for edge cases like the first or last row, but that will
be addressed at the end of the proof.)

Since the columns are sorted, we can establish the following inequalities:
\begin{align*}
&e \leq a \leq c&\\
&f \leq b \leq d&
\end{align*}

There are two cases: $a$ is smaller than $b$, or $a$ is larger than $b$.

If $a \leq b$, the rows and columns are sorted. Done.

If $a > b$, $a$ and $b$ can be swapped to fix this. Let $b$ be smaller than $a$,
 so we swap $b$ and $a$. 

\begin{verbatim}
| e |     | f |
  .         .
  .         .
  .         .
| b | ... | a |
  .         .
  .         .
  .         .
| c |     | d |
\end{verbatim}

\textbf{Assume for sake of contradiction} that by swapping
the balls, the columns are no longer sorted. This implies that either $b < e$ or
$c < b$ (but not both because then $c < e$ and this is false since the columns
were sorted to begin with).

\textbf{Case $b < e$:}\\
Recall from the originally established inequalities that $e < a$ and 
$f < b$. Also $b < a$, which is why we swapped $b$ and $a$. From these, we can 
construct the following inequality:
$$ f < b < e < a $$
This implies that $f < e$, which means that $e$ and $f$ need to be swapped.
Once they are, we can rewrite the inequality as:
$$ e < f < b < a $$
With that correction, we maintain that the columns are sorted because 
$e < b$ and $f < a$. Contradiction.

\textbf{Case $c < b$:}\\
Recall from the originally established inequalities that $a < c$
We know $c < b$ and $b < a$. This implies $c < a$, but this contradicts the 
first sentence of this case. Therefore this case never occurs. The columns
must be sorted.

\textbf{Right-most $a$:}\\
We have proved by contradiction, that for an arbitrary $a$ and $b$, if they are 
out of order and you swap them, the original column of $a$ remains sorted. Since
we did this for an arbitrary $a$, the principle extends to all positions in the 
row except the right-most edge of the row, because if the right-most edge of the
row were $a$, there is no position $b$ to the right of it.

However, if the principle is true for all elements in the row but the last, 
it is \textbf{forced to be true} for the last position. This is because the right-most 
two elements must be sorted (when m-1 is $a$, m must be $b$, and they are swapped 
to be in the right order), AND the left-most m-2 elements must be sorted (the
principle is true for all elements but the last).

\textbf{Top and bottom rows:}\\
What about the top row and the bottom row? With the top row, we can apply the 
same reasoning as for any other row, but we ignore the case where $b < e$ because
you can't choose an $e$ for a $b$ in the top row. Likewise for the bottom row.

Therefore, by a non-constructive proof, sorting the rows maintains that the 
columns are sorted.

\end{question}
\begin{question}{Last Ditch Attempt}
\textbf{Overview: }
\begin{enumerate}
\item Partition the set of all tilings
\item Show the cardinality of each partition in the form of a recurrance
\item Simplify the recurrance
\item Prove a closed form of the recurrance in terms of the Fibonacci sequence.
\end{enumerate}
\textbf{Partition the set of all tilings}\\
All of the tilings can be partitioned based on the left-most block(s). All 
tilings must fall into a partition, and no tiling can be in more than one 
partition. We call $f_k$ the number of tilings of a $2xk$ board.

Partition 1: A red 2x1 block as the left-most block. There's 1 way to put the red
block on the left, and $f_{k-1}$ ways to tile the remaining 2xk-1 board.

Partition 2: Two red 1x2 blocks on top of each other as the left-most blocks.
There's one way to put the initial blocks on the left, and $f_{k-2}$ ways to tile
the remaining 2xk-2 board.

Partition 3: A 2xi block, for all integers $i$ such that $2 \leq i \leq k$ as the left-most block. There are 
2 ways to pick a color for the left-most 2xi block, and $\sum_{i=2}^{k}{f_{k-i}}$
ways to tile the remaining right 2xi board. The total number of ways to tile this
type of board is the product of 2 and the sum. The sum is there to account for the range 
of possibilities for $i$.

\textbf{Recurrance for $f_k$}\\
Therefore, the cardinality of the set of all tilings is the sum of the 
cardinalities of the partitions (by rule of sum):
$$ f_k = f_{k-1} + f_{k-2} + 2 \sum_{i=2}^{k}{f_{k-i}} $$

\textbf{Simplifying the recurrance}\\
\begin{align*}
f_k &= f_{k-1} + f_{k-2} + 2 \sum_{i=2}^{k}{f_{k-i}} && \mbox{the recurrance}\\
    &= f_{k-1} + f_{k-2} + 2f_{k-2} + 2 \sum_{i=3}^{k}{f_{k-i}} && \mbox{extract a term from the sum}\\
    &= 2f_{k-1} - f_{k-1} + 2f_{k-2} + f_{k-2} + 2 \sum_{i=3}^{k}{f_{k-i}} && \mbox{manipulate using algebra}\\
    &= 2f_{k-1} + 2f_{k-2} + f_{k-2} + 2 \sum_{i=3}^{k}{f_{k-i}} - (f_{k-2} + f_{k-3} + 2 \sum_{i=2}^{k-1}{f_{k-1-i}}) && \mbox{replace $f_{k-1}$ with it's recurrance}\\
    &= 2f_{k-1} + 2f_{k-2} + 2 \sum_{i=3}^{k}{f_{k-i}} - f_{k-3} - 2 \sum_{i=2}^{k-1}{f_{k-1-i}} && \mbox{simplify using algebra}\\
    &= 2f_{k-1} + 2f_{k-2} + 2 \sum_{i=3}^{k}{f_{k-i}} - f_{k-3} - 2 \sum_{i=3}^{k}{f_{k-i}} && \mbox{change indices of summation}\\
    &= 2f_{k-1} + 2f_{k-2} - f_{k-3} && \mbox{simplify using algebra}
\end{align*}

Therefore: $$f_k = 2f_{k-1} + 2f_{k-2} - f_{k-3}$$

\textbf{Prove a closed form of the recurrance in terms of the Fibonacci sequence.}\\
$F_n$ is the $n^{th}$ term in the fibonnaci sequence.\\
$F_0 = 1$\\
$F_1 = 1$\\
$F_n = F_{n-1} + F_{n-2}$\\
\textbf{Claim:}\\
For all $n \in \mathbb{N}$, $f_n = F_{n}^2$.

\textbf{Base Case:}\\
$f_0 = 1$ since there is one way to tile a 2x0 board, and that is to place 0 
tiles on the board.\\
$f_1 = 1$ since there is one way to tile a 2x1 board, and that is to place a 2x1
red tile on the board.\\
$f_2 = f_1 + f_0 + 2 \sum_{i=2}^{2}{f_{2-i}} = 1 + 1 + 2f_0 = 4$\\

$F_{0}^2 = 1^2 = 1$\\
$F_{1}^2 = 1^2 = 1$\\
$F_{2}^2 = (F_{1} + F_{2})^2 = (1+1)^2 = 4$\\

Therefore the claim holds for $n = 1,2,3$.

\textbf{Inductive Hypothesis:}\\
Assume for all $j \in \mathbb{N}$ upto some $k \in \mathbb{N}, k \geq 3$, $f_j = F_{j}^2$.

\textbf{Inductive Step:}\\
Consider $F_{k+1}$
\begin{align*}
\quad F_{k+1} &= F_{k} + F_{k-1}        &\text{def of $F_{n}$}\\
\implies    \quad F_{k+1} - F_{k} &= F_{k-1}        &\text{algebra}\\
\implies    \quad (F_{k+1} - F_{k})^{2} &= (F_{k-1})^{2}    &\text{square both sides}\\
\implies    \quad (F_{k+1})^{2} - 2F_{k}F_{k+1} + (F_{k})^{2} &=
(F_{k-1})^{2}        &\text{expand}\\
\implies    \quad 2(F_{k+1})^{2} + 2(F_{k})^{2} - (F_{k-1})^{2} &=
(F_{k+1})^{2} + 2F_{k}F_{k+1} + (F_{k})^{2}  &\text{algebra}\\
\implies    \quad 2(F_{k+1})^{2} + 2(F_{k})^{2} - (F_{k-1})^{2} &= (F_{k}
+ F_{k+1})^{2}            &\text{factor}\\
\implies    \quad 2f_{k} + 2f_{k-1} - f_{k-2} &= (F_{k} + F_{k+1})^{2}
&\text{by the IH}\\
\implies    \quad 2f_{k} + 2f_{k-1} - f_{k-2} &= (F_{k+2})^{2}
&\text{def of $F_{n}$}\\
\implies    \quad f_{k+1} &= (F_{k+2})^{2}           &\text{def of $f_{n}$}
\end{align*}

Therefore, by mathematical induction, the claim holds for all $n \in \mathbb{N}$.

\end{question}



\begin{question}{Extra Credit}
\end{question}
\end{document}
