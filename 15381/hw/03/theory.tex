\documentclass[11pt,letterpaper]{article}

\usepackage{amsmath}
\usepackage{amssymb}
\usepackage{fancyhdr}

\oddsidemargin0cm
\topmargin-2cm
\textwidth16.5cm
\textheight23.5cm

\newcommand{\question}[2] {\vspace{.25in} \hrule\vspace{0.5em}
\noindent{\bf #1: #2} \vspace{0.5em}
\hrule \vspace{.10in}}
\renewcommand{\part}[1] {\vspace{.10in} {\bf (#1)}}

\newcommand{\myname}{Karan Sikka}
\newcommand{\myandrew}{ksikka@cmu.edu}
\newcommand{\myhwnum}{03}

\setlength{\parindent}{0pt}
\setlength{\parskip}{5pt plus 1pt}

\pagestyle{fancyplain}
\lhead{\fancyplain{}{\textbf{HW\myhwnum}}}
\rhead{\fancyplain{}{\myname\\ \myandrew}}
\chead{\fancyplain{}{15-381}}

\begin{document}

\medskip

\thispagestyle{plain}
\begin{center}                  % Center the following lines
{\Large 15-381 Assignment \myhwnum} \\
\myname \\
\myandrew \\
\today
\end{center}


\question{1}{Task 1.1}
$S, B, P$

\question{2}{Task 1.2}
$|X| = 3$, so there are $3!$ different orderings, which is $9$.

\question{3}{Task 1.3}
\begin{table}[!htbp]
    \begin{tabular}{l|l|l|l|l}
    W & R & I & K & Probablity \\ \hline
    W & W & W & W & 0.4608     \\
    W & W & W & S & 0.1152     \\
    W & W & S & W & 0.1152     \\
    W & W & S & S & 0.0288     \\ \hline
    W & S & W & W & 0.0162     \\
    W & S & W & S & 0.0378     \\
    W & S & S & W & 0.0378     \\
    W & S & S & S & 0.0882     \\ \hline
    S & W & W & W & 0.0192     \\
    S & W & W & S & 0.0048     \\
    S & W & S & W & 0.0048     \\
    S & W & S & S & 0.0012     \\ \hline
    S & S & W & W & 0.0063     \\
    S & S & W & S & 0.0147     \\
    S & S & S & W & 0.0147     \\
    S & S & S & S & 0.0343     \\
    \end{tabular}
\end{table}

\question{4}{Task 1.4}
$Pr(K=w) = 0.675$
$Pr(K=s) = 0.325$

\question{5}{Task 1.5}
$3*8=24$ multiplications, and $7$ summations.

\question{6}{Task 1.6}
$Pr(K) = \sum_{R,W,I} Pr(K,R,W,I)$

$Pr(K=w) = \sum_{r \in R} Pr(K=w | R=r)[\sum_{w \in W}Pr(W=w)Pr(R=r|W=w)]$
$      = 0.3(0.9*0.2 + 0.7*0.1) + 0.8(0.9*0.8 + 0.7*0.1)$
$      = 0.675$

and from this result, we derive

$Pr(K=s) = 1 - Pr(K=w) = 1-0.675 = 0.325$


\question{7}{Task 1.7}
4 multiplications and 3 additions.

Variable elimination results in fewer multiplications and divisions.

\question{8}{Task 1.8}
We would have increased the number of calculations since the summing wouldn't have been as efficient.

\question{9}{Task 2.1}
$z = f(x,y) = -(x-3)^3 - (y-17)^2 + 200$\\
$\nabla f(x,y) = g(x,y) = \langle -2(x-3), -2(y-17) \rangle $\\
$       = (6-2x, 34-2y)$

\question{10}{Task 2.2}
$0.4*g(x,y) = 0.4*g(-2, 10) = 0.4 \langle 10,14 \rangle = \langle 4, 5.6 \rangle$\\
$\Delta x= 4, \Delta y = 5.6$\\
Therefore, $x=2, y=15.6$

Repeating the above procedure with the results as the inputs, we get,\\
$x = 2.8, y=16.72$

And once more,\\
$x = 2.98, y=16.944$

The end resulting locations are\\
$$\langle (-2, 10), (2, 15.6), (2.8, 16.72), (2.96, 16.944) \rangle$$


\question{11}{Task 2.3}
With this large step size, the hobbit goes in a cycle $\langle (-2, 10),(8,24),(-2,10) \rangle$

If the step size were smaller, the hobbit would take many more steps to reach the same position.
But if the step size is too small, they will meet the stopping condition too quickly, and stop at
less optimal position.

\question{12}{Task 2.4}
To get the max x and y, we solve the following eqns:

$6-2x = 0$ and $34-2y = 0$

The maximum is at $(3,17)$

Here, the gradients are exactly 0, whereas the other points are not exactly at 0 and could be bettered by moving further.




\question{13}{Task 3.1}

\begin{table}[!htbp]
    \begin{tabular}{lll}
    ~   & $B=0$     & B=1     \\
    $A=0$ & $.5(\epsilon)$   & $.5(1-\epsilon)$ \\
    $A=1$ & $.5(1-\epsilon)$ & $.5(\epsilon)$   \\
    \end{tabular}
\end{table}

\question{14}{Task 3.2}
In the code

\question{15}{Task 3.3}
The runtime increases. As epsilon gets smaller, it takes longer to get from a random sample
to the true sample, because the true sample is farther away from an average random sample.

Had you direct sampled, it would have taken much much longer. But because Gibbs is an MCMC algorithm,
we don't throw away the samples - we merely improve them. So even though the runtime increases when you
reduce epsilon, it's not as bad as it would be with direct sampling.

\question{16}{Task 3.4}
a. \\
A 0.181\\
B 0.5002\\
C 0.4551\\
D 0.5532\\
E 0.5002\\
F 0.4996\\
G 0.5025\\


b. 0.5532\\
c. .0507\\

\end{document}

