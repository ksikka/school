\documentclass[11pt,letterpaper]{article}

\usepackage{amsmath}
\usepackage{amssymb}
\usepackage{fancyhdr}

\oddsidemargin0cm
\topmargin-2cm
\textwidth16.5cm
\textheight23.5cm

\newcommand{\question}[1] {\vspace{.25in} \hrule\vspace{0.5em}
\noindent{\bf #1} \vspace{0.5em}
\hrule \vspace{.10in}}
\renewcommand{\part}[1] {\vspace{.10in} {\bf (#1)}}

\newcommand{\myname}{Karan Sikka}
\newcommand{\myandrew}{ksikka@cmu.edu}
\newcommand{\myhwnum}{06}

\setlength{\parindent}{0pt}
\setlength{\parskip}{5pt plus 1pt}

\pagestyle{fancyplain}
\lhead{\fancyplain{}{\textbf{HW\myhwnum}}}
\rhead{\fancyplain{}{\myname\\ \myandrew}}
\chead{\fancyplain{}{15-381}}

\begin{document}

\medskip

\thispagestyle{plain}
\begin{center}                  % Center the following lines
{\Large 21-301 Assignment \myhwnum} \\
\myname \\
\myandrew \\
\today
\end{center}

\question{1.1}
Yes. Steal, Steal. If player 1 steals, player 2's optimum decision is to steal, and vice versa..

\question{1.2}
Yes. Steal, steal. We know the players both steal at the last iteration since there is no social punishment for doing so.
Given that knowledge, (Steal, Steal) is the subgame equilibrium for the subgame before that, and the one before that, and so on.
So by backwards induction, the players will (Steal, Steal) at every iteration.

\question{1.3}

Yes. Consider the singleton subgame of just the last game in the finite series of games.
Recall that by Nash's Existence Theorem, every finite game has at least one Nash equilibrium.
Therefore there must exist a subgame perfect Nash Equilibrium.

\question{2}

Prove:

                                Col Player picks strategy q
                                ===========================
Row Player picks strategy p     |
                                |
                                ===========================
                                |
                                |
                                ===========================

max p min q M <= min q max p M

AFSOC

max p min q M > min q max p M



Assume that "not knowing" your opponents strategy in advance
is strictly better than "knowing it".

However, there's no way this is possible,
since "not knowing" something
can't give you an advantage over "knowing it"
because "knowing it" potentially gives you the advantage
to pick a strategy that is as good if not better.


\end{document}

