\documentclass[11pt,letterpaper]{article}

\usepackage{amsmath}
\usepackage{amssymb}
\usepackage{fancyhdr}

\oddsidemargin0cm
\topmargin-2cm
\textwidth16.5cm
\textheight23.5cm

\newcommand{\question}[1] {\vspace{.25in} \hrule\vspace{0.5em}
\noindent{\bf #1} \vspace{0.5em}
\hrule \vspace{.10in}}
\renewcommand{\part}[1] {\vspace{.10in} {\bf (#1)}}

\newcommand{\myname}{Karan Sikka}
\newcommand{\myandrew}{ksikka@cmu.edu}
\newcommand{\myhwnum}{05}

\setlength{\parindent}{0pt}
\setlength{\parskip}{5pt plus 1pt}

\pagestyle{fancyplain}
\lhead{\fancyplain{}{\textbf{HW\myhwnum}}}
\rhead{\fancyplain{}{\myname\\ \myandrew}}
\chead{\fancyplain{}{15-381}}

\begin{document}

\medskip

\thispagestyle{plain}
\begin{center}                  % Center the following lines
{\Large 21-301 Assignment \myhwnum} \\
\myname \\
\myandrew \\
\today
\end{center}

\question{1a}
The main difference in exploration/exploitation and UCB is that exploration affords
us an opportunity to discover unknown properties of the arms, and UCB goes straight to exploitation.

Consider the case where an arm A's reward increases linearly but arm B's reward increases exponentially (as a function of the number of pulls).
Note that initially, the rewards look quite similar.
In UCB, you'd initially see arm A have higher reward than arm B, so you'd pull arm A more frequently
and eventually you'd be so confident that A is the better arm so you'd only pull arm A. Note that here you
made a premature assumption about arm B, which is that it's current performance reflects its future performance.

But if you had explored first, you'd see arm B beat arm A's reward, and it would get higher and higher rewards to outweight arm A.
So in the exploitation phase you'd pull arm B and this strategy clearly beats that of pulling arm A.

The tradeoff is that exploration allows us to learn more about the true properties of the arms at the expense of time or pulls.


\question{1b}
\begin{table}[h!]
    \begin{tabular}{l|l}
    Arm A & Arm B \\ \hline
    100   & 60    \\
    -50   & 60    \\
    100   & 60    \\
    -50   & 60    \\
    \end{tabular}
\end{table}
and so on. 

Arm A will make more money, but it has high variance. Arm B makes less money but has no variance.

\end{document}

