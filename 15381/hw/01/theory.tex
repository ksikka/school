\documentclass[11pt,letterpaper]{article}

\usepackage{amsmath}
\usepackage{amssymb}
\usepackage{fancyhdr}

\oddsidemargin0cm
\topmargin-2cm
\textwidth16.5cm
\textheight23.5cm

\newcommand{\question}[2] {\vspace{.25in} \hrule\vspace{0.5em}
\noindent{\bf #1: #2} \vspace{0.5em}
\hrule \vspace{.10in}}
\renewcommand{\part}[1] {\vspace{.10in} {\bf (#1)}}

\newcommand{\myname}{Karan Sikka}
\newcommand{\myandrew}{ksikka@cmu.edu}
\newcommand{\myhwnum}{01}

\setlength{\parindent}{0pt}
\setlength{\parskip}{5pt plus 1pt}

\pagestyle{fancyplain}
\lhead{\fancyplain{}{\textbf{HW\myhwnum}}}
\rhead{\fancyplain{}{\myname\\ \myandrew}}
\chead{\fancyplain{}{15-381}}

\begin{document}

\medskip

\thispagestyle{plain}
\begin{center}                  % Center the following lines
{\Large 15-381 Assignment \myhwnum} \\
\myname \\
\myandrew \\
9/11/13\\
\end{center}


\question{1}{Task 1.1}
\textbf{Formulation one:}

        Variables:\\
            $B, M, S, H, P, C$, where each variable has domain $\{1,2,3,4,5,6\}$

        Constraints:\\
            (define $m(n) = (n \mod 6) + 1$)
            \begin{enumerate}
            \item $B M S H P C$ must be distinct values
            \item $m(B-M) = 1$
            \item $m(S-H) = 1$
            \item $m(H-P) = 1$
            \item $m(P-C) \neq 1$, $m(S-C) \neq 1$, $m(H-C) \neq 1$
            \item $C = 1$
            \end{enumerate}

\textbf{Formulation two:}

        Variables:\\
            $A_1, A_2, A_3, A_4, A_5, A_6$, where each variable $A_i$ represents the person or animal sitting in seat $i$, and has domain $\{A,B,C,D,E,F\}$

        Constraints:\\
            (define $m(n) = (n \mod 6) + 1$)\\
            TODO

    Lastly, (TODO Why inefficient to run AC3 with more constraints?)


\question{2}{Task 1.2}
NYI

\question{3}{Task 1.3}
NYI

\question{4}{Task 1.4}
NYI

\question{5}{Task 1.5}
NYI

\question{6}{Task 2.1}
    If $X_i$ has constraint $X_i = c$, let its domain $D_i = \{ c \}$ (the singleton set containing c)
    Then the constraint is no longer necessary.

\question{7}{Task 2.2}
    $X_1 + X_2 = X_3$

    Make a variable called $Z$, and let its domain be a pair of values, where the first value may be in the domain of $X_1$ and the second in the domain of $X_2$.
    Then the three equivalent binary constraints are:
    \begin{enumerate}
    \item $Z[1] = X_1$
    \item $Z[2] = X_2$
    \item $Z[1] + Z[2] = X_3$
    \end{enumerate}

\question{8}{Task 2.3}
    $X_1 + X_2 + ... + X_{n-1} = X_n$

    Let $Z$ be an $n-1$ tuple. <TODO domain>
    Then you need a minimum of $n$ constraints:
    \begin{enumerate}
    \item    $Z[1] = X1$
    \item     $Z[2] = X2$
    \item     $...$
    \item     $Z[n-1] = X_{n-1}$
    \item     $Z[1] + Z[2] + ... + Z[n-1] = X_n$
    \end{enumerate}
    (Not good at latex but that should be 1 - $n$.
\question{9}{Task 2.4}
    Given an $n$-ary constraint, stuff $n-1$ of the variables into a new variable, let that be $Z$ in this discussion.
    Then, create $n-1$ constraints, where $Z[i] = X_i$ for $i = 1$ to $n-1$
    Then create an additional constraint, which is the original constraint, but substitute the variables with $Z$ and a positional index. $(X_i -> Z[i])$

    Why? TODO
\end{document}

