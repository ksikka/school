\documentclass[11pt,letterpaper]{article}

\usepackage{amsmath}
\usepackage{amssymb}
\usepackage{fancyhdr}

\oddsidemargin0cm
\topmargin-2cm
\textwidth16.5cm
\textheight23.5cm

\newcommand{\question}[2] {\vspace{.25in} \hrule\vspace{0.5em}
\noindent{\bf #1: #2} \vspace{0.5em}
\hrule \vspace{.10in}}
\renewcommand{\part}[1] {\vspace{.10in} {\bf (#1)}}

\newcommand{\myname}{Karan Sikka}
\newcommand{\myandrew}{ksikka@cmu.edu}
\newcommand{\myhwnum}{02}

\setlength{\parindent}{0pt}
\setlength{\parskip}{5pt plus 1pt}

\pagestyle{fancyplain}
\lhead{\fancyplain{}{\textbf{HW\myhwnum}}}
\rhead{\fancyplain{}{\myname\\ \myandrew}}
\chead{\fancyplain{}{15-210}}

\begin{document}

\medskip

\thispagestyle{plain}
\begin{center}                  % Center the following lines
{\Large 15-210 Assignment \myhwnum} \\
\myname \\
\myandrew \\
Section C\\
\today\\
\end{center}


\question{1}{Task 4.1}
For all of the recurrences, we assume that $T(1)$ is a constant.
\begin{enumerate}
\item
Given the recurrence, we can interpret it to mean:

$$ T(n) \leq 3 T(\frac{n}{4}) + k_{1}n + k_{2}$$
$$ T(n) \geq 3 T(\frac{n}{4}) + k_{3}n + k_{4}$$

for all $n \in \mathbb{N}$. Since these inequalities are so similar,
we will solve the following recurrence of a stronger statement of the
first inequality, and then use the closed form
to prove things about both inequalities:

$$ T(n) = 3 T(\frac{n}{4}) + k_{1}n + k_{2}$$
  
Then we use the tree method. Let $i$ be the index of the row of
the tree:

Value at each node:
$ (\frac{1}{4})^{i} (k_1n + k_2)$

Nodes per row: $ 3^i $

Sum of nodes per row: $ (\frac{3}{4})^i (k_1n + k_2) $

Height of tree: $lg (n)$

Sum of nodes in the tree:

$$ T(n) = \sum_{i=0}^{lg(n)} (\frac{3}{4})^{i} (k_1n + k_2) $$

We will find a closed form of this expression via algebra.

$$ T(n) = (k_1n + k_2) \sum_{i=0}^{lg(n)} (\frac{3}{4})^{i} $$

Now we use the formula for the sum of a finite geometric series,
and we use this closed form to prove the claim.

$$ T(n) \leq (k_1n + k_2) \frac{1-(\frac{3}{4})^{lg(n)+1}}{1-\frac{3}{4}} $$
$$ T(n) \geq (k_3n + k_4) \frac{1-(\frac{3}{4})^{lg(n)+1}}{1-\frac{3}{4}} $$

We observe that the numerators of the fractions will always be less than or equal to $1$, and it 
must be greater than $.25$, so we come up with the following upper and lower bounds:

$$ T(n) \leq 4(k_1n + k_2) 1 $$
$$ T(n) \geq 4(k_3n + k_4) .25 $$

Therefore, $T(n)$ is upper-bounded by $O(n)$ and lower-bounded by it as well.
Then $T(n) \in \Theta(n)$

\item
Given the recurrence, we can interpret it to mean:

$$ T(n) \leq 21 T(\frac{n-3}{23}) + k_1n + k_2 $$
$$ T(n) \geq 21 T(\frac{n-3}{23}) + k_3n + k_4 $$

Claim: $ T(n) \in \Theta(n) $

The proof is by strong induction on $n$.

Base case:\\
$ T(1) $ is some constant, so it is in $ \Theta(1) $.
Since $ n = 1 $, the claim holds for the base case.

Inductive Hypothesis:\\
$ T(x) \in \Theta(x) $ for all $ x < n $. \\
$ T(x) \leq k_5 x + k_6 $ for all $ x < n $. \\
$ T(x) \geq k_7 x + k_8 $ for all $ x < n $. \\

Inductive Step:\\
\begin{enumerate}
\item
  Proof of $ T(n) \in O(n) $
           $$ T(n) = 21 T(\frac{n-3}{23}) + k_1n + k_2 $$ 
  $$ \implies T(n) \leq 21 ( k_5 \frac{n-3}{23} + k_6 ) + k_1n + k_2 $$ 
  $$ \implies T(n) \leq 21 k_5 \frac{n-3}{23} + 21 k_6 + k_1n + k_2 $$ 
  $$ \implies T(n) \leq 21 k_5 \frac{n}{23} - 21 k_5 \frac{3}{23} + 21 k_6 + k_1n + k_2 $$ 
  Therefore, $$ T(n) \in O(n) $$

\item
  Proof of $ T(n) \in \Omega(n) $
           $$ T(n) = 21 T(\frac{n-3}{23}) + k_3n + k_4 $$ 
  $$ \implies T(n) \geq 21 ( k_7 \frac{n-3}{23} + k_8 ) + k_3n + k_4 $$ 
  $$ \implies T(n) \geq 21 k_7 \frac{n-3}{23} + 21 k_8 + k_3n + k_4 $$ 
  $$ \implies T(n) \geq 21 k_7 \frac{n}{23} - 21 k_7 \frac{3}{23} + 21 k_8 + k_3n + k_4 $$ 
  Therefore, $$ T(n) \in \Omega(n) $$

\item
  Since $ T(n) \in O(n) $ and $ T(n) \in \Omega(n) $,
  $$ T(n) \in \Theta(n) $$
\end{enumerate}

\item
Given the recurrence, we can interpret it to mean:

$$ T(n) \leq 2 T(\frac{n}{2}) + k_1 \sqrt{n} + k_2 $$
$$ T(n) \geq 2 T(\frac{n}{2}) + k_3 \sqrt{n} + k_4 $$

Claim: $ T(n) \in \Theta(n) $

The proof is by strong induction on $n$.

Base case:\\
$ T(1) $ is some constant, so it is in $ \Theta(1) $.
Since $ n = 1 $, the claim holds for the base case.

Inductive Hypothesis:\\
$ T(x) \in \Theta(x) $ for all $ x < n $. \\
$ T(x) \leq k_5 x + k_6 $ for all $ x < n $. \\
$ T(x) \geq k_7 x + k_8 $ for all $ x < n $. \\

Inductive Step:\\
\begin{enumerate}
\item
  Proof of $ T(n) \in O(n) $
           $$ T(n) \leq 2 T(\frac{n}{2}) + k_1 \sqrt{n} + k_2 $$
  $$ \implies T(n) \leq 2 (k_5 \frac{n}{2} + k_6) + k_1 \sqrt{n} + k_2 $$
  $$ \implies T(n) \leq 2 k_5 \frac{n}{2} + 2 k_6 + k_1 \sqrt{n} + k_2 $$
  $$ \implies T(n) \leq k_5 n + 2 k_6 + k_1 \sqrt{n} + k_2 $$
  Therefore, $$ T(n) \in O(n) $$

\item
  Proof of $ T(n) \in \Omega(n) $
           $$ T(n) \geq 2 T(\frac{n}{2}) + k_3 \sqrt{n} + k_4 $$
  $$ \implies T(n) \geq 2 k_7 \frac{n}{2} + 2 k_8 + k_3 \sqrt{n} + k_4 $$
  $$ \implies T(n) \geq k_7 n + 2 k_8 + k_3 \sqrt{n} + k_4 $$
  Therefore, $$ T(n) \in \Omega(n) $$

\item
  Since $ T(n) \in O(n) $ and $ T(n) \in \Omega(n) $,
  $$ T(n) \in \Theta(n) $$
\end{enumerate}

\item
Given the recurrence, we can interpret it to mean:

$$ T(n) \leq \sqrt{n} T(\sqrt{n}) + k_1 n^2 + k_2 $$
$$ T(n) \geq \sqrt{n} T(\sqrt{n}) + k_3 n^2 + k_4 $$

Claim: $ T(n) \in \Theta(n^2) $

The proof is by strong induction on $n$.

Base case:\\
$ T(1) $ is some constant, so it is in $ \Theta(1) $.
Since $ n = 1 $, the claim holds for the base case.

Inductive Hypothesis:\\
$ T(x) \in \Theta(x^2) $ for all $ x < n $. \\
$ T(x) \leq k_5 x^2 + k_6 $ for all $ x < n $. \\
$ T(x) \geq k_7 x^2 + k_8 $ for all $ x < n $. \\

Inductive Step:\\
\begin{enumerate}
\item
  Proof of $ T(n) \in O(n^2) $
           $$ T(n) \leq \sqrt{n} T(\sqrt{n}) + k_1 x^2 + k_2 $$
  $$ \implies T(n) \leq \sqrt{n} (k_5 \sqrt{n}^2 + k_6) + k_1 x^2 + k_2 $$
  $$ \implies T(n) \leq \sqrt{n} (k_5 n + k_6) + k_1 x^2 + k_2 $$
  Therefore, $$ T(n) \in O(n^2) $$

\item
  Proof of $ T(n) \in \Omega(n^2) $
           $$ T(n) \geq \sqrt{n} T(\sqrt{n}) + k_3 x^2 + k_4 $$
  $$ \implies T(n) \geq \sqrt{n} (k_7 \sqrt{n}^2 + k_8) + k_3 x^2 + k_4 $$
  $$ \implies T(n) \geq \sqrt{n} (k_7 n + k_8) + k_3 x^2 + k_4 $$
  Therefore, $$ T(n) \in \Omega(n^2) $$

\item
  Since $ T(n) \in O(n^2) $ and $ T(n) \in \Omega(n^2) $,
  $$ T(n) \in \Theta(n^2) $$
\end{enumerate}

\end{enumerate}

\question{2}{Task 5.2}
The 2 parallel recursive calls of $|S|/2$ size each are responsible for the 
first term in the recurrence. The divide step has work of $O(n)$, so it's responsible
for the 2nd term in the recurrence. The combine step has work of $O(n)$ for the following
  reasons. The only functions I call are \texttt{map}, \texttt{merge}, and \texttt{scanI}.
  All of the \texttt{map} calls use a constant time function on a linear input. The merge 
  function has linear work. The scanI function uses a constant time function, so it too
  is linear in work. Therefore the skyline helper function conforms to the recurrence.
The wrapper around skyline which gets rid of redundant points, does so in linear work as well,
since it uses a tabulate, filter, and map with a constant time functions on linear inputs.

\texttt{Proof of closed form of recurrence:}\\
$$W_{skyline} (n)= 2 W_{skyline}(\frac{n}{2}) + O(n) + W_{combine}  $$
$$ \implies W_{skyline} (n)= 2 W_{skyline}(\frac{n}{2}) + O(n) + O(n)  $$
$$ \implies W_{skyline} (n)= 2 W_{skyline}(\frac{n}{2}) + k_1 n + k_2  $$

This is the mergesort recurrence. But I'll prove that $W(n) \in O(n \log( n))$ anyway.
The proof is by induction. Base case: When the length of the sequence is 0 or 1,
the correct answer is returned in constant time, which is in $O(n \log( n))$.

Inductive Hypothesis: \\
$$W_{skyline}(x) \leq k_3 x \log(x) + k_4$$ for natural numbers $x$ less than $n$.

Inductive Step:
$$W_{skyline}(n)= 2 W_{skyline}(\frac{n}{2}) + k_1 n + k_2$$
$$\implies W_{skyline}(n) \leq 2 (k_3 \frac{x}{2} \log(\frac{x}{2}) + k_4) + k_1 n + k_2$$
$$\implies W_{skyline}(n) \leq k_3 x \log(\frac{x}{2}) + \frac{k_4}{2} + k_1 n + k_2$$
$$\implies W_{skyline}(n) \leq k_3 x \log(\frac{x}{2}) - k_3 x \log(2) + \frac{k_4}{2} + k_1 n + k_2$$
$$\implies W_{skyline}(n) \in O(n \log(n))$$








\question{3}{Task 5.3}
A building is defined as $(l,h,r)$ where $0 \leq l < r$ and $h > 0$.

Given a sequence of ``buildings", the skyline function returns a sequence of
"points" sorted by x values, where the x-values correspond exactly to the 
l and r values of buildings, and the y value is the max height of the buildings
which satisfy the condition $l \leq x < r$.

The proof is by structural induction.

\textbf{Base case:}\\
case of \texttt{Seq.empty}: \\
In this case, the empty sequence is returned, which is the correct output
by the definition of the skyline.

case of \texttt{Seq.singleton l,h,r}: \\
In this case, the sequence of \texttt{\textless (l,h), (r,0) \textgreater} is returned, which is correct.

\textbf{Inductive Hypothesis:} \\
We assume that skyline is correct for all sequences of length up to $n-1$.

\textbf{Inductive Step:}
Consider sequence $S$ where $|S| > 1$. Lets call the values of $l$ and $r$
of all of the elements in $S$, the set of ``$x$ values of $S$". The problem 
states that we can assume that the $x$ values of $S$ are unique.

We can partition the sequence of buildings $S$ into $s1$ and $s2$,
where each have at most $|S|/2$ elements.
We call the skyline function on each, and since $|S|/2 < |S|$, the IH applies.
Therefore the results of the skyline calls are correct skyline outputs.
Let \texttt{skyA = skyline s1} and \texttt{skyB = skyline s2}.

By the definition of skyline, \texttt{skyA} contains all of the x-values of $s1$,
and \texttt{skyB} contains all of the x-values of $s2$. Thanks to the IH,
the tuples in \texttt{skyA} and \texttt{skyB} are sorted by x-values. If we merge
the $x$ values of $s2$ into \texttt{skyA}, the resulting sequence would have all of 
the $x$ values of $S$, since the x-values of \texttt{skyA} and \texttt{skyB} partition the
set of $x$ values of $S$. We merge the $x$ values of $s2$ into \texttt{skyA}, and
$s1$ into \texttt{skyB}, such that each entry from $s2$ in \texttt{skyA} will have \texttt{NONE} as its
corresponding $y$ value, and each each entry in \texttt{skyA} not from $s2$ will have \texttt{SOME y}
as its corresponding y value. Likewise for \texttt{skyB} and $s1$.
Then for \texttt{skyA} and \texttt{skyB},
we can perform a copy scan to propagate the height of the skyline at the newly
added points to those points-- replacing \texttt{NONE} with the closest \texttt{SOME} on the left.
The result, is the same skyline as returned by the skyline function, but with
added points that add resolution to the skyline.

Since we merged two sorted sets (\texttt{skyA} and $x$ values of $s2$), the result is sorted.
The result also contains all of the x-values in $S$. Both \texttt{skyA} and \texttt{skyB} do. Therefore,
they each have the same length at this point $(|S|)$, and tuples at the same indices
have the same x-values (since they are distinct and sorted in both $skyA$ and $skyB$).
The y-values of the tuples represent the height of the skyline.

Finally we can do a \texttt{map2}
to return a sequence of points which have the max height of \texttt{skyA[i]} and \texttt{skyB[i]}
for all $i$.

Since we have the max heights over all the $x$ values in the input sequence,
the skyline function is correct in the inductive step. 

\end{document}

