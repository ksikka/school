\documentclass[11pt,letterpaper]{article}

\usepackage{amsmath}
\usepackage{amssymb}
\usepackage{fancyhdr}
\usepackage{textcomp}
\usepackage{upgreek}

\oddsidemargin0cm
\topmargin-2cm
\textwidth16.5cm
\textheight23.5cm

\newcommand{\question}[2] {\vspace{.25in} \hrule\vspace{0.5em}
\noindent{\bf #1: #2} \vspace{0.5em}
\hrule \vspace{.10in}}
\renewcommand{\part}[1] {\vspace{.10in} {\bf (#1)}}

\newcommand{\myname}{Karan Sikka}
\newcommand{\myandrew}{ksikka@cmu.edu}
\newcommand{\myhwnum}{06}

\setlength{\parindent}{0pt}
\setlength{\parskip}{5pt plus 1pt}

\pagestyle{fancyplain}
\lhead{\fancyplain{}{\textbf{HW\myhwnum}}}
\rhead{\fancyplain{}{\myname\\ \myandrew}}
\chead{\fancyplain{}{15-210}}

\begin{document}

\medskip

\thispagestyle{plain}
\begin{center}                  % Center the following lines
{\Large 15-210 Assignment \myhwnum} \\
\myname \\
\myandrew \\
Section C\\
\today\\
\end{center}


\question{1}{Task 3.1}
\begin{enumerate}
\item Convert edge sequence to adjacency table (vertex \textrightarrow vertex set). Call this graph $G$.
\item Let $X$ be a set containing vertexes which are independent. $X$ starts as the empty set.
\item Assign random numbers to the vertexes by using a hash function $h$.
\item Select all vertexes which have the highest $h(v)$ compared to their neighbors.
      Call them ``localMaxVertexes''. 
\item The localMaxVertexes are guaranteed to form an independent set, since
      no more than one localMaxVertex exists in a set of adjacent vertexes.
      Add these independent vertexes to $X$.
\item Find the subgraph of the current graph which doesn't contain
      any localMaxVertexes, or any of their neighbors. Call this $G'$.
      Note that $G'$ is independent from $X$.
\item Repeat from step 3, with $G$ as $G'$, until $G'$ contains no vertexes.
\item When $G'$ contains no vertexes, return $X$, which shall be a 
      maximal independent set.
\end{enumerate}





\question{2}{Task 3.4}
$$p_{v} = \frac{1}{1 + \deg(v) }$$





\question{3}{Task 3.5}
Since a vertex has probability 
$\frac{1}{1 + \deg(v)}$
of being added to the MIS,
it has probability
$\frac{\deg(v)}{1 + \deg(v)}$
of not being added.

Let $X_v$ is be an indicator random variable,
which is 1 if $v$ is not being added, and 0 if added.
Then $\mathbb{E}[X_v] = \frac{\deg(v)}{1 + \deg(v)}$.

Let $X$ be a random variable which is
the number of vertices added to the MIS in the graph after
one step of the algorithm. We see that $$X = \sum_{v \in V} X_v$$

Therefore, by linearity of expectation, $$\mathbb{E}[X] = \sum_{v \in V} \mathbb{E}[X_v] \leq |V| \frac{\Updelta}{1 + \Updelta}$$
Since delta is non-negative, we see that the delta term is some fraction in $[0,1)$.
Call the fractional delta term $\frac{1}{k}$ so that the expected number of vertexes
added to the MIS after one step of the algo is $\frac{|V|}{k}$. 

We know that in our current algorithm that all vertices added to the MIS are removed
from the frontier/input of the recursive call. Therefore, in expectation, at most $\frac{|V|}{k}$
vertices are retained after 1 step of the algorithm.

$$W_{\text{MIS}}(|V|) \leq W(\frac{|V|}{k}) + c_1|V| + c_2$$

Since the input size decreases by a constant every iteration, we know that the work is root dominated, and will
have $W \in O(|V|)$.

Further, more rigorous proof can be found using the tree method,
which shows that at each level, the work at each level is $c_1k^{-i}|V|+c_2$ at level $i$,
and there are $ \log_{k}(|V|)$ of levels in the tree. If we sum up the work in the tree,
we get the summation,

$$W(|V|) \leq \sum_{i=0}^{log_k(|V|)} c_1k^{-i}|V|+c_2$$
$$W(|V|) \leq c_2\log_k(|V|) + c_1|V|\sum_{i=0}^{log_k(|V|)} k^{-i}$$

The summation evaluates to some constant, since $k \in [0,1)$ and
this is the sum of a geometric series. Therfore,
$$W(|V|) \leq c_2\log_k(|V|) + c_1c_3|V|$$
$$W(|V|) \in O(|V|)$$

As for span, we know that the expected number of iterations of the algorithm
is in $O(\log(|V|)$. The span of each step is at most $O(\log(|V|)$, because
all operations are filter/map/reduce (with const. time functions),
and all input sizes are $O(|V|)$. \texttt{isLocalMax} is constant time
because it does work proportional to the degree of the vertex, which
in this case is upper-bounded by a constant.
$$S(\log^2(|V|))$$



\question{4}{Task 4.1}
We'll create a graph with the following properties.
The exams will be vertices. Two exams will be connected by an edge
if there exists a student who is taking both exams. Solving the
graph coloring problem on this graph solves the exam scheduling problem,
because connected graphs must have different colors, representing
that if a student is taking 2 exams, they must occur at different times.
The minimum number of colors necessary is the minimum number of time slots
necessary.

Let $n$ be the number of students and
    $m$ be the number of exams.
Say you are given a sequence of exam sequences. The index into the
outer sequence represents the number of the student in the ordering,
and the inner sequence is a list of exams which the student is taking.

We will build an adjacency matrix for the graph. For each student,
for each exam the student is taking, ``connect'' the exam with all
the other exams the student is taking by updating the adjacency matrix
in constant work for each exam-exam pair. In the worst case,
each student is taking every exam. In this case, the work of this algorithm would be
$$O(nm^{2})$$.



\question{5}{Task 4.5}
Let $n$ be the number of students, and $m$ be the number of exams.
The work of the function is $$O(nm^2\log(nm))$$ because it does what is described in
the reduction for $4.1$ but it has an additional $\log$ factor for the overhead
of searching a BST for mapping string to ints and vice versa. This overshadows
the work of MIS itself.

The span is around $O(\log^3(nm^2))$

\end{document}

