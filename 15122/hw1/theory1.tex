\documentclass[12pt]{exam}
\usepackage{amsmath}
\usepackage[left=1in, right=1in, top=1in, bottom=1in]{geometry}

\newcommand\Cnought{C$_0$}
\newcommand\modulo{\ \texttt{\%}\ }
\newcommand\lshift{\ \texttt{<<}\ }
\newcommand\rshift{\ \texttt{>>}\ }
\newcommand\cgeq{\ \texttt{>=}\ }

\newcommand{\answerbox}[1]{
\begin{framed}
\hspace{5.65in}
\vspace{#1}
\end{framed}}


\pagestyle{head}

\headrule \header{\textbf{15-122 Assignment 1}}{}{\textbf{Page
\thepage\ of \numpages}}

\pointsinmargin \printanswers

\setlength\answerlinelength{2in} \setlength\answerskip{0.3in}

\begin{document}
\addpoints
\begin{center}
\textbf{\large{15-122 : Principles of Imperative Computation
 \\ \vspace{0.2in} Fall 2011
\\  \vspace{0.2in} Assignment 1
}}

 \vspace{0.2in}
 (\large{Theory Part})

 \vspace{0.2in}

 \large{Due: Thursday, September 15, 2011 in class}
\end{center}

\vspace{0.5in}

\hbox to \textwidth{Name:\enspace\hrulefill}


\vspace{0.2in}

\hbox to \textwidth{Andrew ID:\enspace\hrulefill}

\vspace{0.2in}

\hbox to \textwidth{Recitation:\enspace\hrulefill}



% End title box
\vspace{0.5in}

\noindent The written portion of this week's homework will give you some practice
working with the binary representation of integers and reasoning with
invariants.  You should type up your solutions or write them
\textit{neatly} by hand, and you should submit your work in class on the
due date just before lecture or recitation begins.
\vspace{0.2in}

\noindent \textbf{Be sure to staple your homework
before you submit it and make sure your name and section letter is
clearly shown.}

\vspace{0.2in}

\begin{center}
\gradetable[v][questions]
\end{center}


\newpage
\begin{questions}
\question{\textbf{Basics of \Cnought{}}}

\begin{parts}
\part[4]
Let $x$ be an \texttt{int} in the \Cnought{} language.  Express the
following operations in \Cnought{} using only constants and the bitwise operators
(\texttt{\&}, \texttt{|}, \texttt{\^}, \texttt{\~}, \texttt{<<}, \texttt{>>}). [1 pt each]
\vspace{0.2in}
\begin{subparts}
    \subpart Set $a$ equal to $x$ multiplied by 64.
\begin{solution}
\vspace{0.6in}
\end{solution}



\vspace{0.1in}
    \subpart Set $b$ equal to $x \modulo 16$, assuming that $x$ is positive.
\begin{solution}
\vspace{0.6in}
\end{solution}

\vspace{0.1in}
    \subpart Set $c$ equal to $x$ with its middle 12 bits all set to 1.
\begin{solution}
\vspace{0.6in}
\end{solution}

\vspace{0.1in}
    \subpart Set $d$ equal to the intensity of the green component of $x$,
        assuming $x$ stores the packed representation of an RGB color.
The intensity should be a value between 0 and 255, inclusive.
\begin{solution}
\vspace{0.6in}
\end{solution}


\end{subparts}

\newpage


\part[2]
Are the following two \texttt{bool} expressions equivalent in \Cnought{},
     assuming \texttt{x} and \texttt{y} are of type \texttt{int}?  Explain your answer.
\begin{verbatim}
      (x/y < 122) && (y != 0)          (y != 0) && (x/y < 122)
\end{verbatim}

\begin{solution}
\vspace{1in}
\end{solution}

\vspace{0.1in}
\part[4]
For each of the following statements, determine whether the statement is
true or false in \Cnought{}.  If it is true, explain why.
If it is false, give a counterexample to illustrate why.

\begin{subparts}
\vspace{0.1in}
    \subpart For every \texttt{int} $x$: $x + 1 > x$.
\begin{solution}
\vspace{0.6in}
\end{solution}

    \vspace{0.1in}
    \subpart For every \texttt{int} $x$:
          $x \lshift 1$ is equivalent to $x * 2$.
\begin{solution}
\vspace{0.6in}
\end{solution}


    \vspace{0.1in}
    \subpart For every \texttt{int} $x$: $x / 10 * 10 + x \% 10$ is equivalent to $x$.
\begin{solution}
\vspace{0.6in}
\end{solution}


    \vspace{0.1in}
    \subpart For every \texttt{bool} $a$ and \texttt{bool} $b$:
          $a$ \texttt{\&\&} \texttt{!}$b$  is equivalent to \texttt{!(!}$a$ \texttt{||} $b$ \texttt{)}.
\begin{solution}
\vspace{0.6in}
\end{solution}

\end{subparts}
\end{parts}

\newpage


\question{\textbf{Reasoning with Invariants}}
\begin{parts}

The Fibonacci sequence is shown below:
\begin{verbatim}
    1, 1, 2, 3, 5, 8, 13, 21, 34, 55, ...
\end{verbatim}
Each integer in the sequence is the sum of the previous two integers in the sequence. Consider the following implementation for \texttt{fastfib} that returns the n\textsuperscript{th} Fibonacci number
(the body of the loop is not shown).
\begin{verbatim}
int fib(int n)
//@requires n >= 1;
{
     if (n <= 2) return 1;
     else return fib(n-1) + fib(n-2);
}
int fastfib(int n)
//@requires n >= 1;
//@ensures \result == fib(n);
{
     if (n <= 2) return 1;
     int i = 1; int j = 1;
     int k = 2; int x = 3;
     while (x < n)
     //@loop_invariant 3 <= x && x<=n && i==fib(x-1) && j==fib(x-2) && k==i+j;
     {
            // LOOP BODY NOT SHOWN
     }
     return k;
}
\end{verbatim}


\part[3]
Using the precondition and loop invariant, reason that the \texttt{fastfib} function
must return the correct answer, even if you don't know what is in the body of the loop.
You may assume the loop invariant is correct.
\begin{solution}
\vspace{2.55in}
\end{solution}

\newpage
\part[2] Based on the given loop invariant, write the body of the loop.

\begin{solution}
\vspace{3.7in}
\end{solution}

\vspace{0.1in}
\part[2]
What is the largest Fibonacci number that can be generated by your \texttt{fastfib} program before
\textit{overflow} occurs? (Overflow occurs when you add two positive integers together
and get a negative result, or when you add two negative integers together and get a
positive result.) Show how you derived your answer.
\begin{solution}
\vspace{2.8in}
\end{solution}

\end{parts}


\newpage
\question{\textbf{More on Reasoning with Invariants}}
\begin{parts}

A \Cnought{} programmer was writing a function to add up the first $n$ natural numbers
and, after testing, noticed that the first $n$ natural numbers always seemed
to add up to $n (n-1)/2$. To verify this, the programmer added annotations to the function
as shown below:
\vspace{0.1in}
\begin{verbatim}
int sum_first(int n)
//@requires n > 0;
//@ensures 2 * \result == n * (n - 1);
{
     int sum = 0;
     int i = 0;
     while (i < n)
     //@loop_invariant 0 <= i && i <= n;
     //@loop_invariant 2 * sum == i * (i - 1);
     {
        sum = sum + i;
        i = i + 1;
     }
     return sum;
}
\end{verbatim}
Prove that the postcondition (\texttt{ensures}) holds for the function using
the given precondition(\texttt{requires}) and the loop invariants::

\part[1]
Give a brief argument explaining why the loop must terminate.
\begin{solution}
\vspace{1.3in}
\end{solution}

\vspace{0.1in}
\part[1]
Show that each loop invariant is true immediately before the loop condition is tested for the first time.
\begin{solution}
\vspace{1.3in}
\end{solution}



\part[4]
Show that if each loop invariant is true at the start of a
    loop iteration, then the loop invariants are also all true at the end
    of that iteration.
\begin{solution}
\vspace{4.2in}
\end{solution}

\vspace{0.1in}

\part[2]
Show that if the loop terminates, the postcondition must hold.
\begin{solution}
\vspace{2.7in}
\end{solution}



\vspace{0.1in}



    \end{parts}


\end{questions}
\end{document}
