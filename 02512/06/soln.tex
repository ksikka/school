\documentclass[11pt,letterpaper]{article}

\usepackage{amsmath}
\usepackage{amssymb}
\usepackage{fancyhdr}
\usepackage{verbatim}
\usepackage{graphicx}


\oddsidemargin0cm
\topmargin-2cm
\textwidth16.5cm
\textheight23.5cm

\newcommand{\question}[1] {\vspace{.25in} \hrule\vspace{0.5em}
\noindent{\bf #1} \vspace{0.5em}
\hrule \vspace{.10in}}
\renewcommand{\part}[1] {\vspace{.10in} {\bf (#1)}}

\newcommand{\myname}{Karan Sikka}
\newcommand{\myandrew}{ksikka@cmu.edu}
\newcommand{\myhwnum}{06}

\setlength{\parindent}{0pt}
\setlength{\parskip}{5pt plus 1pt}

\pagestyle{fancyplain}
\lhead{\fancyplain{}{\textbf{PS\myhwnum}}}
\rhead{\fancyplain{}{\myname\\ \myandrew}}
\chead{\fancyplain{}{02-512}}

\begin{document}

\medskip

\thispagestyle{plain}
\begin{center}                  % Center the following lines
{\Large 02-512 Assignment \myhwnum} \\
\myname \\
\myandrew \\
\today
\end{center}

\question{1}
\part{a}
The variables we are trying to find are 
the rate of infection $\lambda_1$
and the rate of recovery, $\lambda_2$.

We can run the CTMM as a simulation
and compare it with real data.
The output of the simulation will be sequence of states over time
where each state will be $(S_t,I_t,R_t)$.

Let $(Sr_t, Ir_t, Rr_t)$ be the real data.
One possible objective function to minimize is 
$$ \sum (S_t - Sr_t)^2 + (I_t - Ir_t)^2 + (R_t - Rr_t)^2 $$

TODO Algorithm

\part{b}
Let $G$ be the growth rate,
$x_1$ be the conc of nutrient 1
$x_2$ be the conc of nutrient 2.

$$G = \alpha x_1 + \beta x_2 + c$$

Where $\alpha, \beta, c$ are the parameters we're trying to estimate.

Let $Gr(x_1,x_2)$ be the experimenally determined growth rate.

One possible objective function to minimize is

$$ \sum (Gr(x_1, x_2) - G(x_1, x_2))^2 $$

We can find the parameters which minimize this using
steepest descent.

\part{c}
BB - brown
Bb - brown
bb - blue

 brown = BB + Bb
 blue = bb

TODO

\problem{2}
\part{a}
Let TTR be time to resistance.

TODO GAUSSIAN LIKELIHOOD

\part{b}
TODO GAUSSIAN\^2 LIKELIHOOD

\part{c}
Do both and take the one with the smaller error?

\part{d}
Metropolis

\part{e}
Sampling vs solving

\problem{3}
Given two points, the eqn of a line is derived as follows:

(Points given as
    t2 x2 t1 x1)

t = (x2 - x2)/(t2 - t1) x + b

    t1 - (x2 - x2)/(t2 - t1) x1 = b

    or 

    t2 - (x2 - x2)/(t2 - t1) x2 = b

t - t2 = (x2 - x2)/(t2 - t1) * (x - x2)



if $0 \leq t < 2$, t - 2 = (5/2)(x - 5)\\
if $2 \leq t < 5$, t - 5 = ((6-5)/(5-2))(x-6)\\
if $5 \leq t < 8$, t - 8 = ((10-6)/(8-5))(x-10)\\
if $8 \leq t \leq 10$, t - 10 = ((20-10)/(8-10))(x-20)

\part{b}
TODO interp formula

\part{c}
TODO interp formula + some deriv stuff.

\problem{4}
\part{a}

If $b_i$ is 0, there are no boojum on island $i$. Based on this observation, we note the following:\\
$Pr(b_i = 0| f) = (1-f)^{s_i}$\\
$Pr(b_i = 1| f) = 1 - (1-f)^{s_i}$

$$P(b | \theta) = \Pi_{i=1}^n Pr(b_i = b_i)$$

\part{b}





\end{document}
