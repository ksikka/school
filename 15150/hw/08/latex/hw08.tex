\documentclass[11pt,letterpaper]{article}

\usepackage{amsmath}
\usepackage{amssymb}
\usepackage{fancyhdr}

\oddsidemargin0cm
\topmargin-2cm
\textwidth16.5cm
\textheight23.5cm

\newcommand{\question}[2] {\vspace{.25in} \hrule\vspace{0.5em}
\noindent{\bf #1: #2} \vspace{0.5em}
\hrule \vspace{.10in}}
\renewcommand{\part}[1] {\vspace{.10in} {\bf (#1)}}

\newcommand{\myname}{Karan Sikka}
\newcommand{\myandrew}{ksikka@andrew.cmu.edu}
\newcommand{\myhwnum}{8}

\setlength{\parindent}{0pt}
\setlength{\parskip}{5pt plus 1pt}

\pagestyle{fancyplain}
\lhead{\fancyplain{}{\textbf{HW\myhwnum}}}
\rhead{\fancyplain{}{\myname\\ \myandrew}}
\chead{\fancyplain{}{15-150}}
%% cost graph constants
\newcommand{\costzero}{\mathbf{0}}
\newcommand{\costunit}{\mathbf{1}}

\newcommand{\maximum}[2]{\mathrm{max}(#1, #2)}

%% Evaluation and step judgements
\newcommand{\eval}[2]{{#1} \Longrightarrow {#2}}
\newcommand{\stepseq}[2]{{#1} \longmapsto {#2}}
\newcommand{\nstepseq}[2]{{#1} \longmapsto^* {#2}}

%% SML Syntax
\newcommand{\sml}[1]{\texttt{#1}}
\newcommand{\msml}[1]{\mathtt{#1}}

\newcommand{\mexp}{\msml{e}}
\newcommand{\mexps}[1]{\msml{e_{#1}}}
\newcommand{\mval}{\msml{v}}
\newcommand{\mvals}[1]{\msml{v_{#1}}}
\newcommand{\mcase}[5]{\msml{case}~{#1}~\msml{of}~{#2} => {#3}~|~{#4} => {#5}}
\newcommand{\mapp}[2]{{#1}~{#2}}
\newcommand{\mtrue}{\msml{true}}
\newcommand{\mfalse}{\msml{false}}
\newcommand{\mlist}{\msml{l}}
\newcommand{\mlists}[1]{\msml{l_{#1}}}
\newcommand{\mappend}[2]{{#1} @ {#2}}
\newcommand{\mnat}{\msml{n}}
\newcommand{\mnats}[1]{\msml{n_{#1}}}
\newcommand{\mvar}{\msml{x}}
\newcommand{\mvars}[1]{\msml{x_{#1}}}
\newcommand{\mnil}{\msml{nil}}
\newcommand{\mcons}[2]{{#1}::{#2}}
\newcommand{\mzero}{\msml{0}}
\newcommand{\mpair}[2]{({#1}, {#2})}
\newcommand{\mraise}[1]{\msml{raise}~{#1}}
\newcommand{\mraiseFail}[1]{\mraise{(\msml{Fail}~{#1})}}
\newcommand{\tint}{\msml{int}}
\newcommand{\tlist}[1]{{#1}~\msml{list}}
\newcommand{\talpha}{\msml{'a}}
\newcommand{\tarr}[2]{{#1} \to {#2}}
\newcommand{\tprod}[2]{{#1} * {#2}}
\newcommand{\oftp}[2]{{#1}\ :\ {#2}}

%% Fitted parentheses
\newcommand{\fp}[1]{\left ( {#1} \right )}
\newcommand{\fcp}[1]{\left \{ {#1} \right \}}
\newcommand{\fpfrac}[2]{\fp{\frac{#1}{#2}}}


%% Cost
\newcommand{\work}[2]{W_{\text{#1}}\fp {#2}}
\newcommand{\depth}[2]{D_{\text{#1}}\fp {#2}}
\newcommand{\spce}[2]{S_{\text{#1}}\fp {#2}}

\renewcommand{\O}[1]{O \fp{#1}}

\newcommand{\seq}[1]{\left \langle {#1} \right \rangle}
\renewcommand{\mlist}[1]{\left [ {#1} \right ]}

%% Formatting type ascription properly in text mode
\newcommand{\asc}[2]{{#1}\ :\ {#2}}


\begin{document}

\medskip

\thispagestyle{plain}
\begin{center}                  % Center the following lines
{\Large 15-150 Assignment \myhwnum} \\
\myname \\
\myandrew \\
G \\
\today \\
\end{center}


\question{1}{Task 3.1}
\textbf{Claim 1:} The empty queues are related.\\
Formally, $R(\msml{LQ.emp},\msml{LLQ.emp})$.

\textbf{Proof:} Let \sml{f} and \sml{b} be lists. Consider \sml{LLQ.emp $\cong$ (f,b) $\cong$ ([],[])}.

Consider:
\begin{align*}
      & \sml{f@(rev b)}                      & \\
\cong & \sml{[]@(rev [])}                    &\text{def of LLQ.emp} \\
\cong & \sml{[]@(case [] of [] => [] | ...)} &\text{step} \\
\cong & \sml{[]@[]}                          &\text{evaluation of case} \\
\cong & \sml{[]}                             &\text{Lemma 3}
\end{align*}
Since \sml{f@(rev b)} is equivalent to \sml{LQ.emp $\cong$ []}, 
R(LQ.emp,LLQ.emp) holds by definition of R.

\textbf{Claim 2:} Insertion preserves relatedness.\\
Formally, for all \sml{x:int}, \sml{l:int list}, \sml{f:int list}, \sml{b:int list},
if R(\sml{l}, \sml{(f,b)}), then R(\sml{LQ.ins(x,l)}, \sml{LLQ.ins(x,(f,b))})

\textbf{Proof:}
Let \sml{x:int}, \sml{l:int list}, \sml{f:int list}, \sml{b:int list} be arbitrary,
and assume that:\\
R(\sml{l}, \sml{(f,b)})

We want to show that R(\sml{LQ.ins(x,l)}, \sml{LLQ.ins(x,(f,b))}).

\textbf{Define} \sml{(f',b') $\cong$ LLQ.ins(x,(f,b))}
Consider:
\begin{align*}
      & \sml{(f',b')} &\\
\cong & \sml{LLQ.ins(x,(f,b))}) &\\
\cong & \sml{(f,x::b)} & \text{step} \\
\cong & \sml{(f,x::b)} & \text{step} \\
\end{align*}
Now we know that \sml{(f',b') $\cong$ (f,x::b)}.
Now consider:
\begin{align*}
      & \sml{f' @ (rev b')} & \\
\cong & \sml{f @ (rev (x::b))} &\text{proven above} \\
\cong & \sml{f @ (case (x::b) of [] => [] | x::xs => (rev xs) @ x)} &\text{step} \\
\cong & \sml{f @ ((rev b) @ [x])} &\text{case evaluation} \\
\cong & \sml{(f @ (rev b)) @ [x]} &\text{Lemma 1} \\
\cong & \sml{l @ [x]} &\text{Lemma 1} \\
\cong & \sml{LQ.ins(l,x)} &\text{step, equiv is transitive} \\
\end{align*}
Therefore, by the definition of R, we have proven that R(\sml{LQ.ins(x,l)}, \sml{(f',b')}).
Then, by the our definition of \sml{(f',b')}, we have proven the claim. 

\textbf{Claim 3:} On related queues, removal gives equal integers and related queues.\\
More formally, we must prove the following:
For all \sml{x:int}, \sml{l:int list}, \sml{f:int list}, \sml{b:int list}, if R(\sml{l}, \sml{(f,b)}) then one of the
following is true:

\sml{LQ.rem l $\cong$ NONE} and \sml{LLQ.rem (f,b) $\cong$ NONE}\\
or\\
There exist \sml{x:int, y:int}, \sml{l':int list}, \sml{f':int list}, \sml{b':int list}, such that
\begin{enumerate}
\item \sml{LQ.rem l $\cong$ SOME(x,l')}
\item \sml{LLQ.rem (f,b) $\cong$ SOME(y,(f',b'))}
\item \sml{x = y}
\item R(\sml{l'}, \sml{(f',b')})
\end{enumerate}

\textbf{Proof:}
Let \sml{l:int list}, \sml{f:int list}, \sml{b:int list}.\\
Assume R(\sml{l}, \sml{(f,b)}).

First, note that $l$ may be either \sml{[]} or \sml{x::xs} (empty or non-empty).
We will examine these two cases separately.

\textbf{Case \sml{l$\cong$[]}}
Since we know that \sml{l} and \sml{(f,b)} are related, we know \sml{[]} and \sml{(f,b)} are related.
This means that \sml{[] $\cong$ f @ (rev b)}. We can see by the definition of the \sml{@} function,
that \sml{[]} is only outputted when both inputs are \sml{[]}.
Therefore, \sml{f$\cong$[]} and \sml{(rev b) $\cong$ []}.
However, \sml{rev b} only outputs \sml{[]} when \sml{b$\cong$[]}. Therefore \sml{b$\cong$[]}.

Consider:
\begin{align*}
      & \sml{LQ.rem l} &\\
\cong & \sml{LQ.rem []} &\text{given}\\
\cong & \sml{case [] => NONE | ...} &\text{step}\\
\cong & \sml{NONE}&\text{case evaluation}
\end{align*}
And consider:
\begin{align*}
      & \sml{LLQ.rem (f,b)} &\\
\cong & \sml{LLQ.rem ([],[])} &\text{shown above}\\
\cong & \sml{case ([],[]) of ([],[]) => NONE | ...}&\text{step}\\
\cong & \sml{NONE} & \text{case evaluation}
\end{align*}
We have shown that \sml{LQ.rem l $\cong$ NONE} and \sml{LLQ.rem (f,b) $\cong$ NONE}.
Therefore the claim holds for this case.

\textbf{Case \sml{l$\cong$ x::xs}} where \sml{x:int} and \sml{xs:int list}
Consider:
\begin{align*}
      & \sml{LQ.rem l} & \\
\cong & \sml{LQ.rem x::xs} & \text{given} \\
\cong & \sml{case x::xs of [] = [] | x::xs => SOME(x,xs)} & \text{step} \\
\cong & \sml{SOME(x,xs)} & \text{step}
\end{align*}
Therefore we have shown that \sml{LQ.rem l $\cong$ SOME(x,l')} where \sml{l'} is \sml{xs}.

Since we know that \sml{l} and \sml{f},\sml{b} are related, we know that \sml{l $\cong$ x::xs $\cong$ f @ rev b}. 
By the spec of rev, we know that the length of rev b is equal to the length of b.
By the spec of @, we know that the length of l is equal to the sum length of f and rev b,
which, by the above statement, is equal to the sum of f and b. Since l has a positive length
because it is not \sml{[]}, then \sml{f+b} has a positive length. Therefore, f or b, or both have positive length.

There are 3 cases for the lengths of f and b.
\begin{enumerate}
\item length of f = length of l (b is empty)
\item length of b = length of l (f is empty)
\item f and b are both non-zero
\end{enumerate}

We will see that case 2 becomes case 1 after 1 iteration of LLQ.rem.
Consider case 2: \sml{LLQ.rem ([],b)} which steps to \sml{case ([],b) of ... | ([],\_) => rem (rev back,[])} which steps to
\sml{rem (rev b, [])}. This is clearly case 1 where \sml{f $\cong$ rev b}. We can confirm that this is true, since in case 1,
when we flatten the LLQ we get f@rev[] which is equivalent to f@[], which by lemma 3 is equivalent to f. And, 
since we assumed relatedness, \sml{l $\cong$ f }. In case 2, 
when we flatten the LLQ we get \sml{[]@rev b}, which according to Lemma to is equivalent to \sml{rev b}, and since we assumed
relatedness to l, \sml{l $\cong$ rev b}. After it all, we see case 2 is equivalent to case 1 by transitivity through l.

Therefore, we wil only prove the properties for Case 1 and 3, and say that case 2 becomes case 1.

Now to show \sml{LLQ.rem (f,b) $\cong$ SOME(y,(f',b'))} for case 1 and 3:

Consider (where f = z::zs and b is arbitrary):
\sml{LLQ.rem (f,b)} which is the same as\\
\sml{LLQ.rem (z::zs,b)} which steps to\\
\sml{case (z::zs,b) of ... | (x::xs,\_) => SOME (x, (xs,back))} which evaluates to\\
\sml{SOME (z, (zs,b))}

Clearly, the claim is true for some \sml{y, f' and b'. where y is z, f' is zs, and b' is b}.

Now we claim that \sml{x = y}. Recall that \sml{x} refers to \sml{l $\cong$ x::xs} and y  refers to \sml{f $\cong$ z::zs}.
Also recall that \sml{l $\cong$ f@rev b} by relatedness.
Proof: \\
\sml{l $\cong$ f@rev b} is true by relatedness, so by equivalence,\\
\sml{x::xs $\cong$ z::zs@rev b}

Clearly, \sml{x = z} because the first elements of two non-empty but congruent lists
are congruent and above we stated "y is z" so \sml{x $\cong$ y}.

The proof is trivial... I value my sleep...
\end{document}

