\documentclass[11pt,letterpaper]{article}

\usepackage{amsmath}
\usepackage{amssymb}
\usepackage{fancyhdr}

\oddsidemargin0cm
\topmargin-2cm
\textwidth16.5cm
\textheight23.5cm

\newcommand{\question}[2] {\vspace{.25in} \hrule\vspace{0.5em}
\noindent{\bf #1: #2} \vspace{0.5em}
\hrule \vspace{.10in}}
\renewcommand{\part}[1] {\vspace{.10in} {\bf (#1)}}

\newcommand{\myname}{Karan Sikka}
\newcommand{\myandrew}{ksikka@andrew.cmu.edu}
\newcommand{\myhwnum}{6}

\setlength{\parindent}{0pt}
\setlength{\parskip}{5pt plus 1pt}

\pagestyle{fancyplain}
\lhead{\fancyplain{}{\textbf{HW\myhwnum}}}
\rhead{\fancyplain{}{\myname\\ \myandrew}}
\chead{\fancyplain{}{15-150}}
%% cost graph constants
\newcommand{\costzero}{\mathbf{0}}
\newcommand{\costunit}{\mathbf{1}}

\newcommand{\maximum}[2]{\mathrm{max}(#1, #2)}

%% Evaluation and step judgements
\newcommand{\eval}[2]{{#1} \Longrightarrow {#2}}
\newcommand{\stepseq}[2]{{#1} \longmapsto {#2}}
\newcommand{\nstepseq}[2]{{#1} \longmapsto^* {#2}}

%% SML Syntax
\newcommand{\sml}[1]{\texttt{#1}}
\newcommand{\msml}[1]{\mathtt{#1}}

\newcommand{\mexp}{\msml{e}}
\newcommand{\mexps}[1]{\msml{e_{#1}}}
\newcommand{\mval}{\msml{v}}
\newcommand{\mvals}[1]{\msml{v_{#1}}}
\newcommand{\mcase}[5]{\msml{case}~{#1}~\msml{of}~{#2} => {#3}~|~{#4} => {#5}}
\newcommand{\mapp}[2]{{#1}~{#2}}
\newcommand{\mtrue}{\msml{true}}
\newcommand{\mfalse}{\msml{false}}
\newcommand{\mlist}{\msml{l}}
\newcommand{\mlists}[1]{\msml{l_{#1}}}
\newcommand{\mappend}[2]{{#1} @ {#2}}
\newcommand{\mnat}{\msml{n}}
\newcommand{\mnats}[1]{\msml{n_{#1}}}
\newcommand{\mvar}{\msml{x}}
\newcommand{\mvars}[1]{\msml{x_{#1}}}
\newcommand{\mnil}{\msml{nil}}
\newcommand{\mcons}[2]{{#1}::{#2}}
\newcommand{\mzero}{\msml{0}}
\newcommand{\mpair}[2]{({#1}, {#2})}
\newcommand{\mraise}[1]{\msml{raise}~{#1}}
\newcommand{\mraiseFail}[1]{\mraise{(\msml{Fail}~{#1})}}
\newcommand{\tint}{\msml{int}}
\newcommand{\tlist}[1]{{#1}~\msml{list}}
\newcommand{\talpha}{\msml{'a}}
\newcommand{\tarr}[2]{{#1} \to {#2}}
\newcommand{\tprod}[2]{{#1} * {#2}}
\newcommand{\oftp}[2]{{#1}\ :\ {#2}}

%% Fitted parentheses
\newcommand{\fp}[1]{\left ( {#1} \right )}
\newcommand{\fcp}[1]{\left \{ {#1} \right \}}
\newcommand{\fpfrac}[2]{\fp{\frac{#1}{#2}}}


%% Cost
\newcommand{\work}[2]{W_{\text{#1}}\fp {#2}}
\newcommand{\depth}[2]{D_{\text{#1}}\fp {#2}}
\newcommand{\spce}[2]{S_{\text{#1}}\fp {#2}}

\renewcommand{\O}[1]{O \fp{#1}}

\newcommand{\seq}[1]{\left \langle {#1} \right \rangle}
\renewcommand{\mlist}[1]{\left [ {#1} \right ]}

%% Formatting type ascription properly in text mode
\newcommand{\asc}[2]{{#1}\ :\ {#2}}


\begin{document}

\medskip

\thispagestyle{plain}
\begin{center}                  % Center the following lines
{\Large 15-150 Assignment \myhwnum} \\
\myname \\
\myandrew \\
G \\
2/28/12 \\
\end{center}


\question{1}{Task 2.2}
For all \emph{cs} :  $\mathtt{char list}$ and $k : \mathtt{char list} \rightarrow \mathtt{bool}$,
if $\mathtt{match Wild}$ \emph{cs k} $\cong \mathtt{true}$
then $\exists p, s$ such that $p@s \cong cs$ and $p \in L(\_)$ and \emph{k s} $\cong \mathtt{true}$.

\textbf{Proof:}\\
Assume \texttt{match Wild} \emph{cs k} $\cong$ \texttt{true}

Then
\begin{verbatim}
case Wild of 
  ... | Wild => case cs of 
                     [] => false 
                   | c'::cs' => k cs' 
      | ...
\end{verbatim}
must be true, so by inversion, 

\begin{verbatim}
cs \cong c'::cs', and k cs' = true
\end{verbatim}

Let $p = c'$
Let $s = cs'$

Then we know 
\begin{verbatim}
cs = c'::cs' = p@s
\end{verbatim}

$p \in L(Wild)$ because $p = c'$ and c' is a character since it is an elem of a char list.

\sml{k cs' $\cong$ k s $\cong$ true} as shown above.


\question{2}{Task 2.4}
Assume \texttt{match Both(r1,r2)} \emph{cs k} $\cong$ \texttt{true}
Then
\begin{verbatim}
case Both(r1,r2) of
  ... | Both (r1,r2) => match r1 cs (fn cs' => match r2 cs (fn cs'' =>
          charlisteq(cs',cs'') andalso (k cs'') )) 
      | ...
\end{verbatim}
must be true, so by inversion,
\begin{verbatim}
match r1 cs (fn cs' => match r2 cs (fn cs'' => charlisteq(cs',cs'') andalso (k cs'') )) 
\end{verbatim}
must be true.

Invoke the IH for \sml{r1} where \sml{k $\cong$ (fn cs' => match r2 cs (fn cs'' => charlisteq(cs',cs'') andalso (k cs'') ))} 

Therefore, there exists a p1 and s1 such that p $\in L(r1)$ and \sml{k s1 $\cong$ true}, for the k above.
From the fact that \sml{k s1} is true:

\sml {k s1 $\cong$ true\\
$\cong$ (fn cs' => match r2 cs (fn cs'' => charlisteq(cs',cs'') andalso (k cs'') )) s1\\
$\cong$ match r2 cs (fn cs'' => charlisteq(s1,cs'') andalso (k cs'') )}


Now we can use the IH for r2 to say that there exists a p2 $\in L(r2)$ and an s2 such that k s2 $\cong$ true for the k above.

\sml{k s2 $\cong$ true\\
$\cong$(fn cs'' => charlisteq(s1,cs'') andalso (k cs'') ) s2\\
$\cong$charlisteq(s1,s2) andalso (k s2)}

Since this is true, we know that s1 and s2 are equal We can say that \sml{s1 = s2 = s}.

Also, \sml{k s2} is true, so \sml{k s} is true.

All that's left to show is that \sml{p@s $\cong$ cs}. 

We know that \sml{p1@s1 $\cong$ cs} and \sml{p2@s2 $\cong$ cs}

This means that \sml{p1@s $\cong$ p2@s $\cong$ cs}, and therefore \sml{p1 $\cong$ p2  $\cong$ p}.

We have proven that \sml{p@s  $\cong$ cs} and \sml{k s $\cong$ true}.
 

\question{3}{Task 2.6}
\textbf{Claim:} For all \sml{cs : char list}, if \sml{matchany cs $\cong$ true} then
$\exists$ \sml{p, s} such that \sml{p@s $\cong$ cs} with \sml{k s $\cong$ true}.

We proceed by structural induction on cs.

Case []:\\
Assume \sml{matchany [] $\cong$ true}

\sml{matchany [] $\cong$ true\\
$\cong$ (case [] of [] => k cs | x::xs => k cs orelse match r xs matchany)\\
$\cong$ k cs $\cong$ true\\}
This works when we let \sml{p = []} and \sml{s = cs}. Therefore the base case holds.

Case \sml{x::xs}:\\
Inductive hypothesis: The claim holds for xs.\\
Assume \sml{matchany x::xs $\cong$ true}\\

\sml{(case x::xs of [] => k cs | x::xs => k cs orelse match r xs matchany)\\
$\cong$ k cs orelse match r xs matchany\\}

By inversion on orelse, either \sml{k cs} is true or \sml{match r xs matchany} is true.
If \sml{k cs} is true, then let \sml{p = [], s = cs}, and you're done.

If \sml{match r xs matchany} is true, invoke the IH.
Let p = x and s = xs. Clearly, \sml{p@s $\cong$ cs}, and \sml{k s} by the IH. QED.

\end{document}

