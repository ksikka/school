\documentclass[11pt,letterpaper]{article}

\usepackage{amsmath}
\usepackage{amssymb}
\usepackage{fancyhdr}

\oddsidemargin0cm
\topmargin-2cm
\textwidth16.5cm
\textheight23.5cm

\newcommand{\question}[2] {\vspace{.25in} \hrule\vspace{0.5em}
\noindent{\bf #1: #2} \vspace{0.5em}
\hrule \vspace{.10in}}
\renewcommand{\part}[1] {\vspace{.10in} {\bf (#1)}}

\newcommand{\myname}{Karan Sikka}
\newcommand{\myandrew}{ksikka@andrew.cmu.edu}
\newcommand{\myhwnum}{10}

\setlength{\parindent}{0pt}
\setlength{\parskip}{5pt plus 1pt}

\pagestyle{fancyplain}
\lhead{\fancyplain{}{\textbf{HW\myhwnum}}}
\rhead{\fancyplain{}{\myname\\ \myandrew}}
\chead{\fancyplain{}{15-150}}
%% SML font
\newcommand{\sml}[1]{\texttt{#1}}

%% Evaluation and step judgements
\newcommand{\eval}[2]{{#1} \Longrightarrow {#2}}
\newcommand{\evalto}{\ensuremath{\Longrightarrow}}
\newcommand{\stepseq}[2]{{#1} \longmapsto {#2}}
\newcommand{\nstepseq}[2]{{#1} \longmapsto^* {#2}}

%% SML Syntax in math mode
\newcommand{\msml}[1]{\mathtt{#1}}

\newcommand{\mval}{\msml{v}}
\newcommand{\mvalp}{\msml{v'}}
\newcommand{\mvalpp}{\msml{v''}}
\newcommand{\mconst}{\msml{c}}
\newcommand{\mconsts}[1]{\msml{c_{#1}}}
\newcommand{\mvals}[1]{\msml{v_{#1}}}
\newcommand{\mcase}[5]{\msml{case}~{#1}~\msml{of}~{#2} => {#3}~|~{#4} => {#5}}
\newcommand{\mapp}[2]{{#1}~{#2}}
\newcommand{\mpair}[2]{({#1}, {#2})}
\newcommand{\mtriple}[3]{({#1}, {#2}, {#3})}
\newcommand{\mnil}{\msml{nil}}
\newcommand{\mcons}[2]{{#1}::{#2}}
\newcommand{\mnat}[1]{\msml{\overline{#1}}}
\newcommand{\mrtimes}[2]{{#1} ** {#2}}
\newcommand{\mrfrac}[2]{{#1} // {#2}}
\newcommand{\asc}[2]{{#1}\ :\ {#2}}
\newcommand{\smlarr}[2]{{#1}\ ->\ {#2}}
\newcommand{\econs}[2]{{#1}\ ::\ {#2}}
\newcommand{\tint}{\msml{int}}
\newcommand{\tstr}{\msml{string}}
\newcommand{\tchr}{\msml{char}}
\newcommand{\tbool}{\msml{bool}}
\newcommand{\tfso}{\msml{fsobject}}
\newcommand{\tlst}[1]{{#1}~\msml{list}}
\newcommand{\ttree}[1]{{#1}~\msml{tree}}
\newcommand{\tarr}[2]{{#1} \to {#2}}
\newcommand{\tprod}[2]{{#1} * {#2}}
\newcommand{\tnf}{\msml{nf}}
\newcommand{\oftp}[2]{{#1}\ :\ {#2}}

\newcommand{\oftpm}[2]{\msml{{#1}}\ :\ {#2}}
\newcommand{\ce}[2]{{#1} \cong {#2}}
\newcommand{\mappm}[2]{{\msml{#1}}~{\msml{#2}}}
\newcommand{\vnil}{\texttt{[]}}

%% Propositional calculus
\newcommand{\pand}{\wedge}
\newcommand{\por}{\vee}
\newcommand{\pimp}{\supset}



\begin{document}

\medskip

\thispagestyle{plain}
\begin{center}                  % Center the following lines
{\Large 15-150 Assignment \myhwnum} \\
\myname \\
\myandrew \\
G \\
\today \\
\end{center}


\question{2.1}{Pebbling 2DFS}

\begin{center}
\begin{tabular}{l|l|l}
  Time Step & Pebble 1 & Pebble 2\\
  \hline\hline
  1 & A& \\
  \hline
  2 & B& C\\
  \hline
  3 & D& E\\
  \hline
  4 & H& I\\
  \hline
  5 & K& L\\
  \hline
  6 & N& F\\
  \hline
  7 & Q& G\\
  \hline
  8 & J&  \\
  \hline
  9 & M&  \\
  \hline
 10 & O& P\\
  \hline
 11 & R&  \\
  \hline
 12 & S&  \\
\end{tabular}
\end{center}

\question{2.2}{Pebbling 2BFS}

\begin{center}
\begin{tabular}{l|l|l}
  Time Step & Pebble 1 & Pebble 2\\
  \hline\hline
  1 & A& \\
  \hline
  2 & B& C\\
  \hline
  3 & D& E\\
  \hline
  4 & F& G\\
  \hline
  5 & H& I\\
  \hline
  6 & J& K\\
  \hline
  7 & L& M\\
  \hline
  8 & N& O\\
  \hline
  9 & P& Q\\
  \hline
 10 & R&  \\
  \hline
 11 & S&  \\
\end{tabular}
\end{center}

\question{3.1}{}
Since \texttt{map} does stuff in parallel, the processor gets to decide 
the schedule in which it performs the operations. \texttt{incr} incorporates
the following operations: (1) getting the value stored in \texttt{terminals}
(2) adding 1 to the value and (3) storing the value back in the ref.

Say the ref has a value of 4, and two calls to \texttt{incr} need to be done.
The processor may choose to do the following:
\begin{enumerate}
\item get the value from terminals (4)
\item get the value from terminals (4)
\item increment 1 to the one of the values (4 to 5)
\item increment 1 to the other value (4 to 5)
\item store the value in the ref (4 to 5)
\item store the other value in the ref (5 to 5)
\end{enumerate}
At the end of these two parallel \texttt{incr} operations, the value of 
terminals has changed from 4 to 5. However, by the spec of incr,
the value should have changed from 4 to 5 to 6. This is why different schedules
will lead to different values for the integer.

\question{3.2}{}
Yes it is benign in sequential contexts, because at the beginning of each
sequential call, the value of 0 is stored in the ref, and during the call, only 
that call will affect the ref.

\question{3.3}{}
No, it is not benign in parallel contexts. Multiple calls to \texttt{next\_move} will alter
the same reference \texttt{terminals}. Therefore the result of the value 
in \texttt{terminals} is affected by the other instances of \texttt{next\_move}.

\question{3.4}{}
Yes it is benign in sequential contexts, because each call has it's own
ref, which is initialized to zero to start and is not affected by other
sequential calls to \texttt{next\_move}.

\question{3.5}{}
Yes it is benign in parallel contexts. Parallel calls to \texttt{next\_move} do not
affect each other since each one has it's own copy of the ref \texttt{terminals}.

\question{4.1}{}
\textbf{Lemma 3:}\\
For all \sml{l : 'a list}, \sml{r : 'a list}\\
\begin{center}\sml{revTwoPiles(l, r) = (rev l) @ r}\end{center}

The proof is by induction on \sml{l}:\\

\textbf{Base Case:}\\
\emph{\texttt{l $\cong$ []}}
\begin{align*}
      & \sml{revTwoPiles([], r)} &\\
\cong & \sml{case [] of [] => r | ...} & \text{step}\\ 
\cong & \sml{r} & \text{step}\\
\cong & \sml{[] @ r} & \text{Lemma 1}\\
\cong & \sml{(case [] of [] => [] | ...) @ r} & \text{rev step} \\
\cong & \sml{(rev l) @ [z]} & \text{rev step}
\end{align*}

\textbf{Inductive Hypothesis:}\\
\emph{\texttt{l $\cong$ x::xs}}\\
Assume \sml{revTwoPiles(xs, r) $\cong$ (rev xs) @ r}

\textbf{Inductive Step:}\\
\emph{\texttt{l $\cong$ x::xs}}\\
\emph{WTS: \sml{revTwoPiles(x::xs, r) $\cong$ (rev x::xs) @ r}}

\begin{align*}
      & \sml{revTwoPiles(x::xs, r)} & \\
\cong & \sml{case ... | (x::xs) => revTwoPiles(xs , x::r)} & \text{step}\\
\cong & \sml{revTwoPiles(xs , x::r)} & \text{step}\\
\cong & \sml{(rev xs) @ (x::r)} & \text{by the IH}\\
\cong & \sml{(rev xs) @ (x::([] @ r))} & \text{by Lemma 1}\\
\cong & \sml{(rev xs) @ (case x::[] of ... | x::[] => x::([] @ r))} & \text{rev step from @}\\
\cong & \sml{(rev xs) @ (x::[] @ r)} & \text{rev step from @}\\
\cong & \sml{(rev xs) @ ([x] @ r)} & \text{ x cons nil is [x]}\\
\cong & \sml{((rev xs) @ [x]) @ r} & \text{Lemma 1}\\
\cong & \sml{(case x::xs of ... | x :: xs => (rev xs) @ [x]) @ r)} & \text{rev step from rev}\\
\cong & \sml{(rev x::xs) @ r} & \text{rev step from rev}
\end{align*}

\question{4.2}{}
\textbf{Theorem 1:}
For all values \sml{l : 'a list}
\begin{center}\sml{rev l = rev2 l}\end{center}

\textbf{Proof:}\\
Consider:

\sml{rev2 l}\\
$\cong$\sml{let ... in revTwoPiles(l , []) end}\qquad\qquad\qquad step\\
$\cong$\sml{revTwoPiles(l, [])}\qquad\qquad\qquad\qquad\qquad\qquad\qquad step\\
$\cong$\sml{(rev l) @ []}\qquad\qquad\qquad\qquad\qquad\qquad\qquad Lemma 3\\
$\cong$\sml{rev l}\qquad\qquad\qquad\qquad\qquad\qquad\qquad\qquad Lemma 1

\question{5.3}{}
In the case that the key is not in the dictionary, PoorMemoizer will
use the function passed in to compute the result. However, the function
passed in makes recursive calls, those calls will not be memoized.
In the case of computing the nth fibonacci number where n is not
in the dictionary, the recursive function for fibonacci
will call f(n-1) and f(n-2), without checking if (n-2) or (n-1) are in the 
dictionary.

\question{5.5}{}
The effects will not occur if arguments passed to the function 
are already in the dictionary. If you try to memoize a function which utilizes 
the print function, effects will only occur if you pass in a key which is 
not already in the dictionary. If you pass a value which is already
in the dictionary (or if the function does so recursively), the memoized
function will recognize that and will immediately return the function's
previously returned value. In the case of the print function, the printing will
not occur but the function will still return \texttt{()}.
\end{document}
