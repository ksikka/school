\documentclass[11pt,letterpaper]{article}

\usepackage{amsmath}
\usepackage{amssymb}
\usepackage{fancyhdr}

\oddsidemargin0cm
\topmargin-2cm
\textwidth16.5cm
\textheight23.5cm

\newcommand{\question}[1] {\vspace{.25in} \hrule\vspace{0.5em}
\noindent{\bf #1} \vspace{0.5em}
\hrule \vspace{.10in}}
\renewcommand{\part}[1] {\vspace{.10in} {\bf (#1)}}

\newcommand{\myname}{Karan Sikka}
\newcommand{\myandrew}{ksikka@cmu.edu}
\newcommand{\myhwnum}{05}

\setlength{\parindent}{0pt}
\setlength{\parskip}{5pt plus 1pt}

\pagestyle{fancyplain}
\lhead{\fancyplain{}{\textbf{HW\myhwnum}}}
\rhead{\fancyplain{}{\myname\\ \myandrew}}
\chead{\fancyplain{}{21-301}}

\begin{document}

\medskip

\thispagestyle{plain}
\begin{center}                  % Center the following lines
{\Large 21-301 Assignment \myhwnum} \\
\myname \\
\myandrew \\
\today
\end{center}


\question{1}
We know that a maximal chain of $[n]$ contains $n+1$ sets, and there is exactly one set of every size from 0 to $n$.

We know that the sizes of $A_i$ strictly increase in the order they are given (from 1 to $k$). That is that $A_1$ is the smallest and $A_k$ is the largest $A_i$.

We can construct all maximal chains of $[n]$ containing all $A_i$ by the following method:

Let your chain contain all $A_i$. This is a chain given the subset relationship of all the $A_i$.
Notice that this chain may not be maximal if there is not a set of every size from 0 to $n$.
We know that we have sets of sizes $|A_1|, |A_2|, ..., |A_k|$.
\begin{enumerate}
\item Now, add a maximal chain of $2^{A_1}$.
\item Then find $|A_j| - |A_i|$ sets in $2^{[x]}$ of all sizes strictly between $(|A_i|, |A_j|)$ such that they form a chain, and add them to the chain.
\item Finally find $n - |A_k|$ sets in $2^{[x]}$ of all sizes strictly between $(|A_k|, n)$ such that they form a chain, and add them to the chain.
\end{enumerate}

Note that by this method, you can form any maximal chain of $2^{[x]}$ which contains all the $A_i$.
Now we count how many distinct outcomes there are of performing the above procedure.

\begin{enumerate}
\item There are $|A_1|!$ ways of doing step 1.
\item There are $\prod_{i,j | j-i = 1} (|A_j| - |A_i|)!$ ways of doing step 2. This is because you can choose from $(|A_j| - |A_i|)$ elements for the first set you add, one fewer element for the next set, up til there is only one way to form the last set (then you have $A_j$).
\item There are $(n - |A_k|)!$ ways of doing step 3 by the same logic.
\end{enumerate}

The product of those three quantities gives the answer:
$$|A_1|!(n - |A_k|)!\prod_{i,j | j-i = 1} (|A_j| - |A_i|)!$$

\question{2}
We observe that the size of intersection of $\mathcal{F}$ and any maximal chain of $2^{[n]}$,
is at most 2, since if there were 3 sets in common with the chain,
those 3 sets would form a chain which would break the constraint on $\mathcal{F}$ given in the problem.
With 2, you do not break the constraint.

We will use double-counting on the number of (maximal chain, set in $\mathcal{F}$) pairs such that the set is contained in the maximal chain.

An upper bound on this quantity is $2n!$, since there are $n!$ maximal chains, and at most 2 sets in $\mathcal{F}$ can be found in each of them.

We can count this quantity more precisely by counting the number of maximal chains containing $A$ for all $A \in \mathcal{F}$.
This quantity is $\sum_{A \in \mathcal{F}} |A|!(n-|A|)!$, as shown in the proof of Sperner's Theorem in lecture.

Then, we do some algebra:

$$ 1 \geq \sum_{A \in \mathcal{F}} \frac{|A|!(n-|A|)!}{2n!} = \sum_{A \in \mathcal{F}} \frac{1}{2{n \choose \lfloor \frac{n}{2} \rfloor }} = \frac{\mathcal{F}}{2{n \choose \lfloor \frac{n}{2} \rfloor }}$$

Which implies that $\mathcal{F} \geq 2{n \choose \lfloor \frac{n}{2} \rfloor }$ and since $n$ is even, $\mathcal{F} \geq 2{n \choose \frac{n}{2}}$.


\question{3}
Consider the $n$-length sequence where every element is 1.

\textbf{Proof:}

All numbers in the sequence are 1. Choose half of them to be multiplied by $\epsilon = -1$, and naturally the other half will be multiplied by $\epsilon = 1$.
Then the dot product of the epsilon vector and the sequence will be $0$. There are ${n \choose \frac{n}{2}}$ ways to do this.


\question{4}
Since every set has a nonempty intersection with every other set, we realize that there must be at least 1 element which must exist in the set, so it can intersect with another set in the family.
Then to construct an intersecting family, we have at most the freedom to decide for $n-1$ elements whether or not they are in a set in $\mathcal{F}$. In other words, at the very least, one element's fate is decided (it is in the set by constraint of interesting set).
Then the upper bound on the size of an intersecting family is:
$$ |\mathcal{F}| \leq 1 \prod_{i=2}^{n} 2 = 2^{n-1}$$

An intersecting family which matches this bound is (when $n=3$) $\lbrace \lbrace 1,2 \rbrace, \lbrace 2,3 \rbrace, \lbrace 2 \rbrace, \lbrace 1,2,3 \rbrace \rbrace$.

\question{5}

Consider a bipartite graph $(A,B)$ where $A$ is the set of $n$ elements and $B$ is the set of $S_1...S_n$. Notice that $|A| + |B| = 2n$.
An edge exists from a vertex in $A$ to $B$ if an element in $A$ is a member of a subset $S_i$ in $B$. Notice that the degree of a vertex representing $S_i$ in $B$ is equal to $|S_i|$, since it is the number of incoming edges, or number of elements which are members of $S_i$.

The problem states that two distinct $S_i, S_j$ may not have an intersection of more than one element.
In our bipartite graph, if $S_i, S_j$ in $B$ had an intersection of their neighbor sets of two vertexes in $A$, it would look like $K_{2,2}$,
and it would violate the problem statement. Therefore, our bipartite graph does not have $K_{2,2}$.

We know from lecture that the maximum number of edges in a bipartite graph that does not have $K_{2,2}$ is $|V|^{3/2} + |V|$. We'll write this in terms of $n$ and label constants for sake of clarity.

$$ |E| \leq \frac{1}{2}((2n)^{3/2} + 2n) = c_1 n^{3/2} + c_2n = n(c_3 \sqrt{n} + c_2) $$

We wish to find an upper bound on the degree of the smallest $S_i$ in $B$, because doing so would prove that an $S_i$ exists such that the claim in the problem is true.
To maximize the degree of the smallest $S_i$, we consider the graph with the maximum possible number of edges such that $|E| = n(c_3 \sqrt{n} + c_2)$, and we distribute the edges evenly accross vertexes in $B$.
In the above scenario, each vertex in $B$ would have degree $\frac{n(c_3 \sqrt{n} + c_2)}{n} = c_3 \sqrt{n} + c_2$. Thus we've shown that the degree of the minimum-degree vertex is upper-bounded by $c_3 \sqrt{n} + c_2$.

Therefore, in any scenario such as the one given in the question, there will always exist an $S_i$ such that $d(S_i) = |S_i| \leq c_3 \sqrt{n} + c_2$.
Since $n$ is a positive integer, we can do the following algebra:

$$|S_i| \leq c_3 \sqrt{n} + c_2 \leq c_3 \sqrt{n} + c_2 \sqrt{n} = c_4 \sqrt{n} = C \sqrt{n}$$

Thus, we have proven the claim.


\question{6}

Let S be the set containing all 3-tuples $(\lbrace u_1, u_2 \rbrace =,v)$ where $u_i$ are vertices in $A$ and $v$ is a vertex in $B$.
Then $|S| = \sum_{v \in A}{ d(v) \choose 2 }$

The rest is incomplete.

\end{document}

