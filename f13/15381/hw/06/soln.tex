\documentclass[11pt,letterpaper]{article}

\usepackage{amsmath}
\usepackage{amssymb}
\usepackage{fancyhdr}

\oddsidemargin0cm
\topmargin-2cm
\textwidth16.5cm
\textheight23.5cm

\newcommand{\question}[1] {\vspace{.25in} \hrule\vspace{0.5em}
\noindent{\bf #1} \vspace{0.5em}
\hrule \vspace{.10in}}
\renewcommand{\part}[1] {\vspace{.10in} {\bf (#1)}}

\newcommand{\myname}{Karan Sikka}
\newcommand{\myandrew}{ksikka@cmu.edu}
\newcommand{\myhwnum}{06}

\setlength{\parindent}{0pt}
\setlength{\parskip}{5pt plus 1pt}

\pagestyle{fancyplain}
\lhead{\fancyplain{}{\textbf{HW\myhwnum}}}
\rhead{\fancyplain{}{\myname\\ \myandrew}}
\chead{\fancyplain{}{15-381}}

\begin{document}

\medskip

\thispagestyle{plain}
\begin{center}                  % Center the following lines
{\Large 21-301 Assignment \myhwnum} \\
\myname \\
\myandrew \\
\today
\end{center}

\question{1.1}
Yes. Steal, Steal. If player 1 steals, player 2's optimum decision is to steal, and vice versa..

\question{1.2}
Yes. Steal, steal. We know the players both steal at the last iteration since there is no social punishment for doing so.
Given that knowledge, (Steal, Steal) is the subgame equilibrium for the subgame before that, and the one before that, and so on.
So by backwards induction, the players will (Steal, Steal) at every iteration.

\question{1.3}

Yes. Recall that by Nash's Existence Theorem, every finite game has at least one Nash equilibrium.
So you can find one for the last subgame. Then by backwards induction continue to find them all the way up the tree.
Therefore there must exist a subgame perfect Nash Equilibrium.

\question{2}

AFSOC:

$$\max_p \min_q M(p, q) > \min_q \max_p M(p, q)$$

Let's call the max p on the left hand side $p^*$. This is the row player's best guess of the maximizing strategy.
Let's call the max p on the right hand side $p^*(q)$. This is the best row maximizing strategy given any q. Then we get:

$$\min_q M(p^*, q) > \min_q M(p^*(q), q)$$

By the definition of $p^*(q)$, we know that for all p, q:

$$M(p^*(q), q) \geq M(p, q)$$

and therefore

$$M(p^*(q), q) \geq M(p^*, q)$$

but this implies

$$\min_q M(p^*, q) > \min_q M(p^*, q)$$

Which is a contradiction.

\end{document}

