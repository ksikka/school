\documentclass[11pt,letterpaper]{article}

\usepackage{amsmath}
\usepackage{amssymb}
\usepackage{fancyhdr}

\oddsidemargin0cm
\topmargin-2cm
\textwidth16.5cm
\textheight23.5cm

\newcommand{\question}[1] {\vspace{.25in} \hrule\vspace{0.5em}
\noindent{\bf #1} \vspace{0.5em}
\hrule \vspace{.10in}}
\renewcommand{\part}[1] {\vspace{.10in} {\bf (#1)}}

\newcommand{\myname}{Karan Sikka}
\newcommand{\myandrew}{ksikka@cmu.edu}
\newcommand{\myhwnum}{06}

\setlength{\parindent}{0pt}
\setlength{\parskip}{5pt plus 1pt}

\pagestyle{fancyplain}
\lhead{\fancyplain{}{\textbf{HW\myhwnum}}}
\rhead{\fancyplain{}{\myname\\ \myandrew}}
\chead{\fancyplain{}{21-301}}

\begin{document}

\medskip

\thispagestyle{plain}
\begin{center}                  % Center the following lines
{\Large 21-301 Assignment \myhwnum} \\
\myname \\
\myandrew \\
\today
\end{center}

\question{1}
We will construct a sequence of $kl$ by constructing $l$ increasing sequences each of length $k$.

Starting from some starting number, which we will call $a$, consider the increasing sequence $\langle a, a+1, ... , a + k \rangle$.\\
Consider another similarly constructed sequence, but instead start from $a - 1 - k $\\
Do it again, but instead start from $a - 1 - 2k $\\
Repeat until you construct a sequence starting from $a - 1 - lk $. Note that we constructed $l$ sequences of length $k$.

The longest increasing subsequence is an individual increasing sequence of length $k$,
and the longest decreasing subsequence consists of one from every such sequence of which there are $l$.


\question{2}
At most, a Hasse diagram of a poset can have $\alpha * \alpha$ edges between two levels.
The maximum number of times this can occur is equal to the height $\omega$.
This is because you can't have edges which cross levels, since that would break the property that edges must be predecessors.

The addition of a vertex in order to increase the height only increases the number of edges by the width. 
The addition of a vertex in order to increase the width increases the number of edges by the width, and also increases the width for future vertexes.

Therefore, given $n$ vertices, it always produces more edges by having greater width than by having greater height.

Therefore we want to maximize the width and minimize the height.
The minimum height is 1. Visually, this poset looks like a bipartite graph. To maximize the number of edges, we wish the bipartitions to be of equal size.
Therefore, each bipartition is of size $\frac{n}{2}$, and the maximum number of edges occurs in the complete graph case where every node
in $A$ is connected to every node in $B$. The maximum number of edges in this graph is $$\frac{n}{2} \frac{n}{2} = \frac{n^2}{4}$$.

To see an example, we define the poset where $X$ is $n/2$ distinct sets of size 1, and $n/2$ distinct sets of size 2.
We define the relation to be $xRy$ iff $|x| < |y| \vee x = y$. We see that it is reflexive, antisymmetric, and transitive.
The Hasse diagram resembles a complete bipartite graph such as the one in the proof,
since all the sets of size 1 are predecessors of all the sets of size 2.

\question{3}
Given some $k$, consider $R(k,k) = N$. This means that in $K_{N}$ with edges colored either red or blue,
there exists a $K_k$ where all edges are red or blue.

Now consider a sequence of $R(k,k)$ distinct real numbers and construct a complete graph from them.
Color the edge between $(v_i, v_j)$ blue iff $(i < j) \wedge (v_i < v_j)$.
Then color the uncolored edges red.

Given Ramsey's theorem, we know there must exist a red or blue colored $K_k$ in this graph.
If blue, we know there exists a strictly decreasing subsequence of length $k$ given the definition of our coloring.
If red, we know that all $(v_i, v_j)$ in the red clique do not satisfy $v_i < v_j$,
and since the real numbers are distinct, we must have that $v_j > v_j$ for all $i < j$.
This implies there is a strictly increasing subsequence of length $k$.

Combined, we see that there must exist a monotone sequence of length $k$
in a sequence of length $R(k, k)$, which concludes the proof.

\question{4}
Let $n = R(k-1, l) + R(k, l-1)$. Consider a complete graph with $n$ vertices.
We know each vertex is connected to every other vertex, so any vertex $v$ has $n-1$ neighbors,
and therefore $n-1$ incident edges. If all $n-1$ have to be colored red or blue,
then $R(k-1, l)$ of the edges may be colored red.

If not, there must be at least $R(k, l-1)$ blue edges, since we must maintain that there are $n-1$ edges.

Case $R(k-1, l)$ red edges:\\
Consider the $R(k-1, l)$ vertexes connected to $v$ by red edges. They are all connected to each other since
this is a complete graph, and therefore form a sub-complete-graph, which we know must contain an all-red $K_{k-1}$ or an all-blue $K_{l}$.
In the all-red case, we connect that clique with $v$, and this forms a $K_{k}$. Then $n$ is at least $R(k, l)$.

The proof is symmetric for the blue case.

Therefore, $R(k, l) \leq R(k-1, l) + R(k, l-1)$

\question{5}
For any integers $k, l, m > 2$, consider $N = R(R(k,l), m)$.

Consider a complete graph with $N$ vertexes.

By Ramsey's theorem we know there exists either a monochromatic $K_{R(k,l)}$ or a monochromatic $K_{m}$.

Within the monochromatic $K_{R(k,l)}$, we know there must exist a monochromatic $K_{k}$ or $K_{l}$.

Therefore, there is an integer $N$ such that there must be a monochromatic $K_k$, $K_l$, or $K_m$.

\question{6}
We first try to use the inequality from (4):

$R(3, 4) \leq R(k-1, l) + R(k, l-1) = R(2, 4) + R(3, 3) = 4 + 6$


We notice that the terms in the sum are even, and we recall that in lecture, a tighter inequality was mentioned:

We seek to prove that for integers $k, l > 2$, when $R(k-1, l)$, $R(k, l-1)$ are even:

$R(k, l) \leq R(k-1, l) + R(k, l-1) - 1$

(Begin attempt to prove:)

$a = R(k-1, l)   (even)$\\
$b = R(k, l-1)   (even)$\\
$c = a + b       (even)$\\

$X = [c-1]       (odd)$\\

We need to prove that X contains a blue $K_k$ or a red $K_l$.

Let all blue edges denote people know each other, red that they don't.

Let $d_v$ be the number of people that person $v$ knows (number of outgoing blue edges).

The sum of all $d_v$ is even since "knowing" person is mutual, and the sum of the $d_v$ over count $a \implies b$ and $b \implies a$ as distinct.

Since the number of people is odd, then $d_v$ must be even for at least one person.

(End attempt)

I could not prove it in time, but assuming that the inequality is true, you get the following result:

$$R(3, 4) \leq 4 + 6 - 1 = 9$$

Note that we proved the lower bound $R(3,4) \leq 9$ in class.




\end{document}

