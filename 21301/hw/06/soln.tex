\documentclass[11pt,letterpaper]{article}

\usepackage{amsmath}
\usepackage{amssymb}
\usepackage{fancyhdr}

\oddsidemargin0cm
\topmargin-2cm
\textwidth16.5cm
\textheight23.5cm

\newcommand{\question}[1] {\vspace{.25in} \hrule\vspace{0.5em}
\noindent{\bf #1} \vspace{0.5em}
\hrule \vspace{.10in}}
\renewcommand{\part}[1] {\vspace{.10in} {\bf (#1)}}

\newcommand{\myname}{Karan Sikka}
\newcommand{\myandrew}{ksikka@cmu.edu}
\newcommand{\myhwnum}{06}

\setlength{\parindent}{0pt}
\setlength{\parskip}{5pt plus 1pt}

\pagestyle{fancyplain}
\lhead{\fancyplain{}{\textbf{HW\myhwnum}}}
\rhead{\fancyplain{}{\myname\\ \myandrew}}
\chead{\fancyplain{}{21-301}}

\begin{document}

\medskip

\thispagestyle{plain}
\begin{center}                  % Center the following lines
{\Large 21-301 Assignment \myhwnum} \\
\myname \\
\myandrew \\
\today
\end{center}

\question{1}
We will construct a sequence of $kl$ by constructing $l$ increasing sequences each of length $k$.

Starting from some starting number, which we will call $a$, consider the increasing sequence $\langle a, a+1, ... , a + k \rangle$.\\
Consider another similarly constructed sequence, but instead start from $a - 1 - k $\\
Do it again, but instead start from $a - 1 - 2k $\\
Repeat until you construct a sequence starting from $a - 1 - lk $. Note that we constructed $l$ sequences of length $k$.

The longest increasing subsequence is an individual increasing sequence of length $k$,
and the longest decreasing subsequence consists of one from every such sequence of which there are $l$.


\question{2}
At most, a Hasse diagram of a poset can have $\alpha * \alpha$ edges between two levels,
and the number of times this can occur is equal to the height $\gamma$. This is because
you can't have edges which cross levels, since that would break the property that edges must be predecessors.

Then we invoke the Erdos-Szekeres theorem to get the following intermediate result:

$$ |E| \leq \gamma \alpha \alpha \leq n \alpha$$

We see that the maximum number of edges depends on the $n$ and width of the poset.
We know that a large poset is either tall or wide, and to maximize number of edges, this equation tells us it must be as wide as possible.
Therefore, we know that a poset of maximum size should have height of 1.

Visually, this poset looks like a bipartite graph. To maximize the number of edges, we wish the bipartitions to be of equal size.
Therefore, each bipartition is of size $\frac{n}{2}$, and the maximum number of edges occurs in the complete graph case where every node
in $A$ is connected to every node in $B$. The maximum number of edges in this graph is $$\frac{n}{2} \frac{n}{2} = \frac{n^2}{4}$$.

To see an example, we define the poset where $X$ is $n/2$ distinct sets of size 1, and $n/2$ distinct sets of size 2.
We define the relation to be $xRy$ iff $|x| < |y| \vee x = y$. We see that it is reflexive, antisymmetric, and transitive.
The Hasse diagram resembles a complete bipartite graph such as the one in the proof,
since all the sets of size 1 are predecessors of all the sets of size 2.

\question{3}

\question{4}

\question{5}

\question{6}


\end{document}

