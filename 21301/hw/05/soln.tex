\documentclass[11pt,letterpaper]{article}

\usepackage{amsmath}
\usepackage{amssymb}
\usepackage{fancyhdr}

\oddsidemargin0cm
\topmargin-2cm
\textwidth16.5cm
\textheight23.5cm

\newcommand{\question}[1] {\vspace{.25in} \hrule\vspace{0.5em}
\noindent{\bf #1} \vspace{0.5em}
\hrule \vspace{.10in}}
\renewcommand{\part}[1] {\vspace{.10in} {\bf (#1)}}

\newcommand{\myname}{Karan Sikka}
\newcommand{\myandrew}{ksikka@cmu.edu}
\newcommand{\myhwnum}{05}

\setlength{\parindent}{0pt}
\setlength{\parskip}{5pt plus 1pt}

\pagestyle{fancyplain}
\lhead{\fancyplain{}{\textbf{HW\myhwnum}}}
\rhead{\fancyplain{}{\myname\\ \myandrew}}
\chead{\fancyplain{}{21-301}}

\begin{document}

\medskip

\thispagestyle{plain}
\begin{center}                  % Center the following lines
{\Large 21-301 Assignment \myhwnum} \\
\myname \\
\myandrew \\
\today
\end{center}


\question{1}
We know that a maximal chain of $[n]$ contains $n+1$ sets, and there is exactly one set of every size from 0 to $n$.

We know that the sizes of $A_i$ strictly increase in the order they are given (from 1 to $k$). That is that $A_1$ is the smallest and $A_k$ is the largest $A_i$.

We can construct all maximal chains of $[n]$ containing all $A_i$ by the following method:

Let your chain contain all $A_i$. This is a chain given the subset relationship of all the $A_i$.
Notice that this chain may not be maximal if there is not a set of every size from 0 to $n$.
We know that we have sets of sizes $|A_1|, |A_2|, ..., |A_k|$.
\begin{enumerate}
\item Now, add a maximal chain of $2^{A_1}$.
\item Then find $|A_j| - |A_i|$ sets in $2^{[x]}$ of all sizes strictly between $(|A_i|, |A_j|)$ such that they form a chain, and add them to the chain.
\item Finally find $n - |A_k|$ sets in $2^{[x]}$ of all sizes strictly between $(|A_k|, n)$ such that they form a chain, and add them to the chain.
\end{enumerate}

Note that by this method, you can form any maximal chain of $2^{[x]}$ which contains all the $A_i$.
Now we count how many distinct outcomes there are of performing the above procedure.

\begin{enumerate}
\item There are $|A_1|!$ ways of doing step 1.
\item There are $\prod_{i,j | j-i = 1} (|A_j| - |A_i|)!$ ways of doing step 2. This is because you can choose from $(|A_j| - |A_i|)$ elements for the first set you add, one fewer element for the next set, up til there is only one way to form the last set (then you have $A_j$).
\item There are $(n - |A_k|)!$ ways of doing step 3 by the same logic.
\end{enumerate}

The product of those three quantities gives the answer:
$$|A_1|!\prod_{i,j | j-i = 1} (|A_j| - |A_i|)!(n - |A_k|)!$$

\question{2}
We observe that the size of intersection of $\mathcal{F}$ and any maximal chain of $2^{[n]}$,
is at most 2, since if there were 3 sets in common with the chain,
those 3 sets would form a chain which would break the constraint on $\mathcal{F}$ given in the problem.
With 2, you do not break the constraint.

The total number of sets in $\mathcal{F}$ that also exist in some maximal chain is at most $2n!$ (two in a maximal chain, and there are $n!$).

We can also count this quantity by counting the number of maximal chains containing $A$ for all $A \in \mathcal{F}$. This quantity is $\sum_{A \in \mathcal{F}} |A|!(n-|A|)!$, as shown in the proof of Sperner's Theorem in lecture.

Then, we do some algebra:

$$ 1 \geq \sum_{A \in \mathcal{F}} \frac{|A|!(n-|A|)!}{2n!} = \sum_{A \in \mathcal{F}} \frac{1}{2{n \choose \lfloor \frac{n}{2} \rfloor }} = \frac{\mathcal{F}}{2{n \choose \lfloor \frac{n}{2} \rfloor }}$$

Which implies that $\mathcal{F} \geq 2{n \choose \lfloor \frac{n}{2} \rfloor }$ and since $n$ is even, $\mathcal{F} \geq 2{n \choose \frac{n}{2}}$.


\question{3}
Consider the $n$-length sequence where if $n$ is even, every element is 1, but if $n$ is odd, $n-2$ elements are 1, 1 element is -1, and 1 element is 1.5.

\textbf{Proof:}

Case $n$ is even:\\
All numbers in the sequence are 1. Choose half of them to get $\echelon = -1$, and naturally the other half get $\echelon = 1$.
Then the dot product of the echelon vector and the sequence will be $0$. There are ${n \choose \frac{n}{2}}$ ways to do this.

Case $n$ is odd:\\
Put the $n-2$ 1s and the $-1$ in a bag. Note that this forms $n-1$ elements, which is even since $n$ is odd.
If you choose half of them to get $\echelon = -1$ and the other half to get $\echelon = 1$, the sum will either be 1 or -1.
If the sum is 1, choose the 1.5 to be multipled by $\echelon = -1$. Else, $\echelon = 1$. In the first case, the sum will be $1-1.5 = -0.5$, and in the second case, the sum will be $-1+1.5 = 0.5$.
There are ${n \choose \frac{n-1}{2}}$ ways to construct this $\echelon$ vector, and since $n$ is odd, this is equal to ${n \choose \lfloor \frac{n}{2} \rfloor}$ ways.


\question{4}
Since every set has a nonempty intersection with every other set, we realize that there must be at least 1 element which is common to all sets in $\mathcal{F}$.
Then to construct an intersecting family, we have at most the freedom to decide for $n-1$ elements whether or not they are in a set in $\mathcal{F}$. In other words, at the very least, one element's fate is decided (it is in every set by default).
Then the upper bound on the size of an intersecting family is:
$$ |\mathcal{F}| \leq 1 \prod_{i=2}^{n} 2 = 2^{n-1}$$

\question{5}

\question{6}

\question{7}


\end{document}

