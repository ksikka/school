\documentclass[11pt,letterpaper]{article}

\usepackage{amsmath}
\usepackage{amssymb}
\usepackage{fancyhdr}

\oddsidemargin0cm
\topmargin-2cm
\textwidth16.5cm
\textheight23.5cm

\newcommand{\question}[1] {\vspace{.25in} \hrule\vspace{0.5em}
\noindent{\bf #1} \vspace{0.5em}
\hrule \vspace{.10in}}
\renewcommand{\part}[1] {\vspace{.10in} {\bf (#1)}}

\newcommand{\myname}{Karan Sikka}
\newcommand{\myandrew}{ksikka@cmu.edu}
\newcommand{\myhwnum}{04}

\setlength{\parindent}{0pt}
\setlength{\parskip}{5pt plus 1pt}

\pagestyle{fancyplain}
\lhead{\fancyplain{}{\textbf{HW\myhwnum}}}
\rhead{\fancyplain{}{\myname\\ \myandrew}}
\chead{\fancyplain{}{21-301}}

\begin{document}

\medskip

\thispagestyle{plain}
\begin{center}                  % Center the following lines
{\Large 21-301 Assignment \myhwnum} \\
\myname \\
\myandrew \\
\today
\end{center}


\question{1}
Consider a d-regular bipartite graph.

Find it's perfect matching
Take it out. Now we have a $d-1$-regular bipartite graph. Claim holds for all $d$ by induction.

\question{2}

\question{3}
First note that a graph with $|E| = m$ and $|V| = n$ where $m = {n choose 2} = n*(n - 1)/2$ is maximally connected, and therefore a connected graph with $n$ vertices has at most $m$ edges.

If you then add a vertex to a maximally connected graph, you get a disconnected graph with the largest possible number of edges: $m = {n-1 \ choose 2} = (n-1)(n-2)/2$.
This is because if you add a single edge to the graph, it's connected again, since it must connect to the new vertex (all other possible edges already exist).

Therefore, any graph with $m > (n-1)(n-2)/2$ must be connected.

\question{4}
Assume for sake of contradiction that there exists a tree $T$ where two paths of max-length did not have any vertices in common.

We know that the tree is connected.
Therefore there must exist a path from a vertex in one of the max-length paths to a vertex in the other.
However, then we've found a longer path than either of the two that first assumed were max-length.

This is a contradiction and therefore there does not exists a tree satisfying the assumed condition and in any tree, two max-length paths must share at least one vertex.

\question{5}
Call a tree with exactly one vertex of degree $i$ for any $2 \leq i \leq k$ for some $k > 0$ a $k$-weird tree.
We denote the number of edges in a $k$-weird tree as $S_k$.

$S_1 = 1$. This is because if $k=1$, then the tree has no vertices of degree 2 or greater, and the only such tree is the tree with two vertices and one edge.

$S_2 = 3$. You can see this from the tree with three vertices, where one is the root and two are leaves.

For $k>1$, we can form a $k$-weird tree using the following inductive approach. Create a vertex with degree $k$ by connecting it to $k-1$ leaves and a root vertex of a $(k-1)$-weird subtree.
This tree is indeed $k$-weird since it has exactly one vertex of degree $k$, and it has a $k-1$ subtree which shows the weird property holds for all values $k-1$ down till 2.

In this case $S_k = 1 + (k-1) + S_{k-1}$. This is because there is one $k$-degree vertex, $k-1$ of its leaves, and a $k-1$-weird subtree.

$S_k = 1 + (k-1) + S_{k-1}$\\
$    = k + S_{k-1}$\\
$    = k + (k-1) + ... + S_3$\\
$    = k + (k-1) + ... + 3 + S_2$\\
$    = k + (k-1) + ... + 3 + 3$\\
$    = \frac{(k-2)(k+3)}{2} + 3$\\

We've determined the values of $S_k$ for $k=1$, $k=2$, and arbitrary $k$.

\question{6}
For any vertex $v$ of degree $d$ in a tree where $d \geq 3$, there are $d$ neighbors, by definition of degree.
At most one of these neighbors is a parent vertex (not a leaf). The other $d-1$ neighbors must either be leaves or the roots of disjoint subtrees.
In the subtree case, you can apply this argument inductively, to show the number of leaves will always be greater than the number of vertexes of degree at least 3.
(And as a formality, you see the base case holds with 0 vertexes and 1 vertex satisfying this condition.)


\question{7}
(this was shown in class)

The proof is by induction on the number of vertices.

In the base case $|V| = 1$ the claim obviously holds.

Consider a graph $G$ with some $k \geq 1$ vertices which has no cycle and satisfies $|V| = |E| + 1$.
We assume that this is a tree.

If you add a vertex to the graph, the graph is disconnected and is no longer a tree. We seek to add edges in order to make it a tree.

You must add an edge to make the graph connected. You can add an edge in such a way that it does not create
a cycle, by simply extending off of a leaf. However, you cannot add more than that because it would have to connect
to another node in $G$ which was already maximal (given that it was a tree), and it would create a cycle.

Therefore, if you add 1 vertex, you must add exactly 1 edge to create a tree from a tree.

$|V'| = |V| + 1 = |E| + 1 + 1 = |E'| + 1$

and more succinctly

$|V'| = |E'| + 1$


\end{document}

