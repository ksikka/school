\documentclass[11pt,letterpaper]{article}

\usepackage{amsmath}
\usepackage{amssymb}
\usepackage{fancyhdr}

\oddsidemargin0cm
\topmargin-2cm
\textwidth16.5cm
\textheight23.5cm

\newcommand{\question}[1] {\vspace{.25in} \hrule\vspace{0.5em}
\noindent{\bf #1} \vspace{0.5em}
\hrule \vspace{.10in}}
\renewcommand{\part}[1] {\vspace{.10in} {\bf (#1)}}

\newcommand{\myname}{Karan Sikka}
\newcommand{\myandrew}{ksikka@cmu.edu}
\newcommand{\myhwnum}{04}

\setlength{\parindent}{0pt}
\setlength{\parskip}{5pt plus 1pt}

\pagestyle{fancyplain}
\lhead{\fancyplain{}{\textbf{HW\myhwnum}}}
\rhead{\fancyplain{}{\myname\\ \myandrew}}
\chead{\fancyplain{}{21-301}}

\begin{document}

\medskip

\thispagestyle{plain}
\begin{center}                  % Center the following lines
{\Large 21-301 Assignment \myhwnum} \\
\myname \\
\myandrew \\
\today
\end{center}


\question{1}

\question{2}

\question{3}

\question{4}

\question{5}
First note that a graph with $|E| = m$ and $|V| = n$ where $m = {n choose 2} = n*(n - 1)/2$ is maximally connected, and therefore a connected graph with $n$ vertices has at most $m$ edges.

If you then add a vertex to a maximally connected graph, you get a disconnected graph with the largest possible number of edges: $m = {n-1 \ choose 2} = (n-1)(n-2)/2$.
This is because if you add a single edge to the graph, it's connected again, since it must connect to the new vertex (all other possible edges already exist).

Therefore, any graph with $m > (n-1)(n-2)/2$ must be connected.

\question{6}
For any vertex $v$ of degree $d$ in a tree where $d \geq 3$, there are $d$ neighbors, by definition of degree.
At most one of these neighbors is a parent vertex (not a leaf). The other $d-1$ neighbors must either be leaves or the roots of disjoint subtrees.
In the subtree case, you can apply this argument inductively, to show the number of leaves will always be greater than the number of vertexes of degree at least 3.
(And as a formality, you see the base case holds with 0 vertexes and 1 vertex satisfying this condition.)


\question{7}
(this was shown in class)

The proof is by induction on the number of vertices.

In the base case $|V| = 1$ the claim obviously holds.

Consider a graph $G$ with some $k \geq 1$ vertices which has no cycle and satisfies $|V| = |E| + 1$.
We assume that this is a tree.

If you add a vertex to the graph, the graph is disconnected and is no longer a tree. We seek to add edges in order to make it a tree.

You must add an edge to make the graph connected. You can add an edge in such a way that it does not create
a cycle, by simply extending off of a leaf. However, you cannot add more than that because it would have to connect
to another node in $G$ which was already maximal (given that it was a tree), and it would create a cycle.

Therefore, if you add 1 vertex, you must add exactly 1 edge to create a tree from a tree.

$|V'| = |V| + 1 = |E| + 1 + 1 = |E'| + 1$

and more succinctly

$|V'| = |E'| + 1$


\end{document}

