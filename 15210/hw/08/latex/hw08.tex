\documentclass[11pt,letterpaper]{article}

\usepackage{amsmath}
\usepackage{amssymb}
\usepackage{fancyhdr}

\oddsidemargin0cm
\topmargin-2cm
\textwidth16.5cm
\textheight23.5cm

\newcommand{\question}[2] {\vspace{.25in} \hrule\vspace{0.5em}
\noindent{\bf #1: #2} \vspace{0.5em}
\hrule \vspace{.10in}}
\renewcommand{\part}[1] {\vspace{.10in} {\bf (#1)}}

\newcommand{\myname}{Karan Sikka}
\newcommand{\myandrew}{ksikka@cmu.edu}
\newcommand{\myhwnum}{08}

\setlength{\parindent}{0pt}
\setlength{\parskip}{5pt plus 1pt}

\pagestyle{fancyplain}
\lhead{\fancyplain{}{\textbf{HW\myhwnum}}}
\rhead{\fancyplain{}{\myname\\ \myandrew}}
\chead{\fancyplain{}{15-210}}

\begin{document}

\medskip

\thispagestyle{plain}
\begin{center}                  % Center the following lines
{\Large 15-210 Assignment \myhwnum} \\
\myname \\
\myandrew \\
Section C\\
\today\\
\end{center}


\question{1}{Task 3.2}
Instead of using iter, I would use scan.
To make the function associative, I would use Table.merge
instead of Table.insert. This is log squared span because
merge is log span and scan is log span.
The work increases to n squared: n for scan and n for merge.

\question{2}{Task 3.3}
The data structure is $O(n)$ in space complexity, since
the x-keyed bst has n nodes and at each iteration of iter,
a constant number of nodes is added to the search tree.
Because of persistence in functional programming, the whole
tree doesn't have to be ``copied'' at each iteration of iter.
$O(kn) = O(n)$. 

\end{document}

