\documentclass[11pt]{article}
\usepackage{enumerate}
\usepackage{fullpage}
\usepackage{fancyhdr}
\usepackage{amsmath, amsfonts, amsthm, amssymb}
\setlength{\parindent}{0pt}
\setlength{\parskip}{5pt plus 1pt}
\pagestyle{empty}

\def\indented#1{\list{}{}\item[]}
\let\indented=\endlist

\newcounter{questionCounter}
\newcounter{partCounter}[questionCounter]
\newenvironment{question}[2][\arabic{questionCounter}]{%
    \setcounter{partCounter}{0}%
    \vspace{.25in} \hrule \vspace{0.5em}%
        \noindent{\bf #2}%
    \vspace{0.8em} \hrule \vspace{.10in}%
    \addtocounter{questionCounter}{1}%
}{}
\renewenvironment{part}[1][\alph{partCounter}]{%
    \addtocounter{partCounter}{1}%
    \vspace{.10in}%
    \begin{indented}%
       {\bf (#1)} %
}{\end{indented}}
\newcommand{\myname}{Karan Sikka}
\newcommand{\myandrew}{ksikka@andrew.cmu.edu}
\newcommand{\myhwname}{Homework 5}
\newcommand{\myclass}{Physics of Musical Sound}
\begin{document}
\thispagestyle{plain}
\begin{center}
{\Large \myclass -- \myhwname} \\
\myname \\
\myandrew \\
\today
\end{center}
\begin{question}{Problem 1}
\begin{enumerate}[a)]
\item
E$_1$ is a fifth above A$_0$.\\
$f(E_1) = (2^{7/12}) * 2^{-4}(440\text{ Hz})=41.2\text{ Hz}$
\item
F$^{\sharp}_{3}$ is a major sixth above A$_2$\\
$f(F_3^{\sharp}) = (2^{9/12}) * 2^{-2}(440\text{ Hz})=185.0\text{ Hz}$
\item
G$_{7}$ is a minor seventh above A$_6$\\
$f(G_7) = (2^{10/12}) * 2^{2}(440\text{ Hz})=3136.0\text{ Hz}$
\item
D$_{4}^{\flat}$ is a major third above A$_3$\\
$f(D_4^{\flat}) = (2^{4/12}) * 2^{-1}(440\text{ Hz})=277.2\text{ Hz}$
\end{enumerate}
\end{question}

\begin{question}{Problem 2}
\begin{enumerate}[a)]
\item This is a minor third since it spans E,F,G and it's  3 semitones.\\
Just ratio: $6/5 = 1.2$\\
Just cents: $316$\\
Equal temp ratio: $2^{1/4} = 1.189$\\
Equal temp cents: $300$
\item This is a major sixth since it spans G,A,B,C,D,E and it's 9 semitones.\\
Just ratio: $5/3 = 1.667$\\
Just cents: $884$\\
Equal temp ratio: $2^{9/12} = 1.682$\\
Equal temp cents: $900$
\item This is a major third since it spans D,E,F and it's 4 semitones.\\
Just ratio: $15/8 = 1.875$\\
Just cents: $1088$\\
Equal temp ratio: $2^{11/12} = 1.888$\\
Equal temp cents: $1100$
\item This is a major seventh since it spans D,E,F,G,A,B,C and it's 11 semitones.\\
Just ratio: $5/4 = 1.250$\\
Just cents: $386$\\
Equal temp ratio: $2^{1/3}$\\
Equal temp cents: $400$
\end{enumerate}
\end{question}
\begin{question}{Problem 3}
An octave above E$_4$ is E$_5$.\\
In just intonation, the ratio is 2 and the interval is 1200 cents.\\
In equal temperament, the ratio is 2 and the interval is 1200 cents.

A perfect fifth above B$_3$ is F$_4^{\sharp}$\\
In just intonation, the ratio is $3/2 = 1.5$ and the interval is 702 cents.\\
In equal temperament, the ratio is $2^{7/12} = 1.498$ and the interval is 700 cents.

A minor seventh below B$_4^{\flat}$ is C$_4$.\\
In just intonation, the ratio is $9/5 = 1.8$ and the interval is 1018 cents.\\
In equal temperament, the ratio is $2^{10/12} = 1.782$ and the interval is 1000 cents.
\end{question}
\begin{question}{Problem 5}
D$_2$, D$_3$, D$_4$, D$_5$, because the 1st, 2nd, 4th, and 8th harmonics are octaves apart.
The 3rd harmonic is A$_3$ because $\frac{3}{2}D_3$ is a fifth above D$_3$.
Therefore the 6th and 12th harmonics are A$_4$ and A$_5$.
The 5th harmonic is F$_4^{\sharp}$ because $\frac{5}{4}D_3$ is a major third above D$_4$,
    the fourth harmonic.
The 10th harmonic is easily derived from this fact to be F$_5^{\sharp}$.
The 9th harmonic is 9/8 times the 8th harmonic, and 9/8 is a major second. Therefore
the 9th harmonic is a major second above D$_5$, which is E$_5$.

Thus we have all of he harmonics except the 7th and the 11th. Looking at the table
in the back of the book, we can see that D$_2$ has a frequency of 73.42 Hz. The 
7th harmonic has a frequency of $7*73.42 = 513.9$ Hz and the 11th harmonic has a 
frequency of $11*73.42 = 807.6$ Hz. Matching this up with the notes in the back
of the book, we see that the 7th harmonic is approximately C$_5$ and the 11th
harmonic is approximately G$_5$.

The notes of the harmonic series are as follows:\\
D$_2$, D$_3$, A$_3$, D$_4$, F$_4^\sharp$, A$_4$, C$_5$, D$_5$, E$_5$, F$_5^\sharp$, G$_5$, A$_5$
\end{question}
\begin{question}{Problem 5}
To get from the fourth harmonic to the fifth, you multiply by 5/4ths, which tells us
that the fifth harmonic is a major third above the fourth harmonic. Therefore, 
the fourth harmonic is a major third below A$_5$, which is F$_5$. The first 
harmonic is 2 octaves lower, which is F$_3$.

The ninth harmonic is 9/8 times the eigth harmonic, so the eigth harmonic is a 
major second below the ninth. A major second below B$_6^{\flat}$ is A$_6^{\flat}$.
The first harmonic is 3 octaves below the eigth, so it is A$_3^{\flat}$.
\end{question}
\begin{question}{Problem 6}
\begin{enumerate}[a)]
\item
\begin{center}
\begin{tabular}{c || c | c | c | c | c}
Construction & Semitones & Interval & Note in C & Ratio & Cents\\
\hline
2$\cdot$M3 + P5$- 8^{ve}$                  & 3  & m3 & D$^{\sharp}$ & 75/64  & 275 \\
$2\cdot 8^{ve}-\text{P}5-2\cdot \text{M}3$ & 9  & M6 & A            & 125/75 & 925 \\
$2\cdot \text{M}3 - \text{P}5$             & 1  & m2 & C$^{\sharp}$ & 25/24  & 71  \\
$8^{ve} + \text{P}5 - 2\cdot \text{M}3$    & 11 & M7 & B            & 48/25  & 1129\\
\end{tabular}
\end{center}
\item 
m3:Just =  75/64 : 6/5   = 125/128\\
M6:Just = 125/75 : 5/3   = 128/125\\
m2:Just =  25/24 : 16/15 = 125/128\\
M7:Just =  48/25 : 15/8  = 128/125\\
\item Three Just major thirds don't make exactly an octave.
\end{enumerate}
\end{question}
\begin{question}{Problem 7}
\begin{enumerate}[a.]
\item
Cents above $f_1$ for the harmonics of $f_1$ and $f_1+\text{M}3$:\\
\begin{center}
\begin{tabular}{|c || c | c | c | c | c | c | c | c | c | c|}                     \hline
Note              & 1   & 2    & 3    & 4    & 5    & 6    & 7    & 8    & 9    & 10  \\ \hline
$f_1$             & 0   & 1200 & 1902 & 2400 & 2786 & 3102 & 3369 & 3600 & 3804 & 3986\\ \hline
$f_1 + \text{M}3$ & 400 & 1600 & 2302 & 2800 & 3186 & 3502 & 3769 & 4000 & 4204 & 4386\\ \hline
\end{tabular}
\end{center}
This data was collected with the aid of Excel, using the cents to Hz conversion formula. 
Also, a major third is 400 cents in 12-tet.
\item Beat frequency will be lowest when the cents difference is lowest and the harmonic
number is lowest. Therefore, the lowest beat frequency will occur between the 5th harmonic
of $f_1$ and the 4th harmonic of $f_1 + \text{M}3$ (difference of 14 cents.) The next
lowest beat frequency is between the 10th harmonic of $f_1$ and the 8th harmonic of 
$f_1 + \text{M}3$ (difference of 14 cents). Again, you know that one has a lower
beat frequency than the other because it also depends on the harmonic number. 
\item Assume the 1st harmonic for $f_1$ is 440 Hz. 
The 5th harmonic of $f_1$ is $5*400 \text{Hz} = 2200 \text{Hz}$. The 4th harmonic
of $f_1 + \text{M}3$ is $4*2^{4/12}*440 \text{Hz} = 2217.6 \text{Hz}$. The difference of
17.6 Hz is the beat frequency.

The 10th harmonic of $f_1$ is $10*440 \text{Hz} = 4400 \text{Hz}$. The 8th harmonic
of $f_1 + \text{M}3$ is $8*2^{4/12}*440 \text{Hz} = 4434.9 \text{Hz}$. The difference of
34.9 Hz is the beat frequency.
\end{enumerate}
\end{question}
\begin{question}{Problem 8}
The chord A$_3$C$_4^\sharp$E$_4$ in just intonation.

To find the beat frequency, we find the frequency of the notes and give the 
difference.

A$_3$ is $440/2$ Hz = 220 Hz. C$_4^\sharp$ is a M3 above A$_3$, so it's frequency is
$\frac{5}{4}220$ Hz $= 275$ Hz. E$_4$ is a P5 above A$_3$, so it's frequency is
$\frac{3}{2}220$ Hz $= 330$ Hz. The difference between the frequencies is 55 Hz 
between the first two notes, and 55 Hz between the second two notes, and 110 Hz
between the outer two notes. You will notice the 55 Hz beating the most. $T=\frac{1}{f}$
so the period is $\frac{1}{55 \text{Hz}} = 0.1818 \text{s}$.
\end{question}
\end{document}
