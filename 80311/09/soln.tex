\documentclass[11pt,letterpaper]{article}

\usepackage{amsmath}
\usepackage{amssymb}
\usepackage{fancyhdr}

\oddsidemargin0cm
\topmargin-2cm
\textwidth16.5cm
\textheight23.5cm

\newcommand{\question}[1] {\vspace{.25in} \hrule\vspace{0.5em}
\noindent{\bf #1} \vspace{0.5em}
\hrule \vspace{.10in}}
\renewcommand{\part}[1] {\vspace{.10in} {\bf (#1)}}

\newcommand{\myname}{Karan Sikka}
\newcommand{\myandrew}{ksikka@cmu.edu}
\newcommand{\myhwnum}{06}

\setlength{\parindent}{0pt}
\setlength{\parskip}{5pt plus 1pt}

\pagestyle{fancyplain}
\lhead{\fancyplain{}{\textbf{HW\myhwnum}}}
\rhead{\fancyplain{}{\myname\\ \myandrew}}
\chead{\fancyplain{}{80-311}}

\begin{document}

\medskip

\thispagestyle{plain}
\begin{center}                  % Center the following lines
{\Large 80-311 Assignment \myhwnum} \\
\myname \\
\myandrew \\
\today
\end{center}

\question{1.1}
$$ \mathtt{THM}(\varphi) \iff (\exists \triangle (\mathtt{PRF}(\triangle, \varphi))) $$

\question{1.2}
Self-reference lemma: \\
If $\varphi(x)$ is any formula in the language of ZF
with exactly the indicated free variable then there is a sentence $\psi$
such that:

$$ ZF \vdash (\psi \iff \varphi('\psi')) $$

Obtaining Godel's sentence G: \\
Let $\psi = G$ and $\varphi = \neg \mathtt{THM}$. Then we substitute into the self reference lemma to obtain:

$$ ZF \vdash (G \iff \neg \mathtt{THM}(G)) $$
(what is it?)

\question{1.3}

\question{2}

\question{3}




\end{document}

