\documentclass[11pt,letterpaper]{article}

\usepackage{amsmath}
\usepackage{amssymb}
\usepackage{fancyhdr}

\oddsidemargin0cm
\topmargin-2cm
\textwidth16.5cm
\textheight23.5cm

\newcommand{\question}[1] {\vspace{.25in} \hrule\vspace{0.5em}
\noindent{\bf #1} \vspace{0.5em}
\hrule \vspace{.10in}}
\renewcommand{\part}[1] {\vspace{.10in} {\bf (#1)}}

\newcommand{\myname}{Karan Sikka}
\newcommand{\myandrew}{ksikka@cmu.edu}
\newcommand{\myhwnum}{06}

\setlength{\parindent}{0pt}
\setlength{\parskip}{5pt plus 1pt}

\pagestyle{fancyplain}
\lhead{\fancyplain{}{\textbf{HW\myhwnum}}}
\rhead{\fancyplain{}{\myname\\ \myandrew}}
\chead{\fancyplain{}{80-311}}

\begin{document}

\medskip

\thispagestyle{plain}
\begin{center}                  % Center the following lines
{\Large 80-311 Assignment \myhwnum} \\
\myname \\
\myandrew \\
\today
\end{center}

\question{1}
First let us recall that a set is Inductive iff the empty set is in the set, and for all elements in the set, their successors are also in the set.

In this proof, we will assume the existence of two arbitrary inductive sets $x$ and $y$. Then we'll consider the set $x \cap y$ and show that it is inductive, by showing the empty set is its member, and for all its elements, their successors are also in the set.

Assume we have two arbitrary inductive sets $x$ and $y$. Consider the set $x \cap y$.
We know the empty set is a member of $x \cap y$ because the empty set is a member of all sets and the intersection of two sets is a set.

Now, given an arbitrary element $z$ of $x \cap y$, is its successor $S(z)$ in $x \cap y$?
By the definition of the binary intersection, if $z$ is a member of $x \cap y$, then it's a member of $x$ and it's a member of $y$ (and vice versa since the definition is a biconditional). We know $S(z)$ must be in $x$ since $x$ is inductive, and it must be in $y$ for the same reason. Since $S(z)$ is in both $x$ and $y$, it's in $x \cap y$.

Thus we've proved that the intersection of two inductive sets is inductive.


\question{2}
The empty set is in $N$ since every set contains the empty set. Now, assuming $x$ is an arbitrary element in $N$, the successor $S(x)$ is in $N$ by definition.
The previous facts satisfy the definition of an inductive set, so $N$ is inductive.

\question{3}
Assume an arbitrary set $a$ is inductive. We'll show that $N$ is a subset of $a$ by induction. The empty set is in $N$, and in $a$ since $a$ is an inductive set. Assuming an element $x \in N$ is in $a$, we know the successor $S(x)$ is also in $a$ since $a$ is inductive. Therefore all elements of $N$ are elements of $a$.




\end{document}

