\documentclass[11pt,letterpaper]{article}

\usepackage{amsmath}
\usepackage{amssymb}
\usepackage{fancyhdr}
\usepackage{tikz}

\oddsidemargin0cm
\topmargin-2cm
\textwidth16.5cm
\textheight23.5cm

\newcommand{\question}[1] {\vspace{.25in} \hrule\vspace{0.5em}
\noindent{\bf #1} \vspace{0.5em}
\hrule \vspace{.10in}}
\renewcommand{\part}[1] {\vspace{.10in} {\bf (#1)}}

\newcommand{\myname}{Karan Sikka}
\newcommand{\myandrew}{ksikka@cmu.edu}
\newcommand{\myhwnum}{11}

\setlength{\parindent}{0pt}
\setlength{\parskip}{5pt plus 1pt}

\pagestyle{fancyplain}
\lhead{\fancyplain{}{\textbf{HW\myhwnum}}}
\rhead{\fancyplain{}{\myname\\ \myandrew}}
\chead{\fancyplain{}{80-311}}

\begin{document}

\medskip

\thispagestyle{plain}
\begin{center}                  % Center the following lines
{\Large 80-311 Assignment \myhwnum} \\
\myname \\
\myandrew \\
\today
\end{center}

\question{1}
The variant of Turing's thesis states that a solvable puzzle
has a corresponding substitution puzzle where a solution to the latter puzzle
makes it easy to find the solution to the first.

The same type of reasoning used in Turing's original paper is applicable to this new thesis.
In the original thesis, Turing states that a function is effectively calculable if and only if there
exists a turing machine to compute it. He drew a connection between the "computable" and the "calculable".

The variant of the Turing's thesis works in the same way. He is claiming a connection between the "solvable"
and the "computable". However, the variant talks about a substitution puzzle that makes it "easy" to solve the other puzzle,
but he leaves the definition of "easy" as fuzzy, vague, and left up to the reader's interpretation.

So intuitively we can draw connections between the variant and the original thesis but the variant is not well formalized enough for us
to prove using Turings original argumentation.

\question{2.i}
Claim:\\
Let f: $a \rightarrow b$ and x and y be subsets of a, then:\\
$f[x \cup y] = f[x] \cup f[y]$

To prove this we show that the LHS is a subset of the RHS, and the RHS is a subset of the LHS.

Part 1: $LHS \subseteq RHS$\\
First, let an arbitrary element $t$ be a member of $f[x \cup y]$. Then by the defn of image,\\
we know that there exists $u$ such that $f(u) = t$ and $u \in (x \cup y)$.\\
By the defn of union, $u$ must be in either $x$ or $y$.\\
Then $t$ is in either $f[x]$ or $f[y]$, so it is in $f[x] \cup f[y]$, which is the RHS.\\
Therefore any element $t$ which is a member of the LHS is also a member of the RHS.\\
Then $LHS \subseteq RHS$

Part 2: $RHS \subseteq LHS$\\
Consider element $t$ in $f[x] \cup f[y]$.\\
Then there is an element $u$ either in $x$ or $y$ such that $f(u) = t$.\\
Then $u$ is in $x \cup y$. Then $t$ is in $f[x \cup y]$, so every element in the RHS is also in the LHS.
Then $RHS \subseteq LHS$

\question{2.ii}
We want to show that every element in $f[x \cap y]$ is also in $f[x] \cap f[y]$.\\
Consider an arbitrary element $t$ in $f[x \cap y]$.\\
By the definition of image, there exists a $u$ such that $f(u) = t$ and $u \in x \cap y$.\\
By the defn of binary intersection, $u$ is in both $x$ and $y$.\\
Then $t$ is in both $f[x]$ and $f[y]$, so $t \in f[x] \cap f[y]$, proving\\
that $LHS \subseteq RHS$.\\

\question{2.iii}
I conjecture that the formula is true when $f$ is injective.\\

If an element $t$ is in $f[x]$ and $f[y]$, then there is an element $u$ such that $f(u) = t$ and $u \in x$.\\
There is also an element $v$ such that $f(v) = t$ and $v \in y$.

If $f$ is injective, then $f(v) = t \wedge f(u) = t \implies v = u$.\\
Then we can say $v \in y => u \in y$, so $u \in x \cap y$, and $t \in f[x \cap y]$.

Fitch Proof of (ii):\\

\question{3.i}
Representability allows us to express THM in terms of first order logic ZF. Specifically,
it states that
$$THM(\phi) \iff  ZF \vdash thm('\phi')$$

\question{3.ii}
Let $m$ and $n$ be natural numbers.

Note: In this problem, $'m'$ is the set theoretic numeral for $m$.

Claims:\\
1. If $m = n$ then $ZF \vdash m=n$\\
2. If $\neg m = n$ then $ZF \vdash \neg 'm'='n'$

Proof of 1:\\

By induction. It holds true for the base case when $m$ and $n$ are 0, which
is represented by the empty set. For the inductive hypothesis, assume it holds
true for all naturals up to and including some fixed $m$ and $n$ such that $m = n$.

$'m+1' = {'m','m-1', ..., '1', '0'}$\\
$'n+1' = {'n','n-1', ..., '1', '0'}$

and since we know $m=n$ from the inductive hypothesis, then we make a simple substitution to obtain:

$'m+1' = {'n','n-1', ..., '1', '0'}$\\
$'n+1' = {'n','n-1', ..., '1', '0'}$

Since $'m+1'$ and $'n+1'$ are sets with all the same elements, they are in fact the same set theoretic numerals.

Proof of 2:\\

By the hint, we seek to prove that if $m<n$ then $ZF \vdash \neg m = n$.

Proceed by induction.
We assume $n$ is an arbitrary natural number larger than $m$.
(2) holds true for the base case when $m$ is 0, which
is represented by the empty set. $n$ will be presented by some set with elements which do not exist in $'m'$. Thus $\neg 'm' = 'n'$.

For the inductive hypothesis, assume (2) is true for all naturals up to and including some fixed $m$ such that $m < n$.

$'m+1' = {'m','m-1', ..., '1', '0'}$\\
$'n+1' = {'n','n-1', ..., '1', '0'}$

and since we know $m<n$ from the inductive hypothesis, then we make a simple substitution to obtain:

$'m+1' = {'m','m-1' ..., '1', '0'}$\\
$'n+1' = {'n',...,'m','m-1' ..., '1', '0'}$\\

Since $'n+1'$ contains elements not in $'m+1'$ the sets are not equal.

For all $m$ and $n$ in the naturals both claims 1 and 2 hold true.


\question{4}
I don't know









\end{document}

