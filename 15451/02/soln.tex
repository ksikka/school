\documentclass[11pt]{article}

\usepackage{amsmath}
\usepackage{amssymb}
\usepackage{fancyhdr}

\oddsidemargin0cm
\topmargin-2cm
\textwidth16.5cm
\textheight23.5cm

\newcommand{\question}[2] {\vspace{.25in} \hrule\vspace{0.5em}
\noindent{\bf #1: #2} \vspace{0.5em}
\hrule \vspace{.10in}}
\renewcommand{\part}[1] {\vspace{.10in} {\bf (#1)}}

\newcommand{\myname}{Karan Sikka}
\newcommand{\myandrew}{ksikka@cmu.edu}
\newcommand{\myhwnum}{2}

\setlength{\parindent}{0pt}
\setlength{\parskip}{5pt plus 1pt}
 
\pagestyle{fancyplain}
\lhead{\fancyplain{}{\textbf{HW\myhwnum}}}
\rhead{\fancyplain{}{\myname\\ \myandrew}}
\chead{\fancyplain{}{15-451}}

\begin{document}

\medskip

\thispagestyle{plain}
\begin{center}
{\Large 15-451 Assignment \myhwnum} \\
\myname \\
\myandrew \\
Recitation: A \\
\today \\
\end{center}

\question{1}{Max Stacks and Quacks}

\part{a} Queues from stacks:

To prove that a sequence of $n$ ops costs at most $3n$, we set up a hypothetical situation where we spend 3 coins on each operation,\\
impose a cost of 1 coin for each push and pop, and assuming this, show that the algorithm's bank balance is never negative.\\
Then we will know that the cost to run the algorithm will not exceed $3n$.

For an insert operation, the algorithm will spend 1 coin on pushing onto S1, and save 2 coins in the bank.\\

For a dump operation where the stack S1 has size $k$, the algorithm with spend 2 coins on each of the $k$ elements,\\
popping each element off S1 and pushing on to S2.

The only way to get an element onto S1 is via an insert, and the only way to get an element off of S1 is a dump.

Each element on the stack S1 must have contributed 2 coins to the bank, and those 2 coins will be used to get it popped off and pushed onto S1.\\
So we see that the funds saved from an insert are always sufficient to perform the dump.

For a remove operation, there are 2 cases. In one case, S1 is not empty, and we need only spend 1 out of 3 coins received on popping off an element.\\
In the other case, S1 is empty and a dump needs to be performed.\\
The dump is funded from money saved during inserts, as shown above.\\
Then we reduce to the problem to the first case where S1 is not empty, and we need only spend 1 out of 3 coins received on popping off an element.

We showed that the insert, dump, and remove operations can always be funded by money in the bank assuming we spend 3 coins on each operation.\\
Therefore $n$ operations cost at most $3n$.


\question{2}{Little Gauss's Formula}

\part{a} Recall {\em Little Gauss's formula}:

\part{b} Now, equation \ref{little-gauss} can be proven by induction as
follows:

\end{document}

