\documentclass[11pt]{article}

\usepackage{amsmath}
\usepackage{amssymb}
\usepackage{fancyhdr}

\oddsidemargin0cm
\topmargin-2cm
\textwidth16.5cm
\textheight23.5cm

\newcommand{\question}[2] {\vspace{.25in} \hrule\vspace{0.5em}
\noindent{\bf #1: #2} \vspace{0.5em}
\hrule \vspace{.10in}}
\renewcommand{\part}[1] {\vspace{.10in} {\bf (#1)}}

\newcommand{\myname}{Karan Sikka}
\newcommand{\myandrew}{ksikka@cmu.edu}
\newcommand{\myhwnum}{2}

\setlength{\parindent}{0pt}
\setlength{\parskip}{5pt plus 1pt}
 
\pagestyle{fancyplain}
\lhead{\fancyplain{}{\textbf{HW\myhwnum}}}
\rhead{\fancyplain{}{\myname\\ \myandrew}}
\chead{\fancyplain{}{15-451}}

\begin{document}

\medskip

\thispagestyle{plain}
\begin{center}
{\Large 15-451 Assignment \myhwnum} \\
\myname \\
\myandrew \\
Recitation: A \\
\today \\
\end{center}

\question{1}{Max Stacks and Quacks}

\part{a} Queues from stacks:

To prove that a sequence of $n$ ops costs at most $3n$, we set up a hypothetical situation where we spend 3 coins on each operation,\\
impose a cost of 1 coin for each push and pop, and assuming this, show that the algorithm's bank balance is never negative.\\
Then we will know that the cost to run the algorithm will not exceed $3n$.

For an insert operation, the algorithm will spend 1 coin on pushing onto S1, and save 2 coins in the bank.\\

For a dump operation where the stack S1 has size $k$, the algorithm with spend 2 coins on each of the $k$ elements,\\
popping each element off S1 and pushing on to S2.

The only way to get an element onto S1 is via an insert, and the only way to get an element off of S1 is a dump.

Each element on the stack S1 must have contributed 2 coins to the bank, and those 2 coins will be used to get it popped off and pushed onto S1.\\
So we see that the funds saved from an insert are always sufficient to perform the dump.

For a remove operation, there are 2 cases. In one case, S1 is not empty, and we need only spend 1 out of 3 coins received on popping off an element.\\
In the other case, S1 is empty and a dump needs to be performed.\\
The dump is funded from money saved during inserts, as shown above.\\
After the dump, we can reduce to the problem to the first case where S1 is not empty, and we need only spend 1 out of 3 coins received on popping off an element.

We showed that the insert, dump, and remove operations can always be funded by money in the bank assuming we spend 3 coins on each operation.\\
Therefore $n$ operations cost at most $3n$.

\part{b} Max-queue:

A max-queue is simply a queue that supports the additional operation of return-max.

We can implement a max-queue using two stacks as done in part a, and we know the push and pop operations are $O(1)$ amortized.

To implement the return-max function, we will add some data structures so that the max of S1 and S2 can each be found in constant time, and then we will take max(S1, S2) to find the max of the Q.

How to turn a Stack "S" into a Max Stack so that its max can be found in constant time:

\begin{quote}
Create another stack, call it M.
When pushing "e" onto S, also push max(e,max(M)) onto M. Max(M) is always at top of M inductively.
This takes $O(1)$ time so it doesn't affect the cost of the push.

On a pop from S, also pop the max off of M.
This maintains the invariant that the element at the top of the M-stack is the current max of S.
This is also done in $O(1)$ time.

Now the max of S can always be found in constant time by looking at the top of the stack M.
\end{quote}

Notice that the dump is linear in the number of pushes and pops, so it's time is unaffected since the time of the pushes and pops is unaffected.

To return max, simply take the max((max(s1), max(s2)) which is computed in constant amortized time.


\question{2}{You Be the Adversary}

Say there are $n$ elements.

Next we notice the following about the counts of different types of elements:

\begin{enumerate}
\item The number of frees at the beginning is $n$, and the number of frees at the end is 0.
\item The number of tops at the beginning is $0$, and the number of tops at the end is 1.
\item The number of bottoms at the beginning is $0$, and the number of bottoms at the end is 1.
\item The number of middles at the beginning is $0$, and the number of middles at the end is $n-2$.
\end{enumerate}

The number of tops is not more than 1 because then there is ambiguity as to which is the larger top.\\
You would have to compare them to determine that and the lower would become a middle. A symmetrical argument
holds for the number of bottoms. The number of middles are the remaining elements. No element can be free
since that element could be changed to be a Max or Min and it wouldn't be detected by the algorithm.

Define a function $\Phi$ to be $3/2$ times the number of free elements plus the number of tops, plus the number of bottoms\\
At the beginning of the algorithm, $\Phi$ is $\frac{3}{2} n$ since all elements are free and there are no tops nor bottoms\\
At the end of the algorithm, $Phi$ is 2 since there is 1 top and 1 bottom.

We observe the following about the life-cycle of the elements:

\begin{enumerate}
\item At the start of the algorithm, all $n$ elements are free.
\item A free element compared with any other element will turn into either a top or bottom.
\item A top or bottom compared with any other element will either remain as it was, or turn into a middle.
\item A middle compared with any other element will remain a middle.
\end{enumerate}

Now we compute the change in $\Phi$ of each case

\begin{enumerate}
\item free =\> top or bottom: -3/2 + 1 = -1/2
\item top or bottom =\> top or bottom, or middle: 0 or -1
\item middle =\> middle: 0.
\end{enumerate}

Notice that two elements may convert on a comparison, so the only way to exceed a change of -1 is by having\\
two conversions of top or bottom to middles. This could only happen by comparing a top and a bottom, finding\\
that the bottom is greater than the top, and they both convert to middles.

An adversary can choose an input such that this case does not happen. On some input,\\
run the algorithm and observe which elements became top or bottom.\\
Note that an element starts free, then must become only top, or only bottom, and then it may become middle.\\
Then, add a scalar to all the top elements such that the smallest top element is larger than the largest bottom element.\\
Now we know that all tops are larger than all bottoms, and the ordering of tops amongst themselves in unchanged.\\
We can rerun the algorithm with the new input,\\
and we'll see that all tops before are still tops, and any top compared with any bottom, does not convert both\\
to middle, because the top will be greater than the bottom.

So we can find an input such that the change in $\Phi$ is at most -1, and that means $\Phi$ will
start at $\frac{3}{2} n$ and end at $2$, taking $\frac{3}{2} n - 2$ comparisons to get there.

\end{document}

