\documentclass[11pt]{article}

\usepackage{amsmath}
\usepackage{amssymb}
\usepackage{fancyhdr}

\oddsidemargin0cm
\topmargin-2cm
\textwidth16.5cm
\textheight23.5cm

\newcommand{\question}[2] {\vspace{.25in} \hrule\vspace{0.5em}
\noindent{\bf #1: #2} \vspace{0.5em}
\hrule \vspace{.10in}}
\renewcommand{\part}[1] {\vspace{.10in} {\bf (#1)}}

\newcommand{\myname}{Karan Sikka}
\newcommand{\myandrew}{ksikka@cmu.edu}
\newcommand{\myhwnum}{08}

\setlength{\parindent}{0pt}
\setlength{\parskip}{5pt plus 1pt}
 
\pagestyle{fancyplain}
\lhead{\fancyplain{}{\textbf{HW\myhwnum}}}
\rhead{\fancyplain{}{\myname\\ \myandrew}}
\chead{\fancyplain{}{15-451}}

\begin{document}

\medskip

\thispagestyle{plain}
\begin{center}
{\Large 15-451 Assignment \myhwnum} \\
\myname \\
\myandrew \\
Collaborated with Dave Cummings and Sandeep Rao.\\
Recitation: A \\
\today \\
\end{center}

\question{1}{Streaming Medians}
\part{a}

\part{b}

\part{c}

\part{d}

\question{2}{Counting Substrings}

\question{3}{LDIS}
\part{a}
Keep a running max length of the duplicate initial string.

For i from 1 to $|s|/2$, compute the hash of $s_{0,i}$ and $s_{i,2*i}$.\\
If they are equal, update the max length of the duplicate initial string to be $i$.

Finally return the max length of the duplicate initial strings.

Note that computing the hash at each iteration takes constant time.

To deal with false positives, you can verify the solution at the end.

TODO deal with this rigorously

\part{b}

\begin{verbatim}
aaaacc


prefix traversal order:

a => yes
a a => yes
a a a => no
a a a a => no
a a a a c => no
a a a a c c => no

Keep running max length duplicate prefix

1. traverse a, then aa, then aaa...
2. each time, check if the following condition is true:

check if there exists another suffix prefix starting at 2*len(prefix) (assuming string is 1-indexed)

^{prefix}.*
.{len(prefix)*2}{prefix}.*

3. If the condition is true, update the max-length duplicate prefix to len(prefix)

\end{verbatim}

TODO polish the above.














\end{document}





