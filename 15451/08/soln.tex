\documentclass[11pt]{article}

\usepackage{amsmath}
\usepackage{amssymb}
\usepackage{fancyhdr}

\oddsidemargin0cm
\topmargin-2cm
\textwidth16.5cm
\textheight23.5cm

\newcommand{\question}[2] {\vspace{.25in} \hrule\vspace{0.5em}
\noindent{\bf #1: #2} \vspace{0.5em}
\hrule \vspace{.10in}}
\renewcommand{\part}[1] {\vspace{.10in} {\bf (#1)}}

\newcommand{\myname}{Karan Sikka}
\newcommand{\myandrew}{ksikka@cmu.edu}
\newcommand{\myhwnum}{08}

\setlength{\parindent}{0pt}
\setlength{\parskip}{5pt plus 1pt}
 
\pagestyle{fancyplain}
\lhead{\fancyplain{}{\textbf{HW\myhwnum}}}
\rhead{\fancyplain{}{\myname\\ \myandrew}}
\chead{\fancyplain{}{15-451}}

\begin{document}

\medskip

\thispagestyle{plain}
\begin{center}
{\Large 15-451 Assignment \myhwnum} \\
\myname \\
\myandrew \\
Collaborated with Dave Cummings and Sandeep Rao.\\
Recitation: A \\
\today \\
\end{center}

\question{1}{Streaming Medians}
\part{a}

\part{b}

\part{c}

\part{d}

\question{2}{Counting Substrings}

\question{3}{LDIS}
\part{a}

\textbf{Choosing a random prime}:
Let $|\Sigma|$ be the size of the alphabet. Note that it is upper-bounded by a constant.

Turn the string into a binary string by replacing each character with a binary string of $p$ bits where $p = lg(|\Sigma|)$.

If the size of the original string was $T$, it is now $t$ where $t = T lg(|\Sigma|)$ which is in the order of $T$ since log sigma is upper bounded by a constant.

Choose a random prime between 1 and $K = 5*p*t*ln(pt)$.

You can do this by picking a random integer, checking if it's prime,
and trying again if not. This algorithm is expected $O(log(K))$ which is $O(log(p) + log(t) + log(log(p) + log(t))$ which is in $O(log(t))$

\textbf{Modified Karp-Rabin}:
Now we will compute the Karp-Rabin hashes of each prefix and the string of the same length following it.
Starting at i = 1, compute h(s[0:i]), h(s[i:2i])
Increment i and compute the hashes again. Note that this takes constant time since $s[0:i] \implies s[0:i+1]$ and $s[i:2i] \implies s[i+1:2i+2]$ so to compute the hash we only have to do a constant time adjustment to the previously computed hash.

Repeat until $i > t/2$. Return the max $i$ for which the two hash computed at the $ith$ iteration are equal.

\textbf{Proof that Pr[false positive] \textless 1/2}:
Next we will show the probability of a false positive is less than 1/2.

Let the length of the string be $n$. 
Suppose for any fixed locations $i$ and $2i$ the probability of an incorrect match is $\delta$.
Then by a union bound over $n/2$ locations Then the probability of any incorrect match is at most $n/2 * \delta$
We want $n/2 * \delta < 1/2$ or equivalently $\delta < 1/n$.

We make an incorrect match when the hash $a$ of one substring is equal to the hash $b$ of another, modulo some random prime $q$.
Formally, this occurs when $q | a-b$. $a-b$ has at most $p$ distinct prime divisors since it's a $p$-bit number and each prime divisor is at least 2.
If $q | a-b$, then $q$ must have been one of the $p$ prime divisors of $a-b$.

We want to choose a prime such that the chance that it's one of the $p$ prime divisors is less than $1/n$.
Choose $K$ large enough so that there are $pn$ primes between 2 and $K$.

If there are $\pi(x)$ primes between one and $x$, then $\pi(x) \geq \frac{7}{8} \frac{n}{ln(n)}$.
Thus we want to choose K such that
$\pi(K) \geq \frac{7}{8} \frac{K}{ln(K)} > pt$.

Setting $K = 5*p*t*ln(pt)$ acheives this result.




\part{b}

Construct a suffix tree for $s$ in linear time.
We will preprocess the suffix tree by running two $O(n)$ DFSs.

Do a DFS where you keep track of the length of the prefix at a node, called L,
cache the index of the last occurence of P starting on or before index L+1.

Ask your children, what's your last occurence, and since your L+1 is strictly greater than my L+1, I'll filter out the invalid answers and still have the correct last occurence before L+1.



\begin{verbatim}


a => yes
a a => yes
a a a => no
a a a a => no
a a a a c => no
a a a a c c => no

Keep running max length duplicate prefix

1. traverse a, then aa, then aaa...
2. each time, check if the following condition is true:

check if there exists another suffix prefix starting at 2*len(prefix) (assuming string is 1-indexed)

^{prefix}.*
.{len(prefix)*2}{prefix}.*

3. If the condition is true, update the max-length duplicate prefix to len(prefix)

\end{verbatim}

TODO polish the above.














\end{document}





