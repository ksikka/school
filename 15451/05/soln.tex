\documentclass[11pt]{article}

\usepackage{amsmath}
\usepackage{amssymb}
\usepackage{fancyhdr}

\oddsidemargin0cm
\topmargin-2cm
\textwidth16.5cm
\textheight23.5cm

\newcommand{\question}[2] {\vspace{.25in} \hrule\vspace{0.5em}
\noindent{\bf #1: #2} \vspace{0.5em}
\hrule \vspace{.10in}}
\renewcommand{\part}[1] {\vspace{.10in} {\bf (#1)}}

\newcommand{\myname}{Karan Sikka}
\newcommand{\myandrew}{ksikka@cmu.edu}
\newcommand{\myhwnum}{05}

\setlength{\parindent}{0pt}
\setlength{\parskip}{5pt plus 1pt}
 
\pagestyle{fancyplain}
\lhead{\fancyplain{}{\textbf{HW\myhwnum}}}
\rhead{\fancyplain{}{\myname\\ \myandrew}}
\chead{\fancyplain{}{15-451}}

\begin{document}

\medskip

\thispagestyle{plain}
\begin{center}
{\Large 15-451 Assignment \myhwnum} \\
\myname \\
\myandrew \\
Collaborated with Dave Cummings and Sandeep Rao.\\
Recitation: A \\
\today \\
\end{center}

\question{1}{Cell Towers}

\question{2}{Eliminating Negative Edges}
\part{a}
\part{b}
\part{c}

\question{3}{Color Me Red, Color Me Blue}

For convenience, we will call a $(k,n-k)$ partition which minimizes split edges a ``minimizing k-partition''.

We will consider a function kPart as follows:\\

Given a node $n$, value $i (1 \leq i \leq k)$ and bit $b$,\\
The function returns NONE, or SOME $(A,B,S)$.\\

$\mathtt{kPart}(n,k,b) = \mathtt{NONE}$ signifies that no k-partition of vertices in the subtree rooted at $n$ exists, which happens when $(k > |n|)$. \\
$\mathtt{kPart}(n,k,b) = \mathtt{SOME}(A,B,S)$ signifies that $A,B$ is some minimizing k-partition the nodes of the subtree rooted at $n$ such that $n \in A \iff b = 1$, $S$ is the set of split edges\\


We create a memo table for the computation which we assume is available to the function.\\
The table can be accessed by $T[n][k][b]$ where $n$ is a node, $k$ is an int from 0 to $k$, $b$ is a bit.\\
We initialize the table with NULLs, and compute the values in the following recursive function.\\
Note that the table has $n*(k+1)*2$ cells.

The procedure to compute kPart(r,k,b) is as follows:
    % Check if value has already been computed.
    If $T[r][k][i] \neq \text{NULL}$,
       return $T[r][k][i]$
    % If k=0, A is empty, B is rest of nodes.
    If $k = 0$,
       $T[r][k][i] = ({}, {\mathtt{all_nodes}(b)}, {})$
    % If k > n, no set A can be formed...
    If $k > n$,
       $T[r][k][i] = \mathtt{DNE}$

    If both r->left and r->right are NULL,
        If $k=1$
            $T[r][k][i] = ({r}, {}, {})$
        Else
            % k > n and this case would have been caught earlier

    If only r->left is NULL,
        % There are four cases.
        % We will compute the number of split edges in each case,
        % take the case with minimum split edges
        % return the partitioning for that case.
        $T[r][k][i]$ = The $(A,B,S)$ with the minimum $|S|$ over the following answers computed for all possible colorings of r and r->right that might yield a minimizing k-partition:
          $r \in A, r->right \in A$ =>
            $A_r, B_r, S_r = \mathtt{kPart}(r->right,k-1,1)$
            answer is $({r} \cup A_r, B_r, S_r)$
          $r \in A, r->right \in B$ =>
            $A_r, B_r, S_r = \mathtt{kPark}(r->right,k-1,0)$
            answer is $({r} \cup A_r, B_r, S_r \cup {(r, r->right)})$
          $r \in B, r->right \in A$ =>
            $A_r, B_r, S_r = \mathtt{kPark}(r->right,k,1)$
            answer is $(A_r,{r} \cup B_r, S_r \cup {(r, r->right)})$
          $r \in B, r->right \in B$ => 1 + $T[r->right][k][0]$
            $A_r, B_r, S_r = \mathtt{kPark}(r->right,k,0)$
            answer is $(A_r,{r} \cup B_r, S_r)$

    If only r->right is NULL,
        Logic is symmetric to above case.

    %Either $r$ is in $A$ or $B$.
    %Either $r->left$ is NULL or in $A$ or in $B$.
    %Either $r->right$ is NULL or in $A$ or in $B$.

% how did you fund travels

    If r->left and r->right are not NULL,
        $T[r][k][i]$ = The $(A,B,S)$ with the minimum $|S|$ over the following answers computed for all possible colorings of r, r->right, and r->left that might yield a minimizing k-partition:
                Left and right subtrees can be partitioned in at most k+1 ways such that the sum of their A sets should be $k-1$ if $r$ in $A$ or $k$ otherwise.
                We must minimize the number of split edges accross all variations of A-set-size.

          $r \in A, r->left \in A, r->right \in A\\
              minimize |S| over all integral i,j \geq 0 st i+j = k-1$\\
                $A_l, B_l, S_l = \mathtt{kPart}(r->left,i,1)$\\
                $A_r, B_r, S_r = \mathtt{kPart}(r->right,j,1)$
                $answer is (A_l \cup A_r \cup {r},B_l \cup B_r,S_l \cup S_r)$

          $r \in A, r->left \in A, r->right \in B
              minimize |S| over all integral i,j \geq 0 st i+j = k-1
                    \mathtt{kPart}(r->left,i,1) + 1 + \mathtt{kPart}(r->right,j,0)$
                $answer is (A_l \cup A_r \cup {r},B_l \cup B_r,S_l \cup S_r \cup {(r, r->right)})$

          $r \in A, r->left \in B, r->right \in A
              minimize |S| over all integral i,j \geq 0 st i+j = k-1
                1 + \mathtt{kPart}(r->left,i,0) +     \mathtt{kPart}(r->right,j,1)$
                $answer is (A_l \cup A_r \cup {r},B_l \cup B_r,S_l \cup S_r \cup {(r, r->left)})$

          $r \in A, r->left \in B, r->right \in B
              minimize |S| over all integral i,j \geq 0 st i+j = k-1
                1 + \mathtt{kPart}(r->left,i,0) + 1 + \mathtt{kPart}(r->right,j,0)$
                $answer is (A_l \cup A_r \cup {r},B_l \cup B_r,S_l \cup S_r \cup {(r, r->left),(r, r->right)})$

          $r \in B, r->left \in A, r->right \in A
              minimize |S| over all integral i,j \geq 0 st i+j = k
                1 + \mathtt{kPart}(r->left,i,1) + 1 + \mathtt{kPart}(r->right,j,1)$
                $answer is (A_l \cup A_r,B_l \cup B_r \cup {r},S_l \cup S_r \cup {(r, r->left),(r, r->right)})$

          $r \in B, r->left \in A, r->right \in B
              minimize |S| over all integral i,j \geq 0 st i+j = k
                1 + \mathtt{kPart}(r->left,i,1) +     \mathtt{kPart}(r->right,j,0)$
                $answer is (A_l \cup A_r,B_l \cup B_r \cup {r},S_l \cup S_r \cup {(r, r->left)})$

          $r \in B, r->left \in B, r->right \in A
              minimize |S| over all integral i,j \geq 0 st i+j = k
                    \mathtt{kPart}(r->left,i,0) + 1 + \mathtt{kPart}(r->right,j,1)$
                $answer is (A_l \cup A_r,B_l \cup B_r \cup {r},S_l \cup S_r \cup {(r, r->right)})$

          $r \in B, r->left \in B, r->right \in B
              minimize |S| over all integral i,j \geq 0 st i+j = k
                    \mathtt{kPart}(r->left,i,0) +     \mathtt{kPart}(r->right,j,0)$
                $answer is (A_l \cup A_r,B_l \cup B_r \cup {r},S_l \cup S_r)$

        Then add $r$ to the approprate set based on the case which was the min,
        adding $(r, r->right)$ as a split edge if necessary,
        adding $(r, r->left)$ as a split edge if necessary.

   return $T[r][k][i]$


Note that at each node we do at most $O(k)$ work assuming the table is computed.
For $n$ nodes there is at most $O(nk) = O(n^2)$ work.

The cells are never recomputed and each cell is a constant time computation given the other cells. Populating the table takes $O(nk)$ work.


\end{document}

