\documentclass[11pt]{article}

\usepackage{amsmath}
\usepackage{amssymb}
\usepackage{fancyhdr}

\oddsidemargin0cm
\topmargin-2cm
\textwidth16.5cm
\textheight23.5cm

\newcommand{\question}[2] {\vspace{.25in} \hrule\vspace{0.5em}
\noindent{\bf #1: #2} \vspace{0.5em}
\hrule \vspace{.10in}}
\renewcommand{\part}[1] {\vspace{.10in} {\bf (#1)}}

\newcommand{\myname}{Karan Sikka}
\newcommand{\myandrew}{ksikka@cmu.edu}
\newcommand{\myhwnum}{05}

\setlength{\parindent}{0pt}
\setlength{\parskip}{5pt plus 1pt}
 
\pagestyle{fancyplain}
\lhead{\fancyplain{}{\textbf{HW\myhwnum}}}
\rhead{\fancyplain{}{\myname\\ \myandrew}}
\chead{\fancyplain{}{15-451}}

\begin{document}

\medskip

\thispagestyle{plain}
\begin{center}
{\Large 15-451 Assignment \myhwnum} \\
\myname \\
\myandrew \\
Collaborated with Dave Cummings and Sandeep Rao.\\
Recitation: A \\
\today \\
\end{center}

\question{1}{Cell Towers}

\question{2}{Eliminating Negative Edges}
\part{a}
\part{b}
\part{c}

\question{3}{Color Me Red, Color Me Blue}

Create a memo table $T$ indexed by a node, a value from 1 to k, and a bit indicating whether or not the node given by the first index is in set A.
We initialize the table with NULLs.

Given a node $n$, value $i (1 \leq i \leq k)$ and bit $b$,\\
the table cell contains either NULL or some $(A,B,S)$ where $A,B$ partition the nodes of the subtree rooted at $n$ such that $|A| = k$, $n \in A \iff b = 1$, $S$ is the set of split edges, and there does not exist a partition with fewer split edges. 

% add a note about the size of the table.

Now to calculate the $(k, n-k)$ partition of a binary tree $G$, we do the following case analysis:\\

Call the root $r$.
If $k = 0$, return ({}, {all nodes in G})
Either $r$ is in $A$ or $B$.
Either $r->left$ is NULL or in $A$ or in $B$.
Either $r->right$ is NULL or in $A$ or in $B$.
If both left and right are NULL, then if $k=1$ return ({G}, {}, {}) 
If only left is NULL, take the min over |S| of the following cases:
      $r \in A, r->right \in A$ given by $T[r->right][k-1][1]$    
  1 + $r \in A, r->right \in B$ given by $T[r->right][k-1][0]$    
      $r \in B, r->right \in A$ given by $T[r->right][k][1]$    
  1 + $r \in B, r->right \in B$ given by $T[r->right][k][0]$    

Then add $r$ to the approprate set based on the case which was the min,
adding $(r, r->right)$ as a split edge if necessary.

Symmetric logic if only right is NULL.

Now we have the case where left and right are not null. In this case,
left and right subtrees can be partitioned and the sum of their A sets should be $k-1$ if $r$ in $A$ and $k$ otherwise.
We must minimize the number of split edges accross all variations of A-set-size.





\end{document}

