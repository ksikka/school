\documentclass[11pt]{article}

\usepackage{amsmath}
\usepackage{amssymb}
\usepackage{fancyhdr}

\oddsidemargin0cm
\topmargin-2cm
\textwidth16.5cm
\textheight23.5cm

\newcommand{\question}[2] {\vspace{.25in} \hrule\vspace{0.5em}
\noindent{\bf #1: #2} \vspace{0.5em}
\hrule \vspace{.10in}}
\renewcommand{\part}[1] {\vspace{.10in} {\bf (#1)}}

\newcommand{\myname}{Karan Sikka}
\newcommand{\myandrew}{ksikka@cmu.edu}
\newcommand{\myhwnum}{07}

\setlength{\parindent}{0pt}
\setlength{\parskip}{5pt plus 1pt}
 
\pagestyle{fancyplain}
\lhead{\fancyplain{}{\textbf{HW\myhwnum}}}
\rhead{\fancyplain{}{\myname\\ \myandrew}}
\chead{\fancyplain{}{15-451}}

\begin{document}

\medskip

\thispagestyle{plain}
\begin{center}
{\Large 15-451 Assignment \myhwnum} \\
\myname \\
\myandrew \\
Collaborated with Dave Cummings and Sandeep Rao.\\
Recitation: A \\
\today \\
\end{center}

\question{1}{A Densely-Knit Community}
\part{a}
\part{b}
\part{c}
\part{d}

\question{2}{Large + Dense = Difficult}

\question{3}{A Well-Separated Problem}
\part{a}
The problem is in NP because there exists a poly-time verifier as follows:

Define the proof of the solution as a K-element subset of X. We compute distances
between each pair of distinct points and check that that they are greater than or equal to $\Delta$.
If all pairs satisfy the condition, then we verify that this is a solution. Otherwise it is not.

The Well-Separated problem is in NP-Hard because the independent set decision problem which is NP-hard reduces to it.
The reduction is as follows:

\textbf{Independent set}\\
Given a graph $G = (V,E)$ and integer $k$,\\
we want to output YES\\
if there exists a set of vertices of size $k$\\
such that no two of them are adjacent.

To craft our input to the Well-Separated oracle,\\
we construct a set $X$ from the vertices of $G$,\\
let $K = k$,\\
let $\Delta = 1.25$,\\
for all $i \in V$ let $d(i,i) = 0$,\\
for all $i,j \in V : i \neq j \wedge (i,j) \in E$ let $d(i,j) = 1$\\
for all $i,j \in V : i \neq j \wedge (i,j) \notin E$ let $d(i,j) = 1.5$

We pass this input to the well separated problem, which will return YES\\
if there exists a set of elements of size $K$ where all distances are greater than 1.25.

Observe these elements in $X$ map directly to vertices in $V$ which are not adjacent\\
due to the way we constructed the input to the Well-Separated problem.

Also note that the construction of $d$ correctly obeys the triangle inequality, because\\
two of the shortest distances $(1 + 1)$ is still greater than the longest distance $(1.5)$.


\part{b}
We will present the algorithm, prove a condition about it's correctness, and show that it runs in poly time.

\textbf{Algorithm:}\\
Call one set with separation at least $\Delta^{*} / 2$ set $C$.

We will maintain a vector of all points which are potentially in $C$, initially containing all points.

We will also maintain a vector of points which we know are in $C$.

\begin{enumerate}
\item Pick a point $u$ potentially in $C$, and examine how far away all other points are from it.
\item Add $u$ to the set of known points $C$.
\item Remove $u$ from the set of points potentially in $C$.
\item Remove all points not at least $\Delta^{*} / 2$ from $u$ from the set of points potentially in $C$.
\end{enumerate}

Repeat for a total of $K$ iterations,\\
resulting in a set $C$ with $K$ points at least $\Delta^{*} / 2$ away from each other\\
since at each step we eliminate points which could invalidate this invariant.

Now we need a proof that we can perform this $K$ times, or in other words, we never run out of points potentially in $C$ from which to select at each iteration.

\textbf{Proof:}\\
We were allowed to assume there exists some set of $K$ points with separation $\Delta^{*}$.\\
For convenience, let's call these the optimal points.

Lets examine how many optimal points are eliminated from the potentially-in-C set in one iteration of the algorithm.\\
Claim: At most one optimal point is removed from the potentially-in-C set.\\
If we select optimal point $u$, we will see that all the other optimal points are at least $\Delta^{*}$ away from $u$, and they will not be eliminated from the potentially-in-C set in this iteration. $u$ will be the only optimal point removed.\\
If we select non-optimal point $u$, we will see that at most one optimal point is less than $\Delta^{*} / 2$ away from $u$.

This due to the triangle inequality. Consider a nonoptimal point $u$, the nearest optimal point $v$, and any other optimal point $w$.

$$\Delta^{*} \leq d(v,w) \leq d(v,u) + d(u,w)$$

Since $v$ no farther from  $u$ than $w$, $d(v,u)$ may be less than $\Delta^{*} / 2$, but $d(u,w)$ will certainly be at least $\Delta^{*} / 2$.

Therefore at most one optimal point is removed from the potentially-in-C set in each iteration of the algorithm.

We know there were $K$ optimal points to start out with in the potentially-in-C set.

Therefore the algorithm can be run at least $K$ iterations before running out of points to select.

\textbf{Runtime:}\\
The algorithm runs $K$ iterations, each iteration takes $O(|X|)$ work, so the runtime is $O(K*|X|)$.

\part{c}
Modify the algorithm from B to return NONE if it runs out of points potentially in $C$ that are not already in $C$.\\
Observe that the optimum separation delta may only be one of $\binom{|X|}{2}$ values: the distances $d(i,j) : i \neq j$.\\
Sort the values from highest to lowest, and run the algorithm with these values.\\
Return the none-NONE answer corresponding to the input with highest $\Delta^{*}$

Sorting takes $O(|X|^{2} * log(|X|^2))$ and the rest takes $O(K*|X|*|X|^2)$. The limiting step is $O(K*|X|^3)$

\end{document}





