\documentclass[11pt,letterpaper]{article}

\usepackage{amsmath}
\usepackage{amssymb}
\usepackage{fancyhdr}

\oddsidemargin0cm
\topmargin-2cm
\textwidth16.5cm
\textheight23.5cm

\newcommand{\question}[1] {\vspace{.25in} \hrule\vspace{0.5em}
\noindent{\bf #1} \vspace{0.5em}
\hrule \vspace{.10in}}
\renewcommand{\part}[1] {\vspace{.10in} {\bf (#1)}}

\newcommand{\myname}{Karan Sikka}
\newcommand{\myandrew}{ksikka@cmu.edu}
\newcommand{\myhwnum}{01}

\setlength{\parindent}{0pt}
\setlength{\parskip}{5pt plus 1pt}

\pagestyle{fancyplain}
\lhead{\fancyplain{}{\textbf{HW\myhwnum}}}
\rhead{\fancyplain{}{\myname\\ \myandrew}}
\chead{\fancyplain{}{15-451}}

\begin{document}

\medskip

\thispagestyle{plain}
\begin{center}                  % Center the following lines
{\Large 15-451 Assignment \myhwnum} \\
\myname \\
\myandrew \\
\today
\end{center}

\question{1a.}
Compute $n^3/4$ in constant time.\\
Use DeterministicSelect to select the $n^{3/4}$th largest number in $O(n)$ time.\\
Then filter out the elements greater than or equal to it in $O(n)$ time.\\
Now sort the $n^{3/4}$ numbers using mergesort in $O(n^{3/4}log(n^{3/4}))$ time.\\
The algorithm seems to be dominated by the latter expression, but it can be reduced to $O(n)$ as follows:

     $O(n^{3/4}log(n^{3/4}))$ \\
$\leq O(\frac{3}{4} n^{3/4} log(n))$

Notice that $O(n^{1/4}) \geq O(log(n))$ so we can make the following substitution:

$\leq O(\frac{3}{4} n^{3/4} n^{1/4})$ \\
$\leq O(n)$

\question{1b.}
Pair up the elements, and for each pair, compare the elements. ($\frac{n}{2}$ comparisons) \\
Call the larger element a ``winner" and the smaller element a ``loser". \\
Among the $\frac{n}{2}$ winners, find the max by going one by one keeping track of the max so far. ($\frac{n}{2}-1$ comparisons) \\
The minus one term is because you don't have anything to compare the first element to - you assume it as the max at first. \\
This is the max of all the elements.

Among the $\frac{n}{2}$ losers, find the min by going one by one keeping track of the min so far. ($\frac{n}{2}-1$ comparisons) \\
This is the min of all the elements.

The sum of all the comparisons is $(\frac{n}{2}) + (\frac{n}{2}-1) + (\frac{n}{2}-1) = \frac{3n}{2}-2 $

\question{1c.}
In this proof, I will:

1. Show that if the graph of comparisons contains disconnected subgraphs,\\
then you could change the inputs so that the answers to the comparisons remain the same,\\
but the global ordering of the elements changes such that the median changes, making the algorithm incorrect.

2. Show that with only $n-2$ comparisons, you must have multiple disconnected chains in the comparison graph.

Proof:

1. Construct a graph of the comparisons such that the vertices are the inputs and there is a directed edge
between vertices if the vertices are compared by the algorithm.

Notice that if the graph is disconnected, then you can add a scalar to all the numbers in a subgraph not containing the median, \\
without altering the answers to the comparisons.

Furthermore, you can find a scalar to add to this subgraph which will cause the median to be in this subgraph instead of that subgraph.
This is true because adding a scalar to elements can shift them left or right arbitrarily in the ordering.

2. A path of $n$ vertices must have $n-1$ edges. So with $n-2$ comparisons (edges), no path can cover all vertices.\\
The graph must be disconnected. By subclaim 1, an algorithm which performs $n-2$ comparisons cannot be correct on all inputs.

\question{1d.}
Since all recurrences are divide-and-conquer style, we can use the Master theorem to solve all of them.

A. $a=3, b=2, c=1, k=2.5$\\
$r = a/b^{k} = 3/2^{2.5} \approx .5303 < 1$\\
Then the recurrence is in $\Theta(n^k) = \Theta(n^{2.5})$

B. $a=4, b=2, c=1, k=2$\\
$r = a/b^{k} = 4/2^{2} = 1$\\
Then the recurrence is in $\Theta(n^k log(n)) = \Theta(n^{2} log(n))$.

C. $a=5, b=2, c=1, k=1.5$\\
$r = a/b^{k} = 5/2^{1.5} \approx 1.7678 > 1$\\
Then the recurrence is in $\Theta(n^{log_b(a)}) = \Theta(n^{log_2(5)})$.


\question{2}
Lorem ipsum


\end{document}

